
\Chapter[ON TRUE COMPOUND INTEREST]
{XIV}{On true Compound Interest and the Law of Organic Growth}

\First{Let} there be a quantity growing in such a way that
the increment of its growth, during a given time,
shall always be proportional to its own magnitude.
This resembles the process of reckoning interest on
money at some fixed rate; for the bigger the capital,
the bigger the amount of interest on it in a given
time.

Now we must distinguish clearly between two
cases, in our calculation, according as the calculation
is made by what the arithmetic books call ``simple
interest,'' or by what they call ``compound interest.''
For in the former case the capital remains fixed,
while in the latter the interest is added to the capital,
which therefore increases by successive additions.

\Paragraph{{\upshape(1)}~At simple interest.} Consider a concrete case.
Let the capital at start be~£$100$, and let the rate
of interest be $10$~per~cent.\ per~annum. Then the
increment to the owner of the capital will be £$10$
every year. Let him go on drawing his interest
every year, and hoard it by putting it by in a
\DPPageSep{147.png}{135}%
stocking, or locking it up in his safe. Then, if he
goes on for $10$~years, by the end of that time he will
have received $10$~increments of £$10$~each, or~£$100$,
making, with the original~£$100$, a total of~£$200$ in all.
His property will have doubled itself in $10$~years.
If the rate of interest had been $5$~per cent., he would
have had to hoard for $20$~years to double his property.
If it had been only $2$~per~cent., he would have had
to hoard for $50$~years. It is easy to see that if the
value of the yearly interest is $\dfrac{1}{n}$~of the capital, he
must go on hoarding for $n$~years in order to double
his property.

Or, if $y$ be the original capital, and the yearly
interest is~$\dfrac{y}{n}$, then, at the end of $n$~years, his property
will be
\[
y + n\dfrac{y}{n} = 2y.
\]

%[** TN: Several minor numerical errors below are corrected]
\Paragraph{{\upshape(2)}~At compound interest.} As before, let the owner
\Pagelabel{erratum0}%
begin with a capital of~£$100$, earning interest at the
rate of $10$~per~cent.\ per~annum; but, instead of
hoarding the interest, let it be added to the capital
each year, so that the capital grows year by year.
Then, at the end of one year, the capital will have
grown to~£$110$; and in the second year (still at~$10$\%)
this will earn £$11$~interest. He will start the third
year with~£$121$, and the interest on that will be
£$12$.~$2$\textit{s}.; so that he starts the fourth year with
£$133$.~$2$\textit{s}., and so on. It is easy to work it out, and
find that at the end of the ten years the total capital
\DPPageSep{148.png}{136}%
will have grown to £$259$.~$7$\textit{s}.~$6$\textit{d}. In fact, we see that
at the end of each year, each pound will have earned
$\tfrac{1}{10}$~of a pound, and therefore, if this is always added
on, each year multiplies the capital by~$\tfrac{11}{10}$; and if
continued for ten years (which will multiply by this
factor ten times over) will multiply the original
capital by~$\DPtypo{2.59375}{2.59374}$. Let us put this into symbols.
Put $y_0$ for the original capital; $\dfrac{1}{n}$~for the fraction
added on at each of the $n$~operations; and $y_n$ for the
value of the capital at the end of the $n$\textsuperscript{th}~operation.
Then
\[
y_n = y_0\left(1 + \frac{1}{n}\right)^n.
\]

But this mode of reckoning compound interest once
a year, is really not quite fair; for even during the
first year the~£$100$ ought to have been growing. At
the end of half a year it ought to have been at least~£$105$,
and it certainly would have been fairer had
the interest for the second half of the year been
calculated on~£$105$. This would be equivalent to
calling it $5$\%~per half-year; with $20$~operations, therefore,
at each of which the capital is multiplied by~$\tfrac{21}{20}$.
If reckoned this way, by the end of ten years the
capital would have grown to
\DPtypo{£$265$.~$8$\textit{s}.}
       {£$265$.~$6$\textit{s}.~$7$\textit{d}.}; for
\[
(1 + \tfrac{1}{20})^{20} = \DPtypo{2.654}{2.653}.
\]

But, even so, the process is still not quite fair; for,
by the end of the first month, there will be some
interest earned; and a half-yearly reckoning assumes
that the capital remains stationary for six months at
\DPPageSep{149.png}{137}%
a time. Suppose we divided the year into $10$~parts,
and reckon a one-per-cent.\ interest for each tenth of
the year. We now have $100$~operations lasting over
the ten years; or
\[
y_n = £100 \left( 1 + \tfrac{1}{100} \right)^{100};
\]
which works out to
\DPtypo{£$270$.~$8$\textit{s}.}
       {£$270$.~$9$\textit{s}.~$7\frac{1}{2}$\textit{d}.}

Even this is not final. Let the ten years be divided
into $1000$~periods, each of $\frac{1}{100}$~of a year; the interest
being $\frac{1}{10}$~per~cent.\ for each such period; then
\[
y_n = £100 \left( 1 + \tfrac{1}{1000} \right)^{1000};
\]
which works out to
\DPtypo{£$271$.~$14$\textit{s}.~$2\frac{1}{2}$\textit{d}.}
       {£$271$.~$13$\textit{s}.~$10$\textit{d}.}

Go even more minutely, and divide the ten years
into $10,000$ parts, each $\frac{1}{1000}$~of a year, with interest
at $\frac{1}{100}$~of $1$~per~cent. Then
\[
y_n = £100 \left( 1 + \tfrac{1}{10,000} \right)^{10,000};
\]
which amounts to
\DPtypo{£$271$.~$16$\textit{s}.~$4$\textit{d}.}
       {£$271$.~$16$\textit{s}.~$3\frac{1}{2}$\textit{d}.}

Finally, it will be seen that what we are trying to
find is in reality the ultimate value of the expression
$\left(1 + \dfrac{1}{n}\right)^n$, which, as we see, is greater than~$2$; and
which, as we take $n$~larger and larger, grows closer
and closer to a particular limiting value. However
big you make~$n$, the value of this expression grows
nearer and nearer to the figure
\[
2.71828\ldots
\]
a number \emph{never to be forgotten}.

Let us take geometrical illustrations of these things.
In \Fig{36}, $OP$~stands for the original value. $OT$~is
\DPPageSep{150.png}{138}%
the whole time during which the value is growing.
It is divided into $10$~periods, in each of which there is
an equal step up. Here $\dfrac{dy}{dx}$~is a constant; and if each
step up is $\frac{1}{10}$~of the original~$OP$, then, by $10$~such
steps, the height is doubled. If we had taken $20$~steps,
\Figure[2.5in]{150a}{36}
each of half the height shown, at the end the height
would still be just doubled. Or $n$~such steps, each
of $\dfrac{1}{n}$~of the original height~$OP$, would suffice to
double the height. This is the case of simple interest.
Here is $1$~growing till it becomes~$2$.

In \Fig{37}, we have the corresponding illustration of
the geometrical progression. Each of the successive
ordinates is to be $1 + \dfrac{1}{n}$, that is, $\dfrac{n+1}{n}$ times as high as
its predecessor. The steps up are not equal, because
each step up is now $\dfrac{1}{n}$~of the ordinate \emph{at that part} of
the curve. If we had literally $10$~steps, with $\left(1 + \frac{1}{10} \right)$
for the multiplying factor, the final total would be
\DPPageSep{151.png}{139}%
$(1 + \tfrac{1}{10})^{10}$ or $\DPtypo{2.593}{2.594}$~times the original~$1$. But if only
we take $n$ sufficiently large (and the corresponding
$\dfrac{1}{n}$ sufficiently small), then the final value $\left(1 + \dfrac{1}{n}\right)^n$ to
%[** TN: Minor imprecision in the value of e retained from the original.]
which unity will grow will be~$2.71828$.

\Figure[2.75in]{151a}{37}

\Paragraph{Epsilon.} To this mysterious number $2.7182818$
etc., the mathematicians have assigned as a symbol
the Greek letter~$\epsilon$ (pronounced \emph{epsilon}). All schoolboys
know that the Greek letter~$\pi$ (called \emph{pi}) stands
for $3.141592$ etc.; but how many of them know that
\emph{epsilon} means $2.71828$? Yet it is an even more
important number than~$\pi$!

What, then, is \emph{epsilon}?

Suppose we were to let $1$ grow at simple interest
till it became~$2$; then, if at the same nominal rate of
interest, and for the same time, we were to let $1$ grow
at true compound interest, instead of simple, it would
grow to the value \emph{epsilon}.

This process of growing proportionately, at every
instant, to the magnitude at that instant, some people
\DPPageSep{152.png}{140}%
call \emph{a logarithmic rate} of growing. Unit logarithmic
rate of growth is that rate which in unit time will
cause $1$ to grow to $2.718281$. It might also be
called the \emph{organic rate} of growing: because it is
characteristic of organic growth (in certain circumstances)
that the increment of the organism in a
given time is proportional to the magnitude of the
organism itself.

If we take $100$~per cent.\ as the unit of rate,
and any fixed period as the unit of time, then the
result of letting $1$ grow \emph{arithmetically} at unit rate,
for unit time, will be~$2$, while the result of letting $1$
grow \emph{logarithmically} at unit rate, for the same time,
will be $2.71828\ldots$\,.

\Paragraph{A little more about Epsilon.} We have seen that
\SetOddHead{The Law of Organic Growth}%
we require to know what value is reached by the
expression $\left(1 + \dfrac{1}{n}\right)^n$, when $n$ becomes indefinitely
great. Arithmetically, here are tabulated a lot of
values (which anybody can calculate out by the help
of an ordinary table of logarithms) got by assuming
$n = 2$; $n = 5$; $n = 10$; and so on, up to $n = 10,000$.
\begin{alignat*}{2}
&(1 + \tfrac{1}{2})^2             &&= 2.25.    \\
&(1 + \tfrac{1}{5})^5             &&= \DPtypo{2.489}{2.488}.   \\
&(1 + \tfrac{1}{10})^{10}         &&= 2.594.   \\
&(1 + \tfrac{1}{20})^{20}         &&= 2.653.   \\
&(1 + \tfrac{1}{100})^{100}       &&= \DPtypo{2.704}{2.705}.   \\
&(1 + \tfrac{1}{1000})^{1000}     &&= \DPtypo{2.7171}{2.7169}.  \\
&(1 + \tfrac{1}{10,000})^{10,000} &&= \DPtypo{2.7182}{2.7181}.
\end{alignat*}
\DPPageSep{153.png}{141}%

It is, however, worth while to find another way of
calculating this immensely important figure.

Accordingly, we will avail ourselves of the binomial
theorem, and expand the expression $\left(1 + \dfrac{1}{n}\right)^n$ in that
well-known way.

The binomial theorem\Pagelabel{binomtheo} gives the rule that
\begin{align*}
(a + b)^n &= a^n + n \dfrac{a^{n-1} b}{1!} + n(n - 1) \dfrac{a^{n-2} b^2}{2!} \\
  & \phantom{= a^n\ } + n(n -1)(n - 2) \dfrac{a^{n-3} b^3}{3!} + \text{etc}. \\
\intertext{Putting $a = 1$ and $b = \dfrac{1}{n}$, we get}
\left(1 + \dfrac{1}{n}\right)^n
  &= 1 + 1 + \dfrac{1}{2!} \left(\dfrac{n - 1}{n}\right) + \dfrac{1}{3!} \dfrac{(n - 1)(n - 2)}{n^2} \\
  &\phantom{= 1 + 1\ } + \dfrac{1}{4!} \dfrac{(n - 1)(n - 2)(n - 3)}{n^3} + \text{etc}.
\end{align*}

Now, if we suppose $n$ to become indefinitely great,
say a billion, or a billion billions, then $n - 1$, $n - 2$,
and $n - 3$, etc., will all be sensibly equal to~$n$; and
then the series becomes
\[
\epsilon = 1 + 1 + \dfrac{1}{2!} + \dfrac{1}{3!} + \dfrac{1}{4!} + \text{etc}.\ldots
\]

By taking this rapidly convergent series to as
many terms as we please, we can work out the sum to
any desired point of accuracy. Here is the working
for ten terms:
\DPPageSep{154.png}{142}%
\begin{center}
\begin{tabular}{@{}r<{\qquad}@{}l@{}}
                & $1.000000$ \\
dividing by~$1$ & $1.000000$ \\
dividing by~$2$ & $0.500000$ \\
dividing by~$3$ & $0.166667$ \\
dividing by~$4$ & $0.041667$ \\
dividing by~$5$ & $0.008333$ \\
dividing by~$6$ & $0.001389$ \\
dividing by~$7$ & $0.000198$ \\
dividing by~$8$ & $0.000025$ \\
dividing by~$9$ & $0.000002$ \\
\cline{2-2}
\multicolumn{2}{r@{}}{Total\quad $2.718281$} \\
\cline{2-2}
\end{tabular}
\end{center}

$\epsilon$ is incommensurable with~$1$, and resembles~$\pi$ in
being an interminable non-recurrent decimal.

\Paragraph{The Exponential Series.} We shall have need of yet
another series.

Let us, again making use of the binomial theorem,
expand the expression $\left(1 + \dfrac{1}{n}\right)^{nx}$, which is the same
as $\epsilon^x$ when we make~$n$ indefinitely great.
\begin{align*}
\epsilon^x
  &= 1^{nx} + nx \frac{1^{nx-1} \left(\dfrac{1}{n}\right)}{1!}
            + nx(nx - 1) \frac{1^{nx - 2} \left(\dfrac{1}{n}\right)^2}{2!} \\
  & \phantom{= 1^{nx}\ }
    + nx(nx - 1)(nx - 2) \frac{1^{nx-3} \left(\dfrac{1}{n}\right)^3}{3!}
    + \text{etc}.\\
  &= 1 + x + \frac{1}{2!} · \frac{n^2x^2 - nx}{n^2}
    + \frac{1}{3!} · \frac{n^3x^3 - 3n^2x^2 + 2nx}{n^3} + \text{etc}. \\
%\DPPageSep{155.png}{143}%
  &= 1 + x + \frac{x^2 -\dfrac{x}{n}}{2!}
    + \frac{x^3 - \dfrac{3x^2}{n} + \dfrac{2x}{n^2}}{3!} + \text{etc}.
\end{align*}

But, when $n$ is made indefinitely great, this simplifies down to the following:
\[
\epsilon^x
  = 1 + x + \frac{x^2}{2!} + \frac{x^3}{3!} + \frac{x^4}{4!} + \text{etc.}\dots
\]

This series is called \emph{the exponential series}.

The great reason why $\epsilon$ is regarded of importance
is that $\epsilon^x$ possesses a property, not possessed by any
other function of~$x$, that \emph{when you differentiate it
its value remains unchanged}\Pagelabel{unchanged}; or, in other words, its
differential coefficient is the same as itself. This can
be instantly seen by differentiating it with respect
to~$x$, thus:
\begin{DPalign*}
\frac{d(\epsilon^x)}{dx}
  &= 0 + 1 + \frac{2x}{1 · 2} + \frac{3x^2}{1 · 2 · 3} + \frac{4x^3}{1 · 2 · 3 · 4} \\
&\phantom{= 0 + 1 + \frac{2x}{1 · 2} + \frac{3x^2}{1 · 2 · 3}\ } + \frac{5x^4}{1 · 2 · 3 · 4 · 5} + \text{etc}.  \\
\lintertext{or}
  &= 1 + x + \frac{x^2}{1 · 2} + \frac{x^3}{1 · 2 · 3} + \frac{x^4}{1 · 2 · 3 · 4} + \text{etc}.,
\end{DPalign*}
which is exactly the same as the original series.

Now we might have gone to work the other way,
and said: Go to;\DPnote{** TN: [sic], presumed archaic expression} let us find a function of~$x$, such
that its differential coefficient is the same as itself.
Or, is there any expression, involving only powers %
\DPPageSep{156.png}{144}%
of $x$, which is unchanged by differentiation? Accordingly;
let us \emph{assume} as a general expression that
\begin{align*}
y &= A + Bx + Cx^2 + Dx^3 + Ex^4 + \text{etc}.,\\
\intertext{(in which the coefficients $A$, $B$, $C$, etc.\ will have to be
determined), and differentiate it.}
\dfrac{dy}{dx} &= B + 2Cx + 3Dx^2 + 4Ex^3 + \text{etc}.
\end{align*}

Now, if this new expression is really to be the same
as that from which it was derived, it is clear that
$A$ \emph{must} $=B$; that $C=\dfrac{B}{2}=\dfrac{A}{1· 2}$; that $D = \dfrac{C}{3} = \dfrac{A}{1 · 2 · 3}$;
that $E = \dfrac{D}{4} = \dfrac{A}{1 · 2 · 3 · 4}$, etc.

The law of change is therefore that\Strut
\[
y = A\left(1 + \dfrac{x}{1} + \dfrac{x^2}{1 · 2} + \dfrac{x^3}{1 · 2 · 3} + \dfrac{x^4}{1 · 2 · 3 · 4} + \text{etc}.\right).
\]

If, now, we take $A = 1$ for the sake of further
simplicity, we have
\[
y = 1 + \dfrac{x}{1} + \dfrac{x^2}{1 · 2} + \dfrac{x^3}{1 · 2 · 3} + \dfrac{x^4}{1 · 2 · 3 · 4} + \text{etc}.
\]

Differentiating it any number of times will give
always the same series over again.

If, now, we take the particular case of $A=1$, and
evaluate the series, we shall get simply
\begin{align*}
\text{when } x &= 1,\quad & y &= 2.718281 \text{ etc.};    & \text{that is, } y &= \epsilon;   \\
\text{when } x &= 2,\quad & y &=(2.718281 \text{ etc.})^2; & \text{that is, } y &= \epsilon^2; \\
\text{when } x &= 3,\quad & y &=(2.718281 \text{ etc.})^3; & \text{that is, } y &= \epsilon^3;
\end{align*}
\DPPageSep{157.png}{145}%
and therefore
\[
\text{when } x=x,\quad y=(2.718281 \text{ etc}.)^x;\quad\text{that is, } y=\epsilon^x,
\]
thus finally demonstrating that
\[
\epsilon^x = 1 + \dfrac{x}{1} + \dfrac{x^2}{1·2} + \dfrac{x^3}{1· 2· 3} + \dfrac{x^4}{1· 2· 3· 4} + \text{etc}.
\]

[\textsc{Note}.---\textit{How to read exponentials}. For the benefit
of those who have no tutor at hand it may be of use
to state that $\epsilon^x$ is read as ``\emph{epsilon to the eksth power};''
or some people read it ``\emph{exponential eks}.'' So $\epsilon^{pt}$ is
read ``\emph{epsilon to the pee-teeth-power}'' or ``\emph{exponential
pee tee}.'' Take some similar expressions:---Thus, $\epsilon^{-2}$ is
read ``\emph{epsilon to the minus two power}'' or ``\emph{exponential
minus two}.'' $\epsilon^{-ax}$ is read ``\emph{epsilon to the minus
ay-eksth}'' or ``\emph{exponential minus ay-eks}.'']

Of course it follows that $\epsilon^y$ remains unchanged if
differentiated with respect to~$y$. Also $\epsilon^{ax}$, which is
equal to $(\epsilon^a)^x$, will, when differentiated with respect
to~$x$, be $a\epsilon^{ax}$, because $a$ is a constant.


\Subsection{Natural or Naperian Logarithms.}
Another reason why $\epsilon$ is important is because it
was made by Napier, the inventor of logarithms, the
basis of his system. If $y$ is the value of $\epsilon^x$, then $x$
is the \emph{logarithm}, to the base~$\epsilon$, of~$y$. Or, if
\begin{DPalign*}
                  y &= \epsilon^x, \\
\lintertext{then} x &= \log_\epsilon y.
\end{DPalign*}

The two curves plotted in \Figs{38}{and}{39} represent
these equations.
\DPPageSep{158.png}{146}%

The points calculated are:
\begin{align*}
\text{For \textsc{Fig}.~38} \left\{
\begin{array}{|c||*{5}{c|}}
\hline
\Strut
\Td[c]{x} & \Td[c]{0} & \Td[l]{0.5}  & \Td[l]{1}    & \Td[l]{1.5}  & \Td[l]{2} \\
\hline
\Strut
\Td[c]{y} & \Td[c]{1} & \Td[l]{1.65} & \Td[l]{2.71} & \Td[l]{4.50} & \Td[l]{\DPtypo{7.69}{7.39}} \\
\hline
\end{array}
\right. \\
\text{For \textsc{Fig}.~39} \left\{
\begin{array}{|c||*{5}{c|}}
\hline
\Strut
\Td[c]{y} & \Td[c]{1} & \Td[l]{2}    & \Td[l]{3}    & \Td[l]{4}    & \Td[l]{8} \\
\hline
\Strut
\Td[c]{x} & \Td[c]{0} & \Td[l]{0.69} & \Td[l]{1.10} & \Td[l]{1.39} & \Td[l]{2.08} \\
\hline
\end{array}
\right.
\end{align*}%
%[** TN: The graphs below are reversed in the original; exchanged.]
\Figures{158b}{158a}{38}{39}\Pagelabel{erratum1}

It will be seen that, though the calculations yield
different points for plotting, yet the result is identical.
The two equations really mean the same thing.

As many persons who use ordinary logarithms,
which are calculated to base~$10$ instead of base~$\epsilon$, are
unfamiliar with the ``natural'' logarithms, it may be
worth while to say a word about them. The ordinary
rule that adding logarithms gives the logarithm of
the product still holds good; or
\[
\log_\epsilon a + \log_\epsilon b = \log_\epsilon ab.
\]
Also the rule of powers holds good;
\[
n × \log_\epsilon a = \log_\epsilon a^n.
\]
\DPPageSep{159.png}{147}%
But as $10$~is no longer the basis, one cannot multiply
by $100$ or~$1000$ by merely adding $2$ or~$3$ to the
index. One can change the natural logarithm to
the ordinary logarithm simply by multiplying it by
$0.4343$; or
\begin{DPalign*}
\log_{10} x &= 0.4343 × \log_{\epsilon} x, \\
\lintertext{and conversely,}
\log_{\epsilon} x &= 2.3026 × \log_{10} x.
\end{DPalign*}


%[** TN: Allowed this material to float to improve page breaks.]
\begin{table}[hp]
\centering
\textsc{A Useful Table of ``Naperian Logarithms''} \\
(Also called Natural Logarithms or Hyperbolic Logarithms)
\[
\begin{array}{c<{\quad}|>{\ }c>{\qquad\qquad}cr<{\quad}|>{\ }c}
\multicolumn{1}{c}{\text{\footnotesize Number}} &
\multicolumn{1}{c}{\footnotesize\log_{\epsilon}} &&
\multicolumn{1}{c}{\text{\footnotesize Number}} &
\multicolumn{1}{c}{\footnotesize\log_{\epsilon}}\DPnote{** TN: The original uses "Log"} \\
\Strut\PadTo[l]{1.1}{1}
    & 0.0000 &&      6 & 1.7918 \\
1.1 & 0.0953 &&      7 & 1.9459 \\
1.2 & 0.1823 &&      8 & 2.0794 \\
1.5 & 0.4055 &&      9 & 2.1972 \\
1.7 & 0.5306 &&     10 & 2.3026 \\
2.0 & 0.6931 &&     20 & 2.9957 \\
2.2 & 0.7885 &&     50 & 3.9120 \\
2.5 & 0.9163 &&    100 & 4.6052 \\
2.7 & 0.9933 &&    200 & 5.2983 \\
2.8 & 1.0296 &&    500 & 6.2146 \\
3.0 & 1.0986 &&  1,000 & 6.9078 \\
3.5 & 1.2528 &&  2,000 & \DPtypo{7.6010}{7.6009} \\
4.0 & 1.3863 &&  5,000 & 8.5172 \\
4.5 & 1.5041 && 10,000 & \DPtypo{9.2104}{9.2103} \\
5.0 & 1.6094 && 20,000 & 9.9035 \\
\end{array}
\]
\end{table}


\Subsection{Exponential and Logarithmic Equations.}\Pagelabel{expolo}
Now let us try our hands at differentiating certain
expressions that contain logarithms or exponentials.

Take the equation:
\[
y = \log_\epsilon x.
\]
First transform this into
\[
\epsilon^y = x,
\]
\DPPageSep{160.png}{148}%
whence, since the differential of $\epsilon^y$ with regard to~$y$ is
the original function unchanged (see \Pageref{unchanged}),
\[
\frac{dx}{dy} = \epsilon^y,
\]
and, reverting from the inverse to the original function,
\[
\frac{dy}{dx}
  = \frac{1}{\ \dfrac{dx}{dy}\ }
  = \frac{1}{\epsilon^y}
  = \frac{1}{x}.
\]

Now this is a very curious result.  It may be
written\Pagelabel{differlog}
\[
\frac{d(\log_\epsilon x)}{dx} = x^{-1}.
\]

Note that $x^{-1}$ is a result that we could never have
got by the rule for differentiating powers. That rule
(\Pageref[page]{multipow}) is to multiply by the power, and reduce the
power by~$1$. Thus, differentiating $x^3$ gave us~$3x^2$;
and differentiating $x^2$ gave~$2x^1$. But differentiating
$x^0$ does not give us $x^{-1}$~or~$0 × x^{-1}$, because $x^0$~is itself
$= 1$, and is a constant. We shall have to come back
to this curious fact that differentiating $\log_\epsilon x$ gives us
$\dfrac{1}{x}$ when we reach the chapter on integrating.

\tb

Now, try to differentiate
\begin{DPalign*}
                              y &= \log_\epsilon(x+a),\\
\lintertext{that is} \epsilon^y &= x+a;
\end{DPalign*}
we have $\dfrac{d(x+a)}{dy} = \epsilon^y$, since the differential of~$\epsilon^y$
remains~$\epsilon^y$.
\DPPageSep{161.png}{149}%
%
\BindMath{\begin{DPalign*}
\lintertext{\indent This gives}
\frac{dx}{dy} &= \epsilon^y = x+a; \\
\intertext{hence, reverting to the original function (see \Pageref{section:3}),
we get}
\frac{dy}{dx} &= \frac{1}{\;\dfrac{dx}{dy}\;} = \frac{1}{x+a}.
\end{DPalign*}\Pagelabel{differ2}%
\tb
\begin{DPalign*}
\lintertext{\indent Next try}
y &= \log_{10} x.
\end{DPalign*}}

First change to natural logarithms by multiplying
by the modulus $0.4343$. This gives us
\begin{DPalign*}
y &= 0.4343 \log_\epsilon x; \\
\lintertext{whence}
\frac{dy}{dx} &= \frac{0.4343}{x}.
\end{DPalign*}

\tb

The next thing is not quite so simple. Try this:\Pagelabel{diffexp}
\[
y = a^x.
\]

Taking the logarithm of both sides, we get
\begin{DPalign*}
\log_\epsilon y &= x \log_\epsilon a, \\
\lintertext{or}
x  = \frac{\log_\epsilon y}{\log_\epsilon a}
  &= \frac{1}{\log_\epsilon a} × \log_\epsilon y.
\end{DPalign*}

Since $\dfrac{1}{\log_\epsilon a}$ is a constant, we get
\[
\frac{dx}{dy}
  = \frac{1}{\log_\epsilon a} × \frac{1}{y}
  = \frac{1}{a^x × \log_\epsilon a};
\]
hence, reverting to the original function.
\[
\frac{dy}{dx} = \frac{1}{\;\dfrac{dx}{dy}\;} = a^x × \log_\epsilon a.
\]
\DPPageSep{162.png}{150}%

We see that, since
\[
\frac{dx}{dy} × \frac{dy}{dx} =1\quad\text{and}\quad
\frac{dx}{dy} = \frac{1}{y} × \frac{1}{\log_\epsilon a},\quad
\frac{1}{y} × \frac{dy}{dx} = \log_\epsilon a.
\]

We shall find that whenever we have an expression
such as $\log_\epsilon y =$ a function of~$x$, we always have
$\dfrac{1}{y}\, \dfrac{dy}{dx} =$ the differential coefficient of the function of~$x$,
so that we could have written at once, from
$\log_\epsilon y = x \log_\epsilon a$,
\[
\frac{1}{y}\, \frac{dy}{dx}
  = \log_\epsilon a\quad\text{and}\quad
\frac{dy}{dx} = a^x \log_\epsilon a.
%[ **"/" presumed][ **F1 - this note was placed after 1/y]
\]

\tb

Let us now attempt further examples.


\Examples.
(1) $y=\epsilon^{-ax}$. Let $-ax=z$; then $y=\epsilon^z$.
\[
\frac{dy}{dx} = \epsilon^z;\quad
\frac{dz}{dx} = -a;\quad\text{hence}\quad
\frac{dy}{dx} = -a\epsilon^{-ax}.
\]

Or thus:
\[
\log_\epsilon y = -ax;\quad
\frac{1}{y}\, \frac{dy}{dx} = -a;\quad
\frac{dy}{dx} = -ay = -a\epsilon^{-ax}.
\]

(2) $y=\epsilon^{\efrac{x^2}{3}}$. Let $\dfrac{x^2}{3}=z$; then $y=\epsilon^z$.
\[
\frac{dy}{dz} = \epsilon^z;\quad
\frac{dz}{dx} = \frac{2x}{3};\quad
\frac{dy}{dx} = \frac{2x}{3}\, \epsilon^{\efrac{x^2}{3}}.
\]

Or thus:
\[
\log_\epsilon y = \frac{x^2}{3};\quad
\frac{1}{y}\, \frac{dy}{dx} = \frac{2x}{3};\quad
\frac{dy}{dx} = \frac{2x}{3}\, \epsilon^{\efrac{x^2}{3}}.
\]
\DPPageSep{163.png}{151}%

(3) $y = \epsilon^{\efrac{2x}{x+1}}$.
\begin{DPalign*}
\log_\epsilon y &= \frac{2x}{x+1},\quad
\frac{1}{y}\, \frac{dy}{dx} = \frac{2(x+1)-2x}{(x+1)^2}; \\
\lintertext{hence}
\frac{dy}{dx} &= \frac{2}{(x+1)^2} \epsilon^{\efrac{2x}{x+1}}.
\end{DPalign*}

Check by writing $\dfrac{2x}{x+1}=z$.

(4) $y=\epsilon^{\sqrt{x^2+a}}$.\quad $\log_\epsilon y=(x^2+a)^{\efrac{1}{2}}$.
\[
\frac{1}{y}\, \frac{dy}{dx} = \frac{x}{(x^2+a)^{\efrac{1}{2}}}\quad\text{and}\quad
\frac{dy}{dx} = \frac{x × \epsilon^{\sqrt{x^2+a}}}{(x^2+a)^{\efrac{1}{2}}}.
\]
\DPchg{(}{}For if $(x^2+a)^{\efrac{1}{2}}=u$ and $x^2+a=v$, $u=v^{\efrac{1}{2}}$,
\[
\frac{du}{dv} = \frac{1}{{2v}^{\efrac{1}{2}}};\quad
\frac{dv}{dx} = 2x;\quad
\frac{du}{dx} = \frac{x}{\DPtypo{(x^2+)a}{(x^2+a)}^{\efrac{1}{2}}}.\DPchg{)}{}
\]

Check by writing $\sqrt{x^2+a}=z$.

(5) $y=\log(a+x^3)$. Let $(a+x^3)=z$; then $y=\log_\epsilon z$.
\[
\frac{dy}{dz} = \frac{1}{z};\quad
\frac{dz}{dx} = 3x^2;\quad\text{hence}\quad
\frac{dy}{dx} = \frac{3x^2}{a+x^3}.
\]

(6) $y=\log_\epsilon\{{3x^2+\sqrt{a+x^2}}\}$. Let $3x^2 + \sqrt{a+x^2}=z$;
then $y=\log_\epsilon z$.
\begin{align*}
\frac{dy}{dz}
  &= \frac{1}{z};\quad \frac{dz}{dx} = 6x + \frac{x}{\sqrt{x^2+a}}; \\
\frac{dy}{dx}
  &= \frac{6x + \dfrac{x}{\sqrt{x^2+a}}}{3x^2 + \sqrt{a+x^2}}
   = \frac{x(1 + 6\sqrt{x^2+a})}{(3x^2 + \sqrt{x^2+a}) \sqrt{x^2+a}}.
\end{align*}
\DPPageSep{164.png}{152}%

(7) $y=(x+3)^2 \sqrt{x-2}$.
\begin{align*}
\log_\epsilon y
  &= 2 \log_\epsilon(x+3)+ \tfrac{1}{2} \log_\epsilon(x-2). \\
\frac{1}{y}\, \frac{dy}{dx}
  &= \frac{2}{(x+3)} + \frac{1}{2(x-2)}; \\
\frac{dy}{dx}
  &= (x+3)^2 \sqrt{x-2} \left\{\frac{2}{x+3} + \frac{1}{2(x-2)}\right\}.
\end{align*}

(8) $y=(x^2+3)^3(x^3-2)^{\efrac{2}{3}}$.
\begin{align*}
\log_\epsilon y
  &= 3 \log_\epsilon(x^2+3) + \tfrac{2}{3} \log_\epsilon(x^3-2); \\
\frac{1}{y}\, \frac{dy}{dx}
  &= 3 \frac{2x}{(x^2+3)} + \frac{2}{3} \frac{3x^2}{x^3-2}
   = \frac{6x}{x^2+3} + \frac{2x^2}{x^3-2}.
\end{align*}
\DPchg{(}{}For if $y=\log_\epsilon(x^2+3)$, let $x^2+3=z$ and $u=\log_\epsilon z$.
\[
\frac{du}{dz} = \frac{1}{z};\quad
\frac{dz}{dx} = 2x;\quad
\frac{du}{dx} = \frac{2x}{x^2+3}.
\]
Similarly, if $v=\log_\epsilon(x^3-2)$, $\dfrac{dv}{dx} = \dfrac{3x^2}{x^3-2}$\DPchg{)}{} and
\[
\frac{dy}{dx}
  = (x^2+3)^3(x^3-2)^{\efrac{2}{3}}
    \left\{ \frac{6x}{x^2+3} + \frac{2x^2}{x^3-2} \right\}.
\]

(9) $y=\dfrac{\sqrt[2]{x^2+a}}{\sqrt[3]{x^3-a}}$.
\begin{DPalign*}
\log_\epsilon y
  &= \frac{1}{2} \log_\epsilon(x^2+a) - \frac{1}{3} \log_\epsilon(x^3-a). \\
\frac{1}{y}\, \frac{dy}{dx}
  &= \frac{1}{2}\, \frac{2x}{x^2+a} - \frac{1}{3}\, \frac{3x^2}{x^3-a}
   = \frac{x}{x^2+a} - \frac{x^2}{x^3-a} \\
\lintertext{and}
\frac{dy}{dx}
  &= \frac{\sqrt[2]{x^2+a}}{\sqrt[3]{x^3-a}}
     \left\{ \frac{x}{x^2+a} - \frac{x^2}{x^3-a} \right\}.
\end{DPalign*}
\DPPageSep{165.png}{153}%

(10) $y=\dfrac{1}{\log_\epsilon x}$
\[
\frac{dy}{dx}
  = \frac{\log_\epsilon x × 0 - 1 × \dfrac{1}{x}}
         {\log_\epsilon^2 x}
  = -\frac{1}{x \log_\epsilon^2x}.
\]

(11) $y=\sqrt[3]{\log_\epsilon x} = (\log_\epsilon x)^{\efrac{1}{3}}$. Let $z=\log_\epsilon x$; $y=z^{\efrac{1}{3}}$.
\[
\frac{dy}{dz} = \frac{1}{3} z^{-\efrac{2}{3}};\quad
\frac{dz}{dx} = \frac{1}{x};\quad
\frac{dy}{dx} = \frac{1}{3x \sqrt[3]{\log_\epsilon^2 x}}.
\]

(12) $y=\left(\dfrac{1}{a^x}\right)^{ax}$.
\begin{DPalign*}
\log_\epsilon y
  &= ax(\log_\epsilon 1 - \log_\epsilon a^x) = -ax \log_\epsilon a^x. \\
\frac{1}{y}\, \frac{dy}{dx}
  &= -ax × a^x \log_\epsilon a - a \log_\epsilon a^x. \displaybreak[1] \\
\lintertext{and}
\frac{dy}{dx}
  &= -\left(\frac{1}{a^x}\right)^{ax}
      (x × a^{x+1} \log_\epsilon a + a \log_\epsilon a^x).
\end{DPalign*}

Try now the following exercises.


\Exercises{XII} (See \Pageref[page]{AnsEx:XII} for Answers.)
\begin{Problems}
\Item{(1)} Differentiate $y=b(\epsilon^{ax} -\epsilon^{-ax})$.

\Item{(2)} Find the differential coefficient with respect to~$t$
of the expression $u=at^2+2\log_\epsilon t$.

\Item{(3)} If $y=n^t$, find $\dfrac{d(\log_\epsilon y)}{dt}$.

\Item{(4)} Show that if $y=\dfrac{1}{b}·\dfrac{a^{bx}}{\log_\epsilon a}$,\quad $\dfrac{dy}{dx}=a^{bx}$.

\Item{(5)} If $w=pv^n$, find $\dfrac{dw}{dv}$.
\end{Problems}
\DPPageSep{166.png}{154}%

Differentiate
\begin{Problems}[2]
\Item{(6)} $y=\log_\epsilon x^n$.
\Item{(7)} $y=3\epsilon^{-\efrac{x}{x-1}}$.

\ResetCols{2}
\Item{(8)} $y=(3x^2+1)\epsilon^{-5x}$.
\Item{(9)} $y=\log_\epsilon(x^a+a)$.

\ResetCols{1}
\Item{(10)} $y=(3x^2-1)(\sqrt{x}+1)$.

\ResetCols{2}
\Item{(11)} $y=\dfrac{\log_\epsilon(x+3)}{x+3}$.
\Item{(12)} $y=a^x × x^a$.

\ResetCols{1}
\Item{(13)} It was shown by Lord Kelvin that the speed of
signalling through a submarine cable depends on the
value of the ratio of the external diameter of the core
to the diameter of the enclosed copper wire. If this
ratio is called~$y$, then the number of signals~$s$ that can
be sent per minute can be expressed by the formula
\[
s=ay^2 \log_\epsilon \frac{1}{y};
\]
where $a$ is a constant depending on the length and
the quality of the materials. Show that if these are
given, $s$~will be a maximum if $y=1 ÷ \sqrt{\epsilon}$.

\Item{(14)} Find the maximum or minimum of
\[
y=x^3-\log_\epsilon x.
\]

\Item{(15)} Differentiate $y=\log_\epsilon(ax\epsilon^x)$.

\Item{(16)} Differentiate $y=(\log_\epsilon ax)^3$.
\end{Problems}
\tb


\Section{The Logarithmic Curve.}

Let us return to the curve which has its successive
ordinates in geometrical progression, such as that
represented by the equation $y=bp^x$.

We can see, by putting $x=0$, that $b$ is the initial
height of~$y$.

Then when
\[
x=1,\quad y=bp;\qquad
x=2,\quad y=bp^2;\qquad
x=3,\quad y=bp^3,\quad \text{etc.}
\]
\DPPageSep{167.png}{155}%

Also, we see that $p$ is the numerical value of the
ratio between the height of any ordinate and that of
the next preceding it. In \Fig{40}, we have taken $p$
as~$\frac{6}{5}$; each ordinate being $\frac{6}{5}$~as high as the preceding
one.

\Figures{167a}{167b}{40}{41}

If two successive ordinates are related together
thus in a constant ratio, their logarithms will have a
constant difference; so that, if we should plot out
a new curve, \Fig{41}, with values of~$\log_\epsilon y$ as ordinates,
it would be a straight line sloping up by equal steps.
In fact, it follows from the equation, that
\begin{DPalign*}
\log_\epsilon y &= \log_\epsilon b + x · \log_\epsilon p, \\
\lintertext{whence}
\log_\epsilon y &- \log_\epsilon b = x · \log_\epsilon p.
\end{DPalign*}

Now, since $\log_\epsilon p$ is a mere number, and may be
written as $\log_\epsilon p=a$, it follows that
\[
\log_\epsilon \frac{y}{b}=ax,
\]
and the equation takes the new form
\[
y = b\epsilon^{ax}.
\]
\DPPageSep{168.png}{156}%


\Section{The Die-away Curve.}

If we were to take $p$ as a proper fraction (less than
unity), the curve would obviously tend to sink downwards,
as in \Fig{42}, where each successive ordinate
is $\frac{3}{4}$~of the height of the preceding one.

The equation is still
\[
y=bp^x;
\]
\Figure[2.5in]{168a}{42}
but since $p$ is less than one, $\log_\epsilon p$ will be a negative
quantity, and may be written~$-a$; so that $p=\epsilon^{-a}$,
and now our equation for the curve takes the form
\[
y=b\epsilon^{-ax}.
\]

The importance of this expression is that, in the
case where the independent variable is \emph{time}, the
equation represents the course of a great many
physical processes in which something is \emph{gradually
dying away}. Thus, the cooling of a hot body is
represented (in Newton's celebrated ``law of cooling'')
by the equation
\[
\theta_t=\theta_0 \epsilon^{-at};
\]
\DPPageSep{169.png}{157}%
where $\theta_0$ is the original excess of temperature of a
hot body over that of its surroundings, $\theta_t$~the excess
of temperature at the end of time~$t$, and $a$~is a constant---namely,
the constant of decrement, depending
on the amount of surface exposed by the body, and
on its coefficients of conductivity and emissivity,
etc.

A similar formula,
\[
Q_t=Q_0 \epsilon^{-at},
\]
is used to express the charge of an electrified body,
originally having a charge~$Q_0$, which is leaking away
with a constant of decrement~$a$; which constant
depends in this case on the capacity of the body and
on the resistance of the leakage-path.

Oscillations given to a flexible spring die out after
a time; and the dying-out of the amplitude of the
motion may be expressed in a similar way.

In fact $\epsilon^{-at}$ serves as a \emph{die-away factor} for all
those phenomena in which the rate of decrease
is proportional to the magnitude of that which is
decreasing; or where, in our usual symbols, $\dfrac{dy}{dt}$~is
proportional at every moment to the value that~$y$ has
at that moment. For we have only to inspect the
curve, \Fig{42} above, to see that, at every part of it,
the slope~$\dfrac{dy}{dx}$ is proportional to the height~$y$; the
curve becoming flatter as $y$~grows smaller. In symbols,
thus
\begin{DPgather*}
y=b\epsilon^{-ax}\\
\DPPageSep{170.png}{158}%
\lintertext{or}
\log_\epsilon y
  = \log_\epsilon b - ax \log_\epsilon \epsilon
  = \log_\epsilon b - ax,\\
\lintertext{\rlap{and, differentiating,}}
\frac{1}{y}\, \frac{dy}{dx} = -a;\\
\lintertext{hence} \frac{dy}{dx} = b\epsilon^{-ax} × (-a) = -ay;
\end{DPgather*}
or, in words, the slope of the curve is downward, and
proportional to~$y$ and to the constant~$a$.

We should have got the same result if we had
taken the equation in the form
\begin{DPalign*}
y &= bp^x; \\
\lintertext{for then}
\frac{dy}{dx}
  &= bp^x × \log_\epsilon p. \\
\lintertext{\indent But}
\log_\epsilon p &= -a; \\
\lintertext{giving us}
\frac{dy}{dx} &= y × (-a) = -ay,
\end{DPalign*}
as before.

\Paragraph{The Time-constant.} In the expression for the ``die-away
factor''~$\epsilon^{-at}$, the quantity~$a$ is the reciprocal of
another quantity known as ``\emph{the time-constant},'' which
we may denote by the symbol~$T$. Then the die-away
factor will be written~$\epsilon^{-\efrac{t}{T}}$; and it will be seen, by
making $t = T$ that the meaning of~$T$ $\left(\text{or of}~\dfrac{1}{a}\right)$ is that
this is the length of time which it takes for the original
quantity (called $\theta_0$ or~$Q_0$ in the preceding instances)
to die away $\dfrac{1}{\epsilon}$th~part---that is to $0.3678$---of its
original value.
\DPPageSep{171.png}{159}%

The values of $\epsilon^x$ and~$\epsilon^{-x}$ are continually required
in different branches of physics, and as they are given
in very few sets of mathematical tables, some of the
values are tabulated \DPchg{here}{on \Pageref{littletable}} for convenience.

%[** TN: Allow table to float; re-worded preceding sentence]
\begin{table}[p]
\Pagelabel{littletable}%
\[
\setlength{\arraycolsep}{1.5em}% [** Hard-coded length]
\begin{array}{| .{2,2} | .{5,4} | .{1,6} | .{1,6} |}
\hline
\multicolumn{1}{|c|}{\Strut x} &
  \multicolumn{1}{c|}{\epsilon^x} &
  \multicolumn{1}{c|}{\epsilon^{-x}} &
  \multicolumn{1}{c|}{1-\epsilon^{-x}} \\
\hline
\Strut
0.00   &     1.0000 & 1.0000   & 0.0000   \\
0.10   &     1.1052 & 0.9048   & 0.0952   \\
0.20   &     1.2214 & 0.8187   & 0.1813   \\
0.50   &     1.6487 & 0.6065   & 0.3935   \\
0.75   &     2.1170 & 0.4724   & 0.5276   \\
0.90   &     2.4596 & 0.4066   & 0.5934   \\
1.00   &     2.7183 & 0.3679   & 0.6321   \\
1.10   &     3.0042 & 0.3329   & 0.6671   \\
1.20   &     3.3201 & 0.3012   & 0.6988   \\
1.25   &     3.4903 & 0.2865   & 0.7135   \\
1.50   &     4.4817 & 0.2231   & 0.7769   \\
1.75   &     \DPtypo{5.754}{5.755}  & 0.1738   & 0.8262   \\
2.00   &     7.389  & 0.1353   & 0.8647   \\
2.50   &    \DPtypo{12.183}{12.182}  & 0.0821   & 0.9179   \\
3.00   &    \DPtypo{20.085}{20.086}  & 0.0498   & 0.9502   \\
3.50   &    33.115  & 0.0302   & 0.9698   \\
4.00   &    54.598  & 0.0183   & 0.9817   \\
4.50   &    90.017  & 0.0111   & 0.9889   \\
5.00   &   148.41   & 0.0067   & 0.9933   \\
5.50   &   244.69   & 0.0041   & 0.9959   \\
6.00   &   403.43   & 0.00248  & 0.99752  \\
7.50   &  1808.04   & \DPtypo{0.00053}{0.00055}  & 0.99947  \\
10.00  & 22026.5    & 0.000045 & 0.999955 \\
\hline
\end{array}
\]
\end{table}

As an example of the use of this table, suppose
there is a hot body cooling, and that at the beginning
\DPPageSep{172.png}{160}%
of the experiment (\IE~when $t = 0$) it is $72°$~hotter than
the surrounding objects, and if the time-constant of its
cooling is $20$~minutes (that is, if it takes $20$~minutes
for its excess of temperature to fall to $\dfrac{1}{\epsilon}$~part of~$72°$),
then we can calculate to what it will have fallen in
any given time~$t$. For instance, let $t$ be $60$~minutes.
Then $\dfrac{t}{T} = 60 ÷ 20 = 3$, and we shall have to find the
value of~$\epsilon^{-3}$, and then multiply the original~$72°$ by
this. The table shows that $\epsilon^{-3}$ is~$0.0498$. So that
at the end of $60$~minutes the excess of temperature
will have fallen to $72° × 0.0498 = 3.586°$.

\tb

\clearpage%[** TN: Page break dependent on text block size]
\Examples{Further Examples.}
(1) The strength of an electric current in a conductor
at a time $t$~secs.\ after the application of the
electromotive force producing it is given by the expression
$C = \dfrac{E}{R}\left\{1 - \epsilon^{-\efrac{Rt}{L}}\right\}$.

The time constant is~$\dfrac{L}{R}$.

If $E = 10$, $R =1$, $L = 0.01$; then when $t$~is very large
the term~$\epsilon^{-\efrac{Rt}{L}}$ becomes~$1$, and $C = \dfrac{E}{R} = 10$; also
\[
\frac{L}{R} = T = 0.01.
\]

Its value at any time may be written:
\[
C = 10 - 10\epsilon^{-\efrac{t}{0.01}},
\]
\DPPageSep{173.png}{161}%
the time-constant being~$0.01$. This means that it
takes $0.01$~sec.\ for the variable term to fall by
$\dfrac{1}{\epsilon} = 0.3678$ of its initial value $10\epsilon^{-\efrac{0}{0.01}} = 10$.

To find the value of the current when $t = 0.001~\text{sec.}$,
say, $\dfrac{t}{T} = 0.1$, $\epsilon^{-0.1} = 0.9048$ (from table).

It follows that, after $0.001$~sec., the variable term
is $0.9048 × 10 = 9.048$, and the actual current is
$10 - 9.048 = 0.952$.

Similarly, at the end of $0.1$~sec.,
\[
\frac{t}{T} = 10;\quad \epsilon^{-10} = 0.000045;
\]
the variable term is $10 × 0.000045 = 0.00045$, the current
being~$9.9995$.

(2) The intensity~$I$ of a beam of light which has
passed through a thickness $l$~cm.\ of some transparent
medium is $I = I_0\epsilon^{-Kl}$, where $I_0$~is the initial intensity
of the beam and $K$ is a ``constant of absorption.''

This constant is usually found by experiments. If
it be found, for instance, that a beam of light has
its intensity diminished by~18\% in passing through
$10$~cms.\ of a certain transparent medium, this means
that $82 = 100 × \epsilon^{-K×10}$ or $\epsilon^{-10K} = 0.82$, and from the
table one sees that $10K = 0.20$ very nearly; hence
$K = 0.02$.

To find the thickness that will reduce the intensity
to half its value, one must find the value of~$l$ which
satisfies the equality $50 = 100 × \epsilon^{-0.02l}$, or $0.5 = \epsilon^{-0.02l}$.
\DPPageSep{174.png}{162}%
It is found by putting this equation in its logarithmic
form, namely,
\[
\log 0.5 = -0.02 × l × \log \epsilon,
\]
which gives
\[
%[** TN: Original numerator supposed to mean "-1 + .6990"?]
l = \frac{\DPchg{\overset{-}{1}.6990}{-0.3010}}{-0.02 × 0.4343}
 = \DPtypo{34.5}{34.7}~\text{centimetres nearly}.
\]

(3) The quantity~$Q$ of a radio-active substance
which has not yet undergone transformation is known
to be related to the initial quantity~$Q_0$ of the substance
by the relation $Q = Q_0 \epsilon^{-\lambda t}$, where $\lambda$ is a constant
and $t$~the time in seconds elapsed since the transformation
began.

For ``Radium~$A$,'' if time is expressed in seconds,
experiment shows that $\lambda = 3.85 × 10^{-3}$. Find the time
required for transforming half the substance. (This
time is called the ``mean life'' of the substance.)

We have $0.5 = \epsilon^{-0.00385t}$.\Pagelabel{erratum0a}%
\begin{DPalign*}
\log 0.5 &= -0.00385t × \log \epsilon; \\
\lintertext{and}
t &= 3\text{ minutes very nearly}.
\end{DPalign*}


\Exercises{XIII} (See \Pageref[page]{AnsEx:XIII} for Answers.)
\begin{Problems}
\Item{(1)} Draw the curve $y = b \epsilon^{-\efrac{t}{T}}$; where $b = 12$, $T = 8$,
and $t$ is given various values from $0$~to~$20$.

\Item{(2)} If a hot body cools so that in $24$~minutes its
excess of temperature has fallen to half the initial
amount, deduce the time-constant, and find how long
it will be in cooling down to $1$~per~cent.\ of the original
excess.
\DPPageSep{175.png}{163}%

\Item{(3)} Plot the curve $y = 100(1-\epsilon^{-2t})$.

\Item[XIII:4]{(4)} The following equations give very similar curves:
\begin{align*}
\text{(i)}\   y &= \frac{ax}{x + b}; \\
\text{(ii)}\  y &= a(1 - \epsilon^{-\efrac{x}{b}}); \\
\text{(iii)}\ y &= \frac{a}{90°} \arctan \left(\frac{x}{b}\right).
\end{align*}

Draw all three curves, taking $a= 100$ millimetres;
$b = 30$ millimetres.

\Item{(5)} Find the differential coefficient of~$y$ with respect
to~$x$, if
\[
(\textit{a})~y = x^x;\quad
(\textit{b})~y = (\epsilon^x)^x;\quad
(\textit{c})~y = \epsilon^{x^x}.
\]

\Item{(6)} For ``Thorium~$A$,'' the value of~$\lambda$ is~$5$; find the
``mean life,'' that is, the time taken by the transformation
of a quantity~$Q$ of ``Thorium~$A$'' equal to
half the initial quantity~$Q_0$ in the expression
\[
Q = Q_0 \epsilon^{-\lambda t};
\]
$t$~being in seconds.

\Item{(7)} A condenser of capacity $K = 4 × 10^{-6}$, charged
to a potential $V_0 = 20$, is discharging through a resistance
of $10,000$~ohms. Find the potential~$V$ after (\textit{a})~$0.1$
second; (\textit{b})~$0.01$ second; assuming that the fall of
potential follows the rule $V = V_0 \epsilon^{-\efrac{t}{KR}}$.

\Item{(8)} The charge~$Q$ of an electrified insulated metal
sphere is reduced from $20$ to $16$~units in $10$~minutes.
Find the coefficient~$\mu$ of leakage, if $Q = Q_0 × \epsilon^{-\mu t}$; $Q_0$
being the initial charge and $t$~being in seconds. Hence
find the time taken by half the charge to leak away.
\DPPageSep{176.png}{164}%

\Item{(9)} The damping on a telephone line can be ascertained
from the relation $i = i_0 \epsilon^{-\beta l}$, where $i$~is the
strength, after $t$~seconds, of a telephonic current of
initial strength~$i_0$; $l$~is the length of the line in kilometres,
and $\beta$~is a constant. For the Franco-English
submarine cable laid in 1910, $\beta = 0.0114$. Find the
damping at the end of the cable ($40$~kilometres), and
the length along which $i$~is still $8$\%~of the original
current (limiting value of very good audition).

\Item{(10)} The pressure~$p$ of the atmosphere at an altitude
$h$~kilometres is given by $p=p_0 \epsilon^{-kh}$; $p_0$~being the
pressure at sea-level ($760$~millimetres).

The pressures at $10$,~$20$ and~$50$ kilometres being
$199.2$, $42.2$, $0.32$ respectively, find~$k$ in each case.
Using the mean value of~$k$, find the percentage error
in each case.

\Item{(11)} Find the minimum or maximum of $y = x^x$.

\Item{(12)} Find the minimum or maximum of $y = x^{\efrac{1}{x}}$.

\Item{(13)} Find the minimum or maximum of $y = xa^{\efrac{1}{x}}$.
\end{Problems}
\DPPageSep{177.png}{165}%

