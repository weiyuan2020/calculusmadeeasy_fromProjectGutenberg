
\Chapter{XVI}{Partial Differentiation}

\First{We} sometimes come across quantities that are functions
of more than one independent variable. Thus,
we may find a case where $y$ depends on two other
variable quantities, one of which we will call $u$ and
the other~$v$. In symbols\Pagelabel{partialdiff}
\begin{DPalign*}
y &= f(u, v). \\
\intertext{\indent Take the simplest concrete case.}
\lintertext{\indent Let }  y &= u×v. \\
\intertext{What are we to do? If we were to treat $v$ as a
constant, and differentiate with respect to~$u$, we
should get}
dy_v &= v\, du; \\
\intertext{or if we treat $u$ as a constant, and differentiate with
respect to~$v$, we should have:}
dy_u &= u\, dv.
\end{DPalign*}

The little letters here put as subscripts are to show
which quantity has been taken as constant in the
operation.

Another way of indicating that the differentiation
has been performed only \emph{partially}, that is, has been
performed only with respect to \emph{one} of the independent
\DPPageSep{188.png}{176}%
variables, is to write the differential coefficients with
Greek deltas, like~$\partial$, instead of little~$d$. In this way
\begin{align*}
\DStrut
\frac{\partial y}{\partial u} &= v, \\
\frac{\partial y}{\partial v} &= u.
\end{align*}

If we put in these values for $v$~and~$u$ respectively,
we shall have
\[
\left.
\begin{aligned}
\DStrut
dy_v &= \frac{\partial y}{\partial u}\, du, \\
dy_u &= \frac{\partial y}{\partial v}\, dv,
\end{aligned} \right\}
\quad\text{which are \emph{partial differentials}.}
\]

But, if you think of it, you will observe that the
total variation of~$y$ depends on \emph{both} these things at
the same time. That is to say, if both are varying,
the real $dy$ ought to be written
\[
dy = \frac{\partial y}{\partial u}\, du + \dfrac{\partial y}{\partial v}\, dv;
\]
and this is called a \emph{total differential}. In some books
it is written $dy = \left(\dfrac{dy}{du}\right)\, du + \left(\dfrac{dy}{dv}\right)\, dv$.


\Example{(1).} Find the partial differential coefficients
of the expression $w = 2ax^2 + 3bxy + 4cy^3$.
The answers are:
\[
\left.
\begin{aligned}
\frac{\partial w}{\partial x} &= 4ax + 3by. \\
\frac{\partial w}{\partial y} &= 3bx + 12cy^2.
\end{aligned} \right\}
\]
\DPPageSep{189.png}{177}%

The first is obtained by supposing $y$ constant, the
second is obtained by supposing $x$ constant; then
\[
dw = (4ax+3by)\, dx + (3bx+12cy^2)\, dy.
\]


\Example{(2).} Let $z = x^y$. Then, treating first $y$
and then $x$ as constant, we get in the usual way
\[
\left.
\begin{aligned}
\dfrac{\partial z}{\partial x} &= yx^{y-1}, \\
\dfrac{\partial z}{\partial y} &= x^y × \log_\epsilon x,
\end{aligned}\right\}
\]
so that $dz = yx^{y-1}\, dx + x^y \log_\epsilon x \, dy$.


\Example{(3).} A cone having height $h$ and radius
of base~$r$\DPchg{,}{} has volume $V=\frac{1}{3} \pi r^2 h$. If its height remains
constant, while $r$ changes, the ratio of change of
volume, with respect to radius, is different from ratio
of change of volume with respect to height which
would occur if the height were varied and the radius
kept constant, for
\[
\left.
\begin{aligned}
\frac{\partial V}{\partial r} &= \dfrac{2\pi}{3} rh, \\
\frac{\partial V}{\partial h} &= \dfrac{\pi}{3} r^2.
\end{aligned}\right\}
\]

The variation when both the radius and the height
change is given by $dV = \dfrac{2\pi}{3} rh\, dV + \dfrac{\pi}{3} r^2\, dh$.


\Example{(4).} \Pagelabel{Example4} In the following example $F$~and~$f$
denote two arbitrary functions of any form whatsoever.
For example, they may be sine-functions, or
exponentials, or mere algebraic functions of the two
\DPPageSep{190.png}{178}%
independent variables, $t$~and~$x$. This being understood,
let us take the expression
\begin{DPalign*}
                    y &= F(x+at) + f(x-at), \\
\lintertext{or, }   y &= F(w) + f(v); \\
\lintertext{where } w &= x+at,\quad \text{and}\quad v = x-at. \\
\lintertext{\indent Then } \frac{\partial y}{\partial x}
                      &= \frac{\partial F(w)}{\partial w} · \frac{\partial w}{\partial x}
                       + \frac{\partial f(v)}{\partial v} · \frac{\partial v}{\partial x} \\
                      &= F'(w) · 1 + f'(v) · 1
\intertext{(where the figure~$1$ is simply the coefficient of~$x$ in
$w$~and~$v$);}
\lintertext{and }     \frac{\partial^2 y}{\partial x^2}
                      &= F''(w) + f''(v). && \\
\lintertext{\indent Also } \frac{\partial y}{\partial t}
                      &= \frac{\partial F(w)}{\partial w} · \frac{\partial w}{\partial t}
                      + \frac{\partial f(v)}{\partial v} · \frac{\partial v}{\partial t} \\
                      &= F'(w) · a - f'(v) a; \\
\lintertext{and }     \frac{\partial^2 y}{\partial t^2}
                      &= F''(w)a^2 + f''(v)a^2; \\
\lintertext{whence }  \frac{\partial^2 y}{\partial t^2}
                      &= a^2\, \frac{\partial^2 y}{\partial x^2}.
\end{DPalign*}

This differential equation is of immense importance
in mathematical physics.

\Section{Maxima and Minima of Functions of two
Independent Variables.}


\Example{(5).} Let us take up again Exercise~IX.,
\Pageref{Ex9No4}, No.~4.

Let $x$~and~$y$ be the length of two of the portions of
the string. The third is $30-(x+y)$, and the area of the
\DPPageSep{191.png}{179}%
triangle is $A = \sqrt{s(s-x)(s-y)(s-30+x+y)}$, where
$s$ is the half perimeter, $15$, so that $A = \sqrt{15P}$, where
\begin{align*}%[** TN: This display centered in the original]
P &= (15-x)(15-y)(x+y-15) \\
  &= xy^2 + x^2y - 15x^2 - 15y^2 - 45xy + 450x + 450y - 3375.
\end{align*}

Clearly $A$ is maximum when $P$ is maximum.
\[
dP = \dfrac{\partial P}{\partial x}\, dx + \dfrac{\partial P}{\partial y}\, dy.
\]
For a maximum (clearly it will not be a minimum in
this case), one must have simultaneously
\BindMath{%
\[
\dfrac{\partial P}{\partial x} = 0 \quad\text{and}\quad
\dfrac{\partial P}{\partial y} = 0;
\]
\begin{DPalign*}
%[** Can't top-align the equations without enlarging the right brace]
\lintertext{\raisebox{0.5\baselineskip}{that is,}}
\left.
\begin{aligned}
2xy - 30x + y^2 - 45y + 450 &= 0, \\
2xy - 30y + x^2 - 45x + 450 &= 0.
\end{aligned}
\right\}
\end{DPalign*}}%

An immediate solution is $x=y$.

If we now introduce this condition in the value
of~$P$, we find
\[
P = (15-x)^2 (2x-15) = 2x^3 - 75x^2 + 900x - 3375.
\]
For maximum or minimum, $\dfrac{dP}{dx} = 6x^2 - 150x + 900 = 0$,
which gives $x=15$ or $x=10$.

Clearly $x=15$ gives minimum area; $x=10$ gives
the maximum, for $\dfrac{d^2 P}{dx^2} = 12x - 150$, which is $+30$ for
$x=15$ and $-30$ for $x=10$.


\Example{(6).} Find the dimensions of an ordinary
railway coal truck with rectangular ends, so that,
for a given volume~$V$ the area of sides and floor
together is as small as possible.
\DPPageSep{192.png}{180}%

The truck is a rectangular box open at the top.
Let $x$ be the length and $y$ be the width; then the
depth is~$\dfrac{V}{xy}$. The surface area is $S=xy + \dfrac{2V}{x} + \dfrac{2V}{y}$.
\[
dS = \frac{\partial S}{\partial x}\, dx
   + \frac{\partial S}{\partial y}\, dy
   = \left(y - \frac{2V}{x^2}\right) dx
   + \left(x - \frac{2V}{y^2}\right) dy.
\]
For minimum (clearly it won't be a maximum here),
\[
y - \frac{2V}{x^2} = 0,\quad
x - \frac{2V}{y^2} = 0.
\]

Here also, an immediate solution is $x = y$, so that
$S = x^2 + \dfrac{4V}{x}$,\quad $\dfrac{dS}{dx}= 2x - \dfrac{4V}{x^2} =0$ for minimum, and
\[
x = \sqrt[3]{2V}.
\]


\Exercises{XV} (See \Pageref[page]{AnsEx:XV} for Answers.)
\begin{Problems}
\Item{(1)} Differentiate the expression $\dfrac{x^3}{3} - 2x^3y - 2y^2x + \dfrac{y}{3}$
with respect to $x$~alone, and with respect to $y$~alone.

\Item{(2)} Find the partial differential coefficients with
respect to $x$,~$y$ and~$z$, of the expression
\[
x^2yz + xy^2z + xyz^2 + x^2y^2z^2.
\]

\Item{(3)} Let $r^2 = (x-a)^2 + (y-b)^2 + (z-c)^2$.

Find the value of $\dfrac{\partial r}{\partial x} +
                   \dfrac{\partial r}{\partial y} +
                   \dfrac{\partial r}{\partial z}$. Also find the value
of $\dfrac{\partial^2r}{\partial x^2} +
    \dfrac{\partial^2r}{\partial y^2} +
    \dfrac{\partial^2r}{\partial z^2}$.

\Item{(4)} Find the total differential of~$y=u^v$.
\DPPageSep{193.png}{181}%

\Item{(5)} Find the total differential of $y=u^3 \sin v$; of
$y = (\sin x)^u$; and of $y = \dfrac{\log_\epsilon u}{v}$.

\Item{(6)} Verify that the sum of three quantities $x$,~$y$,~$z$,
whose product is a constant~$k$, is maximum when
these three quantities are equal.

\Item{(7)} Find the maximum or minimum of the function
\[
u = x + 2xy + y.
\]

\Item{(8)} The post-office regulations state that no parcel
is to be of such a size that its length plus its girth
exceeds $6$~feet. What is the greatest volume that
can be sent by post (\textit{a})~in the case of a package of
rectangular cross section; (\textit{b})~in the case of a package
of circular cross section.

\Item{(9)} Divide $\pi$ into $3$~parts such that the continued
product of their sines may be a maximum or minimum.

\Item{(10)} Find the maximum or minimum of $u = \dfrac{\epsilon^{x+y}}{xy}$.

\Item{(11)} Find maximum and minimum of
\[
u = y + 2x - 2 \log_\epsilon y - \log_\epsilon x.
\]

\Item{(12)} A telpherage bucket of given capacity has
the shape of a horizontal isosceles triangular prism
with the apex underneath, and the opposite face open.
Find its dimensions in order that the least amount
of iron sheet may be used in its construction.
\end{Problems}
\DPPageSep{194.png}{182}%

