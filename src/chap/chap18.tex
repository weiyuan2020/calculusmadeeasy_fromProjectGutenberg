

\Chapter[HOW TO INTEGRATE]
{XVIII}{Integrating as the Reverse of Differentiating}

\First{Differentiating} is the process by which when $y$~is
given us (as a function of~$x$), we can find~$\dfrac{dy}{dx}$.

\Pagelabel{revdif}%
Like every other mathematical operation, the
process of differentiation may be reversed; thus, if
differentiating $y = x^4$ gives us $\dfrac{dy}{dx} = 4x^3$; if one begins
with $\dfrac{dy}{dx} = 4x^3$ one would say that reversing the process
would yield $y = x^4$.  But here comes in a curious
point. We should get $\dfrac{dy}{dx} = 4x^3$ if we had begun with
\emph{any} of the following:~$x^4$, or~$x^4 + a$, or~$x^4 + c$, or~$x^4$
with \emph{any} added constant. So it is clear that in
working backwards from $\dfrac{dy}{dx}$ to~$y$, one must make
provision for the possibility of there being an added
constant, the value of which will be undetermined
\DPPageSep{204.png}{192}%
until ascertained in some other way. So, if differentiating
$x^n$ yields~$nx^{n-1}$, going backwards from
$\dfrac{dy}{dx} = nx^{n-1}$ will give us $y = x^n + C$; where $C$~stands
for the yet undetermined possible constant.

Clearly, in dealing with powers of~$x$, the rule for
working backwards will be: Increase the power by~$1$,
then divide by that increased power, and add the
undetermined constant.

So, in the case where
\[
\frac{dy}{dx} = x^n,
\]
working backwards, we get
\[
y = \frac{1}{n + 1} x^{n+1} + C.
\]

If differentiating the equation $y = ax^n$ gives us
\[
\frac{dy}{dx} = anx^{n-1},
\]
it is a matter of common sense that beginning with
\[
\frac{dy}{dx} = anx^{n-1},
\]
and reversing the process, will give us
\[
y = ax^n.
\]
So, when we are dealing with a multiplying constant,
we must simply put the constant as a multiplier of
the result of the integration.
\DPPageSep{205.png}{193}%

Thus, if $\dfrac{dy}{dx} = 4x^2$, the reverse process gives us
$y = \frac{4}{3}x^3$.

But this is incomplete. For we must remember
that if we had started with
\[
y = ax^n + C,
\]
where $C$ is any constant quantity whatever, we should
equally have found
\[
\frac{dy}{dx} = anx^{n-1}.
\]

So, therefore, when we reverse the process we must
always remember to add on this undetermined constant,
even if we do not yet know what its value
will be.

This process, the reverse of differentiating, is called
\emph{integrating}; for it consists in finding the value of
the whole quantity~$y$ when you are given only an
expression for~$dy$ or for~$\dfrac{dy}{dx}$. Hitherto we have as
much as possible kept $dy$~and~$dx$ together as a differential
coefficient: henceforth we shall more often
have to separate them.

If we begin with a simple case,
\[
\frac{dy}{dx} = x^2.
\]

We may write this, if we like, as
\[
dy = x^2\, dx.
\]

Now this is a  ``differential equation''  which informs
us that an element of~$y$ is equal to the corresponding
element of~$x$ multiplied by~$x^2$. Now, what we want
\DPPageSep{206.png}{194}%
is the integral; therefore, write down with the proper
symbol the instructions to integrate both sides, thus:
\[
\int dy = \int x^2\, dx.
\]

[Note as to reading integrals: the above would be
read thus:
\begin{quote}
``\emph{Integral dee-wy \emph{equals} integral eks-squared dee-eks}.'']
\end{quote}

We haven't yet integrated: we have only written
down instructions to integrate---if we can. Let us
try. Plenty of other fools can do it---why not we
also? The left-hand side is simplicity itself. The
sum of all the bits of~$y$ is the same thing as $y$~itself.
So we may at once put:
\[
y = \int x^2\, dx.
\]

But when we come to the right-hand side of the
equation we must remember that what we have got
to sum up together is not all the~$dx$'s, but all such
terms as~$x^2\, dx$; and this will \emph{not} be the same as
$x^2 \ds\int dx$, because $x^2$~is not a constant. For some of the
$dx$'s will be multiplied by big values of~$x^2$, and some
will be multiplied by small values of~$x^2$, according to
what $x$~happens to be. So we must bethink ourselves
as to what we know about this process of integration
being the reverse of differentiation. Now, our rule
for this reversed process---see \Pageref{revdif} \textit{ante}---when
dealing with~$x^n$ is ``increase the power by one, and\Pagelabel{diffrule}
divide by the same number as this increased power.''
\DPPageSep{207.png}{195}%
That is to say, $x^2\, dx$ will be changed\footnote
  {You may ask, what has become of the little~$dx$ at the end?
  Well, remember that it was really part of the differential coefficient,
  and when changed over to the right-hand side, as in the~$x^2\, dx$,
  serves as a reminder that $x$~is the independent variable with respect
  to which the operation is to be effected; and, as the result of the
  product being totalled up, the power of~$x$ has increased by \emph{one}.
  You will soon become familiar with all this.}
to~$\frac{1}{3} x^3$. Put
this into the equation; but don't forget to add the
``constant of integration''~$C$ at the end. So we get:
\[
y = \tfrac{1}{3} x^3 + C.
\]

You have actually performed the integration. How
easy!

Let us try another simple case.

\begin{DPalign*}
\lintertext{\indent Let}
\dfrac{dy}{dx} &= ax^{12},
\end{DPalign*}
where $a$ is any constant multiplier. Well, we found
when differentiating (see \Pageref{differ}) that any constant
factor in the value of~$y$ reappeared unchanged in the
value of~$\dfrac{dy}{dx}$. In the reversed process of integrating,
it will therefore also reappear in the value of~$y$. So
we may go to work as before, thus
\begin{align*}
dy &= ax^{12} · dx,\\
\int dy &= \int ax^{12} · dx,\\
\int dy &= a \int x^{12}\, dx,\\
y &= a × \tfrac{1}{13} x^{13} + C.
\end{align*}

So that is done. How easy!
\DPPageSep{208.png}{196}%

We begin to realize now that integrating is a
process of \emph{finding our way back}, as compared with
differentiating. If ever, during differentiating, we
have found any particular expression---in this example
$ax^{12}$---we can find our way back to the~$y$ from which
it was derived. The contrast between the two
processes may be illustrated by the following remark
due to a well-known teacher. If a stranger were set
down in Trafalgar Square, and told to find his way to
Euston Station, he might find the task hopeless. But
if he had previously been personally conducted from
Euston Station to Trafalgar Square, it would be
comparatively easy to him to find his way back to
Euston Station.

\Section{Integration of the Sum or Difference of two
Functions.}

\begin{DPalign*}
\lintertext{\indent Let}
\frac{dy}{dx} &= x^2 + x^3, \\
\lintertext{then}
dy &= x^2\, dx + x^3\, dx.
\end{DPalign*}

There is no reason why we should not integrate
each term separately: for, as may be seen on \Pageref{sumdiffer},
we found that when we differentiated the sum of two
separate functions, the differential coefficient was
simply the sum of the two separate differentiations.
So, when we work backwards, integrating, the integration
will be simply the sum of the two separate
integrations.
\DPPageSep{209.png}{197}%

Our instructions will then be:
\begin{align*}
\int dy
  &= \int (x^2 + x^3)\, dx \\
  &= \int x^2\, dx + \int x^3\, dx   \\
y &= \tfrac{1}{3} x^3 + \tfrac{1}{4} x^4 + C.
\end{align*}

If either of the terms had been a negative quantity,
the corresponding term in the integral would have
also been negative. So that differences are as readily
dealt with as sums.

\Section{How to deal with Constant Terms.}

Suppose there is in the expression to be integrated
a constant term---such as this:
\[
\frac{dy}{dx} = x^n + b.
\]

This is laughably easy. For you have only to
remember that when you differentiated the expression
$y = ax$, the result was $\dfrac{dy}{dx} = a$. Hence, when you work
the other way and integrate, the constant reappears
multiplied by~$x$. So we get
\begin{align*}
dy &= x^n\, dx + b · dx,  \\
\int dy &= \int x^n\, dx + \int b\, dx, \\
y &= \frac{1}{n+1} x^{n+1} + bx + C.
\end{align*}

Here are a lot of examples on which to try your
newly acquired powers.

\tb
\DPPageSep{210.png}{198}%


\Examples.
(1) Given $\dfrac{dy}{dx} = 24x^{11}$. Find~$y$.\qquad \textit{Ans}.\ $y = 2x^{12} + C$.

(2) Find $\ds\int (a + b)(x + 1)\, dx$.\qquad It is $(a + b) \ds\int (x + 1)\, dx$ \\
or\quad $(a + b) \left[\ds\int x\, dx + \ds\int dx\right]$\quad or\quad $(a + b) \left(\dfrac{x^2}{2} + x\right) + C$.

(3) Given $\dfrac{du}{dt} = gt^{\efrac{1}{2}}$. Find~$u$.\qquad \textit{Ans}.\ $u = \frac{2}{3} gt^{\efrac{3}{2}} + C$.

(4) $\dfrac{dy}{dx} = x^3 - x^2 + x$. Find~$y$.
\BindMath{\begin{align*}
dy &= (x^3 - x^2 + x)\, dx\quad\text{or} \\
dy &= x^3\, dx - x^2\, dx + x\, dx;\quad
y = \int x^3\, dx - \int x^2\, dx + \int x\, dx;
\end{align*}
\begin{DPalign*}
\lintertext{and}
y &= \tfrac{1}{4} x^4 - \tfrac{1}{3} x^3 + \tfrac{1}{2} x^2 + C.
\end{DPalign*}}%

(5) Integrate $9.75x^{2.25}\, dx$.\qquad \textit{Ans}.\ $y = 3x^{3.25} + C$.

\tb

All these are easy enough. Let us try another case.%
\SetOddHead{Easiest Integrations}%

\begin{DPalign*}
\lintertext{Let}
\dfrac{dy}{dx} &= ax^{-1}.
\end{DPalign*}

Proceeding as before, we will write
\[
dy = a x^{-1} · dx,\quad \int dy = a \int x^{-1}\, dx.
\]

Well, but what is the integral of $x^{-1}\, dx$?

If you look back amongst the results of differentiating
$x^2$ and~$x^3$ and $x^n$,~etc., you will find we never
got~$x^{-1}$ from any one of them as the value of~$\dfrac{dy}{dx}$.
We got~$3x^2$ from~$x^3$; we got~$2x$ from~$x^2$; we got~$1$
from~$x^1$ (that is, from $x$~itself); but we did not get
$x^{-1}$ from~$x^0$, for two very good reasons. \emph{First}, $x^0$~is
simply~$= 1$, and is a constant, and could not have
\DPPageSep{211.png}{199}%
a differential coefficient. \emph{Secondly}, even if it could
be differentiated, its differential coefficient (got by
slavishly following the usual rule) would be $0 × x^{-1}$,
and that multiplication by zero gives it zero value!
Therefore when we now come to try to integrate
$x^{-1}\, dx$, we see that it does not come in anywhere
in the powers of~$x$ that are given by the rule:
\[
\int x^n\, dx = \dfrac{1}{n+1} x^{n+1}.
\]
It is an exceptional case.

Well; but try again. Look through all the various
differentials obtained from various functions of~$x$, and
try to find amongst them~$x^{-1}$. A sufficient search
will show that we actually did get $\dfrac{dy}{dx} = x^{-1}$ as the
result of differentiating\Pagelabel{intex1} the function $y = \log_\epsilon x$ (see
\Pageref{differlog}). %[ ** Page]

Then, of course, since we know that differentiating
$\log_\epsilon x$ gives us~$x^{-1}$, we know that, by reversing the
process, integrating $dy = x^{-1}\, dx$ will give us $y = \log_\epsilon x$.
But we must not forget the constant factor~$a$ that
was given, nor must we omit to add the undetermined
constant of integration. This then gives us as the
solution to the present problem,
\[
y = a \log_\epsilon x + C.
\]

\NB---Here note this very remarkable fact, that we
could not have integrated in the above case if we had
not happened to know the corresponding differentiation.
If no one had found out that differentiating
$\log_\epsilon x$ gave~$x^{-1}$, we should have been utterly stuck by
\DPPageSep{212.png}{200}%
the problem how to integrate~$x^{-1}\, dx$. Indeed it should
be frankly admitted that this is one of the curious
features of the integral calculus:---that you can't
integrate anything before the reverse process of differentiating
something else has yielded that expression
which you want to integrate. No one, even to-day,
is able to find the general integral of the expression,
\[
\frac{dy}{dx} = a^{-x^2},
\]
because $a^{-x^2}$ has never yet been found to result from
differentiating anything else.


\Subsection{Another simple case.}
Find $\ds\int (x + 1)(x + 2)\, dx$.

On looking at the function to be integrated, you
remark that it is the product of two different functions
of~$x$. You could, you think, integrate $(x + 1)\, dx$ by
itself, or $(x + 2)\, dx$ by itself. Of course you could.
But what to do with a product? None of the differentiations
you have learned have yielded you for the
differential coefficient a product like this. Failing
such, the simplest thing is to multiply up the two
functions, and then integrate. This gives us
\[
\int (x^2 + 3x + 2)\, dx.
\]
And this is the same as
\[
\int x^2\, dx + \int 3x\, dx + \int 2\, dx.
\]
And performing the integrations, we get
\[
\tfrac{1}{3} x^3 + \tfrac{3}{2} x^2 + 2x + C.
\]
\DPPageSep{213.png}{201}%


\Section[Some Other Integrals]{Some other Integrals.}

Now that we know that integration is the reverse
of differentiation, we may at once look up the differential
coefficients we already know, and see from
what functions they were derived. This gives us the
following integrals ready made:\Pagelabel{intex2}\Pagelabel{differ3}\Pagelabel{cosax}%
\begin{alignat*}{4}
&x^{-1} &&\text{(\Pageref{differlog});}\qquad &&
  \int x^{-1}\, dx      &&= \log_\epsilon x + C. \\
%
%\label{intex2}
&\frac{1}{x+a} && \text{(\Pageref{differ2});} &&
  \int \frac{1}{x+a}\, dx &&= \log_\epsilon (x+a) + C. \\
%
&\epsilon^x && \text{(\Pageref{unchanged});} &&
  \int \epsilon^x\, dx    &&= \epsilon ^x + C. \\
%
&\epsilon^{-x} &&&&
  \int \epsilon^{-x}\, dx &&= -\epsilon^{-x} + C \\
%
\intertext{(for if $y = - \dfrac{1}{\epsilon^x}$,\quad $\dfrac{dy}{dx} = -\dfrac{\epsilon^x × 0 - 1 × \epsilon^x}{\epsilon^{2x}} = \epsilon^{-x}$).}
%
&\sin x && \text{(\Pageref{differcos});} &&
  \int \sin x\, dx        &&= -\cos x + C. \\
%
&\cos x && \text{(\Pageref{differsin});} &&
  \int \cos x\, dx        &&= \sin x + C. \\
%
\intertext{\indent Also we may deduce the following:}
%
&\log_\epsilon x; &&&&
  \int\log_\epsilon x\, dx &&= x(\log_\epsilon x - 1) + C \\
%
\intertext{(for if $y = x \log_\epsilon x - x$,\quad $\dfrac{dy}{dx} = \dfrac{x}{x} + \log_\epsilon x - 1 = \log_\epsilon x$).}
\DPPageSep{214.png}{202}%
%
&\log_{10} x;   &&&&
  \int\log_{10} x\, dx &&= 0.4343x (\log_\epsilon x - 1) + C. \\
%
&a^x && \text{(\Pageref{diffexp});}  &&
  \int a^x\, dx        &&= \dfrac{a^x}{\log_\epsilon a} + C. \\
%
% \label{cosax}
&\cos ax; &&&& \int\cos ax\, dx     &&= \frac{1}{a} \sin ax + C \\
\intertext{(for if $y = \sin ax$, $\dfrac{dy}{dx} = a \cos ax$; hence to get $\cos ax$
one must differentiate $y = \dfrac{1}{a} \sin ax$).}
%
&\sin ax; &&&& \int\sin ax\, dx     &&= -\frac{1}{a} \cos ax + C. \\
\end{alignat*}

Try also $\cos^2\theta$; a little dodge will simplify matters:
\begin{DPgather*}
\cos 2\theta = \cos^2\theta - \sin^2\theta
             = \DPtypo{2\cos 2\theta - 1}{2\cos^2 \theta - 1}; \\
\lintertext{hence}
\cos^2\theta = \tfrac{1}{2}(\DPtypo{\cos^2 \theta + 1}{\cos 2\theta + 1}),
\end{DPgather*}
\begin{DPalign*}
\lintertext{and}
\int\cos^2 \theta\, d\theta
  &= \tfrac{1}{2} \int (\cos 2\theta + 1)\, d\theta \\
  &= \tfrac{1}{2} \int \cos 2 \theta\, d\theta + \tfrac{1}{2} \int d\theta. \\
  &= \frac{\sin 2\theta}{4} + \frac{\theta}{2} + C.\text{ (See also \Pageref{moreexamples}).} %[ ** Page]
\end{DPalign*}


See also the Table of Standard Forms on \Pagerange{stdforms1}{stdforms2}.
You should make such a table for yourself, putting
in it only the general functions which you have
successfully differentiated and integrated. See to it
that it grows steadily!
\DPPageSep{215.png}{203}%


\Section[Double Integrals]{On Double and Triple Integrals.}

In many cases it is necessary to integrate some
expression for two or more variables contained in it;
and in that case the sign of integration appears more
than once. Thus,
\[
\iint f(x,y,)\, dx\, dy
\]
means that some function of the variables $x$~and~$y$
has to be integrated for each. It does not matter in
which order they are done. Thus, take the function
$x^2 + y^2$. Integrating it with respect to~$x$ gives us:
\[
\int (x^2+y^2)\, dx = \tfrac{1}{3} x^3 + xy^2.
\]

Now, integrate this with respect to~$y$:
\[
\int (\tfrac{1}{3} x^3 + xy^2)\, dy = \tfrac{1}{3} x^3y + \tfrac{1}{3} xy^3,
\]
to which of course a constant is to be added. If we
had reversed the order of the operations, the result
would have been the same.

In dealing with areas of surfaces and of solids, we
have often to integrate both for length and breadth,
and thus have integrals of the form
\[
\iint u · dx\, dy,
\]
where $u$ is some property that depends, at each point,
on~$x$ and on~$y$. This would then be called a \emph{surface-integral}.
It indicates that the value of all such
\DPPageSep{216.png}{204}%
elements as $u · dx · dy$ (that is to say, of the value of~$u$
over a little rectangle $dx$~long and $dy$~broad) has to be
summed up over the whole length and whole breadth.

Similarly in the case of solids, where we deal with
three dimensions. Consider any element of volume,
the small cube whose dimensions are $dx$~$dy$~$dz$. If
the figure of the solid be expressed by the function
$f(x, y, z)$, then the whole solid will have the \emph{volume-integral},
\[
\text{volume} = \iiint f(x,y,z) · dx · dy · dz.
\]
Naturally, such integrations have to be taken between
appropriate limits\footnote
  {See \Pageref{limits} for integration between limits.}
in each dimension; and the
integration cannot be performed unless one knows in
what way the boundaries of the surface depend on
$x$,~$y$, and~$z$. If the limits for~$x$ are from $x_1$ to~$x_2$,
those for~$y$ from $y_1$ to~$y_2$, and those for~$z$ from $z_1$
to~$z_2$, then clearly we have
\[
\text{volume} = \int_{z1}^{z2} \int_{y1}^{y2} \int_{x1}^{x2} f(x,y,z) · dx · dy · dz.
\]

There are of course plenty of complicated and
difficult cases; but, in general, it is quite easy to
see the significance of the symbols where they are
intended to indicate that a certain integration has to
be performed over a given surface, or throughout a
given solid space.
\DPPageSep{217.png}{205}%


\Exercises[SIMPLE INTEGRATIONS]{XVII}
(See \Pageref{AnsEx:XVII} for the Answers.)
\begin{Problems}
\Item{(1)} Find  $\ds\int y\, dx$ when  $y^2 = 4 ax$.

\ResetCols{2}
\Item{(2)} Find  $\ds\int \frac{3}{x^4}\, dx$.
\Item{(3)} Find  $\ds\int \frac{1}{a} x^3\, dx$.

\ResetCols{2}
\Item{(4)} Find  $\ds\int (x^2 + a)\, dx$.
\Item{(5)} Integrate $5x^{-\efrac{7}{2}}$.

\ResetCols{1}
\Item{(6)} Find  $\ds\int (4x^3 + 3x^2 + 2x + 1)\, dx$.

\Item{(7)} If $\dfrac{dy}{dx} = \dfrac{ax}{2} + \dfrac{bx^2}{3} + \dfrac{cx^3}{4}$; find~$y$.

\ResetCols{2}
\Item{(8)} Find $\ds\int \left(\frac{x^2 + a}{x + a}\right) dx$.
\Item{(9)}  Find $\ds\int (x + 3)^3\, dx$.

\ResetCols{1}
\Item{(10)} Find $\ds\int (x + 2)(x - a)\, dx$.

\Item{(11)} Find $\ds\int (\sqrt x + \sqrt[3] x) 3a^2\, dx$.

\Item{(12)} Find $\ds\int (\sin \theta - \tfrac{1}{2})\, \frac{d\theta}{3}$.

\ResetCols{2}
\Item{(13)} Find  $\ds\int \cos^2 a \theta\, d\theta$.
\Item{(14)} Find  $\ds\int \sin^2 \theta\, d\theta$.

\ResetCols{2}
\Item{(15)} Find  $\ds\int \sin^2 a \theta\, d\theta$.
\Item{(16)} Find  $\ds\int \epsilon^{3x}\, dx$.

\ResetCols{2}
\Item{(17)} Find  $\ds\int \dfrac{dx}{1 + x}$.
\Item{(18)} Find  $\ds\int \dfrac{dx}{1 - x}$.
\end{Problems}
\DPPageSep{218.png}{206}%
