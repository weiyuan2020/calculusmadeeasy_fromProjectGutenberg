

\Chapter{VII}{Successive Differentiation}

\First{Let} us try the effect of repeating several times over
the operation of differentiating a function (see \Pageref{function}).
Begin with a concrete case.

Let $y = x^5$.
\begin{alignat*}{3}
&\text{First differentiation, }  &&5x^4.              &&         \\
&\text{Second differentiation, } &&5 × 4x^3           &&= 20x^3. \\
&\text{Third differentiation, }  &&5 × 4 × 3x^2       &&= 60x^2. \\
&\text{Fourth differentiation, } &&5 × 4 × 3 × 2x     &&= 120x.  \\
&\text{Fifth differentiation, }  &&5 × 4 × 3 × 2 × 1  &&= 120.   \\
&\text{Sixth differentiation, }  &&                   &&= 0.
\end{alignat*}

There is a certain notation, with which we are
already acquainted (see \Pageref{notation}), used by some writers,
that is very convenient. This is to employ the
general symbol~$f(x)$ for any function of~$x$. Here
the symbol~$f(~)$ is read as ``function of,'' without
saying what particular function is meant. So the
statement $y=f(x)$ merely tells us that $y$ is a function
of~$x$, it may be $x^2$ or $ax^n$, or $\cos x$ or any other complicated
function of~$x$.

The corresponding symbol for the differential coefficient
is~$f'(x)$, which is simpler to write than $\dfrac{dy}{dx}$.
This is called the ``derived function'' of~$x$.
\DPPageSep{062.png}{50}%

Suppose we differentiate over again, we shall get
the ``second derived function'' or second differential
coefficient, which is denoted by~$f''(x)$; and so on.

Now let us generalize.

Let $y = f(x) = x^n$.
\begin{DPalign*}[m]
\lintertext{\indent First differentiation,}     f'(x) &= nx^{n-1}. \\
\lintertext{\indent Second differentiation,}   f''(x) &= n(n-1)x^{n-2}. \\
\lintertext{\indent Third differentiation,}   f'''(x) &= n(n-1)(n-2)x^{n-3}. \\
\lintertext{\indent Fourth differentiation,} f''''(x) &= n(n-1)(n-2)(n-3)x^{n-4}. \\
    &\llap{\text{etc.,}} \text{ etc.}
\end{DPalign*}

But this is not the only way of indicating successive
differentiations. For,
\begin{DPalign*}
\lintertext{if the original function be }             y &= f(x);  \\
\lintertext{once differentiating gives }  \frac{dy}{dx} &= f'(x); \\
\lintertext{twice differentiating gives } \frac{d\left(\dfrac{dy}{dx}\right)}{dx} &= f''(x);
\end{DPalign*}
and this is more conveniently written as~$\dfrac{d^2y}{(dx)^2}$, or
more usually~$\dfrac{d^2y}{dx^2}$. Similarly, we may write as the
result of thrice differentiating, $\dfrac{d^3y}{dx^3} = f'''(x)$.
\tb
\DPPageSep{063.png}{51}%


\Examples.
Now let us try $y = f(x) = 7x^4 + 3.5x^3 - \frac{1}{2}x^2 + x - 2$.
\begin{align*}
\frac{dy}{dx}     &= f'(x) = 28x^3 + 10.5x^2 - x + 1, \\
\frac{d^2y}{dx^2} &= f''(x) = 84x^2 + 21x - 1,        \\
\frac{d^3y}{dx^3} &= f'''(x) = 168x + 21,             \\
\frac{d^4y}{dx^4} &= f''''(x) = 168,                  \\
\frac{d^5y}{dx^5} &= f'''''(x) = 0.
\end{align*}
In a similar manner if $y = \phi(x) = 3x(x^2 - 4)$,
\begin{align*}
\phi'(x)    &= \frac{dy}{dx} = 3\bigl[x × 2x + (x^2 - 4) × 1\bigr] = 3(3x^2 - 4), \\
\phi''(x)   &= \frac{d^2y}{dx^2} = 3 × 6x = 18x, \\
\phi'''(x)  &= \frac{d^3y}{dx^3} = 18, \\
\phi''''(x) &= \frac{d^4y}{dx^4} = 0.
\end{align*}


%[** TN: Heading indented in the original]
\Exercises{IV} (See \Pageref[page]{AnsEx:III} for Answers.)

Find $\dfrac{dy}{dx}$ and $\dfrac{d^2y}{dx^2}$ for the following expressions:
\begin{Problems}[2]
\Item{(1)} $y = 17x + 12x^2$.
\Item{(2)} $y = \dfrac{x^2 + a}{x + a}$.
\ResetCols{1}

\Item{(3)} $y = 1 + \dfrac{x}{1} + \dfrac{x^2}{1×2} + \dfrac{x^3}{1×2×3} + \dfrac{x^4}{1×2×3×4}$.

\Item{(4)} Find the 2nd and~3rd derived functions in
the Exercises~III. (\Pageref{examples2}), No.~1 to No.~7, and in the
Examples given (\Pageref{examples3}), No.~1 to No.~7.
\end{Problems}
\DPPageSep{064.png}{52}%

