
\Chapter{XII}{Curvature of Curves}

\First{Returning} to the process of successive differentiation,
it may be asked: Why does anybody want to
differentiate twice over? We know that when the
variable quantities are space and time, by differentiating
twice over we get the acceleration of a
moving body, and that in the geometrical interpretation,
as applied to curves, $\dfrac{dy}{dx}$~means the \emph{slope} of the
curve. But what can $\dfrac{d^2 y}{dx^2}$~mean in this case? Clearly
it means the rate (per unit of length~$x$) at which the
slope is changing---in brief, it is \emph{a measure of the
curvature of the slope}.

\Figures{124a}{124b}{31}{32}

Suppose a slope constant, as in \Fig{31}.

Here, $\dfrac{dy}{dx}$ is of constant value.
\DPPageSep{125.png}{113}%

Suppose, however, a case in which, like \Fig{32},
the slope itself is getting greater upwards, then
$\dfrac{d\left(\dfrac{dy}{dx}\right)}{dx}$, that is,~$\dfrac{d^2y}{dx^2}$, will be \emph{positive}.

If the slope is becoming less as you go to the right
(as in \Fig{14}, \Pageref{fig:14}), or as in \Fig{33}, then, even %[ ** Page]
though the curve may be going upward, since the
change is such as to diminish its slope, its $\dfrac{d^2y}{dx^2}$ will
be \emph{negative}.

\Figure{125a}{33}

It is now time to initiate you into another secret---how
to tell whether the result that you get by
``equating to zero'' is a maximum or a minimum.
The trick is this: After you have differentiated
(so as to get the expression which you equate to
zero), you then differentiate a second time, and look
whether the result of the second differentiation is
\emph{positive} or \emph{negative}. If $\dfrac{d^2y}{dx^2}$ comes out \emph{positive}, then
you know that the value of~$y$ which you got was
a \emph{minimum}; but if $\dfrac{d^2y}{dx^2}$ comes out \emph{negative}, then
\DPPageSep{126.png}{114}%
the value of~$y$ which you got must be a \emph{maximum}.
That's the rule.

The reason of it ought to be quite evident. Think
of any curve that has a minimum point in it (like
\Fig{15}, \Pageref{fig:15}), or like \Fig{34}, where the point of %[ ** Page]
minimum~$y$ is marked~$M$, and the curve is \emph{concave}
upwards. To the left of~$M$ the slope is downward,
that is, negative, and is getting less negative. To the
right of~$M$ the slope has become upward, and is
\Figures{126a}{126b}{34}{35}
getting more and more upward. Clearly the change
of slope as the curve passes through~$M$ is such that
$\dfrac{d^2y}{dx^2}$~is \emph{positive}, for its operation, as $x$~increases toward
the right, is to convert a downward slope into an
upward one.

Similarly, consider any curve that has a maximum
point in it (like \Fig{16}, \Pageref{fig:16}), or like \Fig{35}, where %[ ** Page]
the curve is \emph{convex}, and the maximum point is
marked~$M$. In this case, as the curve passes through~$M$
from left to right, its upward slope is converted
\DPPageSep{127.png}{115}%
into a downward or negative slope, so that in this
case the ``slope of the slope'' $\dfrac{d^2y}{dx^2}$ is \emph{negative}.

Go back now to the examples of the last chapter
and verify in this way the conclusions arrived at as to
whether in any particular case there is a maximum
or a minimum. You will find below a few worked
out examples.

\tb

(1) Find the maximum or minimum of
\begin{align*}
\text{(\textit{a})}\quad y &= 4x^2-9x-6; \qquad \text{(\textit{b})}\quad y = 6 + 9x-4x^2; \\
\intertext{and ascertain if it be a maximum or a minimum in
each case.}
\text{(\textit{a})}\quad \dfrac{dy}{dx}
  &= 8x-9=0;\quad x=1\tfrac{1}{8},\quad \text{and } y = -11.065.\\
%
\dfrac{d^2y}{dx^2}
  &= 8;\quad \text{it is~$+$; hence it is a minimum.} \\
%
\text{(\textit{b})}\quad \DPtypo{\dfrac{dx}{dy}}{\dfrac{dy}{dx}}
  &= 9-8x=0;\quad x = 1\tfrac{1}{8};\quad \text{and } y = +11.065.\\
%
\dfrac{d^2y}{dx^2}
  &= -8;\quad \text{it is~$-$; hence it is a maximum.}
\end{align*}

(2) Find the maxima and minima of the function
$y = x^3-3x+16$.
\begin{align*}
\dfrac{dy}{dx}
  &= 3x^2 - 3 = 0;\quad x^2 = 1;\quad \text{and } x = ±1.\\
%
\dfrac{d^2y}{dx^2}
  &= 6x;\quad \text{for $x = 1$; it is~$+$};
\end{align*}
hence $x=1$ corresponds to a minimum $y=14$. For
$x=-1$ it is~$-$; hence $x=-1$ corresponds to a maximum
$y=+18$.
\DPPageSep{128.png}{116}%

(3) Find the maxima and minima of $y=\dfrac{x-1}{x^2+2}$.
\[
\frac{dy}{dx} = \frac{(x^2+2) × 1 - (x-1) × 2x}{(x^2+2)^2}
  = \frac{2x - x^2 + 2}{(x^2 + 2)^2} = 0;
\]
or $x^2 - 2x - 2 = 0$, whose solutions are $x =+2.73$ and
$x=-0.73$.
\begin{align*}
\dfrac{d^2y}{dx^2}
  &= - \frac{(x^2 + 2)^2 × (2x-2) - (x^2 - 2x - 2)(4x^3 + 8x)}{(x^2 + 2)^4} \\
  &= - \frac{2x^5 - 6x^4 - 8x^3 - 8x^2 - 24x + 8}{(x^2 + 2)^4}.
\end{align*}

The denominator is always positive, so it is sufficient
to ascertain the sign of the numerator.

If we put $x = 2.73$, the numerator is negative; the
maximum, $y = 0.183$.

If we put $x=-0.73$, the numerator is positive; the
minimum, $y=-0.683$.

(4) The expense~$C$ of handling the products of a
certain factory varies with the weekly output~$P$
according to the relation $C = aP + \dfrac{b}{c+P} + d$, where
$a$,~$b$, $c$,~$d$ are positive constants. For what output
will the expense be least?
\[
\dfrac{dC}{dP} = a - \frac{b}{(c+P)^2} = 0\quad \text{for maximum or minimum;}
\]
hence $a = \dfrac{b}{(c+P)^2}$ and $P = ±\sqrt{\dfrac{b}{a}} - c$.

As the output cannot be negative, $P=+\sqrt{\dfrac{b}{a}} - c$.
\DPPageSep{129.png}{117}%
\begin{DPalign*}
\lintertext{\indent Now}
\frac{d^2C}{dP^2} &= + \frac{b(2c + 2P)}{(c + P)^4},
\end{DPalign*}
which is positive for all the values of~$P$; hence
$P = +\sqrt{\dfrac{b}{a}} - c$ corresponds to a minimum.

(5) The total cost per hour~$C$ of lighting a building
with $N$~lamps of a certain kind is
\[
C = N\left(\frac{C_l}{t} + \frac{EPC_e}{1000}\right),
\]
where $E$ is the commercial efficiency (watts per candle),
\begin{align*}
&\text{$P$~is the candle power of each lamp,} \\
&\text{$t$~is the average life of each lamp in hours,} \\
&\text{$C_l =$ cost of renewal in pence per hour of use,} \\
&\text{$C_e =$ cost of energy per $1000$~watts per~hour.}
\end{align*}

Moreover, the relation connecting the average life
of a lamp with the commercial efficiency at which it
is run is approximately $t = mE^n$, where $m$~and~$n$ are
constants depending on the kind of lamp.

Find the commercial efficiency for which the total
cost of lighting will be least.
%
\begin{DPalign*}
\lintertext{\indent We have}
C &= N\left(\frac{C_l}{m} E^{-n} + \frac{PC_e}{1000} E\right), \\
\dfrac{dC}{dE}
  &= \frac{PC_e}{1000} - \frac{nC_l}{m} E^{-(n+1)} = 0
\end{DPalign*}
for maximum or minimum.
\[
E^{n+1} = \frac{1000 × nC_l}{mPC_e}\quad \text{and}\quad
E = \sqrt[n+1]{\frac{1000 × nC_l}{mPC_e}}.
\]
\DPPageSep{130.png}{118}%

This is clearly for minimum, since
\[
\frac{d^2C}{dE^2} = (n + 1) \frac{nC_l}{m} E^{-(n+2)},
\]
which is positive for a positive value of~$E$.

For a particular type of $16$~candle-power lamps,
$C_l= 17$~pence, $C_e=5$~pence; and it was found that
$m=10$ and~$n=3.6$.
\[
E = \sqrt[4.6]{\frac{1000 × 3.6 × 17}{10 × 16 × 5}}
  = 2.6\text{ watts per candle-power}.
\]


\Exercises{X} (You are advised to plot the graph
of any numerical example.) (See \Pageref{AnsEx:X} for the
Answers.)
\begin{Problems}
\Item{(1)} Find the maxima and minima of
\[
y = x^3 + x^2 - 10x + 8.
\]

\Item{(2)} Given $y = \dfrac{b}{a}x - cx^2$, find expressions for~$\dfrac{dy}{dx}$, and
for~$\dfrac{d^2y}{dx^2}$, also find the value of~$x$ which makes $y$ a
maximum or a minimum, and show whether it is
maximum or minimum.

\Item{(3)} Find how many maxima and how many minima
there are in the curve, the equation to which is
\[
y = 1 - \frac{x^2}{2} + \frac{x^4}{24};
\]
and how many in that of which the equation is
\[
y = 1 - \frac{x^2}{2} + \frac{x^4}{24} - \frac{x^6}{720}.
\]
\DPPageSep{131.png}{119}%

\Item{(4)} Find the maxima and minima of
\[
y=2x+1+\frac{5}{x^2}.
\]

\Item{(5)} Find the maxima and minima of
\[
y=\frac{3}{x^2+x+1}.
\]

\Item{(6)} Find the maxima and minima of
\[
y=\frac{5x}{2+x^2}.
\]

\Item{(7)} Find the maxima and minima of
\[
y=\frac{3x}{x^2-3} + \frac{x}{2} + 5.
\]

\Item{(8)} Divide a number~$N$ into two parts in such a
way that three times the square of one part plus
twice the square of the other part shall be a
minimum.

\Item{(9)} The efficiency~$u$ of an electric generator at
different values of output~$x$ is expressed by the
general equation:
\[
u=\frac{x}{a+bx+cx^2};
\]
where $a$ is a constant depending chiefly on the energy
losses in the iron and $c$~a constant depending chiefly
on the resistance of the copper parts. Find an expression
for that value of the output at which the
efficiency will be a maximum.
\DPPageSep{132.png}{120}%

\Item{(10)} Suppose it to be known that consumption of
coal by a certain steamer may be represented by the
formula $y = 0.3 + 0.001v^3$; where $y$~is the number of
tons of coal burned per hour and $v$~is the speed
expressed in nautical miles per hour. The cost of
wages, interest on capital, and depreciation of that
ship are together equal, per hour, to the cost of
$1$~ton of coal. What speed will make the total cost
of a voyage of $1000$ nautical miles a minimum?
And, if coal costs $10$~shillings per ton, what will that
minimum cost of the voyage amount to?

\Item{(11)} Find the maxima and minima of\Pagelabel{Ex:X11}%
\[
y = ±\frac{x}{6}\sqrt{x(10-x)}.
\]

\Item{(12)} Find the maxima and minima of
\[
y= 4x^3 - x^2 - 2x + 1.
\]
\end{Problems}
\DPPageSep{133.png}{121}%

