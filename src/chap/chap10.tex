
\Chapter[MEANING OF DIFFERENTIATION]
{X}{Geometrical Meaning of Differentiation}

\First{It} is useful to consider what geometrical meaning can
be given to the differential coefficient.

In the first place, any function of~$x$, such, for
example, as~$x^2$, or~$\sqrt{x}$, or~$ax+b$, can be plotted as
a curve; and nowadays every schoolboy is familiar
with the process of curve-plotting.

\Figure{088a}{7}

Let $PQR$, in \Fig{7}, be a portion of a curve plotted
with respect to the axes of coordinates $OX$~and~$OY$.
Consider any point~$Q$ on this curve, where the
abscissa of the point is~$x$ and its ordinate is~$y$.
Now observe how $y$~changes when $x$~is varied. If $x$
\DPPageSep{089.png}{77}%
is made to increase by a small increment~$dx$, to the
right, it will be observed that $y$ also (in \emph{this} particular
curve) increases by a small increment~$dy$ (because this
particular curve happens to be an \emph{ascending} curve).
Then the ratio of $dy$~to~$dx$ is a measure of the degree
to which the curve is sloping up between the two
points $Q$~and~$T$. As a matter of fact, it can be seen
on the figure that the curve between $Q$ and~$T$ has
many different slopes, so that we cannot very well
speak of\Pagelabel{slope} the slope of the curve between $Q$ and~$T$. If,
however, $Q$ and~$T$ are so near each other that the
small portion~$QT$ of the curve is practically straight,
then it is true to say that the ratio~$\dfrac{dy}{dx}$ is the slope of
the curve along~$QT$. The straight line~$QT$ produced
on either side touches the curve along the portion~$QT$
only, and if this portion is indefinitely small, the
straight line will touch the curve at practically
one point only, and be therefore a \emph{tangent} to the
curve.

This tangent to the curve has evidently the same
slope as~$QT$, so that $\dfrac{dy}{dx}$ is the slope of the tangent to
the curve at the point~$Q$ for which the value of~$\dfrac{dy}{dx}$ is
found.

We have seen that the short expression ``the slope
of a curve'' has no precise meaning, because a curve
has so many slopes---in fact, every small portion of a
curve has a different slope. ``The slope of a curve \emph{at
a point}'' is, however, a perfectly defined thing; it is
\DPPageSep{090.png}{78}%
the slope of a very small portion of the curve situated
just at that point; and we have seen that this is the
same as ``the slope of the tangent to the curve at that
point.''

Observe that $dx$ is a short step to the right, and
$dy$ the corresponding short step upwards. These
steps must be considered as short as possible---in fact
indefinitely short,---though in diagrams we have to
represent them by bits that are not infinitesimally
small, otherwise they could not be seen.

\textit{We shall hereafter make considerable use of this
circumstance that $\dfrac{dy}{dx}$ represents the slope of the curve
at any point.}

\Figure{090a}{8}

If a curve is sloping up at~$45°$ at a particular point,
as in \Fig{8}, $dy$ and~$dx$ will be equal, and the value
of $\dfrac{dy}{dx} = 1$.
\DPPageSep{091.png}{79}%

If the curve slopes up steeper than~$45°$ (\Fig{9}),
$\dfrac{dy}{dx}$~will be greater than~$1$.
\Figures{091a}{091b}{9}{10}

If the curve slopes up very gently, as in \Fig{10},
$\dfrac{dy}{dx}$~will be a fraction smaller than~$1$.

For a horizontal line, or a horizontal place in a
curve, $dy=0$, and therefore~$\dfrac{dy}{dx}=0$.
\Figure{091c}{11}

If a curve slopes \emph{downward}, as in \Fig{11}, $dy$~will
be a step down, and must therefore be reckoned of
\DPPageSep{092.png}{80}%
negative value; hence $\dfrac{dy}{dx}$~will have negative sign
also.

If the ``curve'' happens to be a straight line, like
that in \Fig{12}, the value of~$\dfrac{dy}{dx}$ will be the same at
all points along it. In other words its \emph{slope} is constant.

\Figure{092a}{12}

If a curve is one that turns more upwards as it
goes along to the right, the values of~$\dfrac{dy}{dx}$ will become
greater and greater with the increasing steepness, as
in \Fig{13}.
\Figure{092b}{13}
\DPPageSep{093.png}{81}%

If a curve\Pagelabel{curve} is one that gets flatter and flatter as it
goes along, the values of~$\dfrac{dy}{dx}$ will become smaller and
smaller as the flatter part is reached, as in \Fig{14}.

\Figures{093a}{093b}{14}{15}

If a curve first descends, and then goes up again,
as in \Fig{15}, presenting a concavity upwards, then
clearly $\dfrac{dy}{dx}$ will first be negative, with diminishing
values as the curve flattens, then will be zero at the
point where the bottom of the trough of the curve is
reached; and from this point onward $\dfrac{dy}{dx}$ will have
positive values that go on increasing. In such a case
$y$~is said to pass by a \emph{minimum}. The minimum
value of~$y$ is not necessarily the smallest value of~$y$,
it is that value of~$y$ corresponding to the bottom of
the trough; for instance, in \Fig{28} (\Pageref{fig:28}), the %[ ** Page number]
value of~$y$ corresponding to the bottom of the trough
is~$1$, while $y$~takes elsewhere values which are smaller
than this. The characteristic of a minimum is that
$y$~must increase \emph{on either side} of it.
\DPPageSep{094.png}{82}%

\NB---For the particular value of~$x$ that makes
$y$~\emph{a minimum}, the value of $\dfrac{dy}{dx} = 0$.

If a curve first ascends and then descends, the
values of~$\dfrac{dy}{dx}$ will be positive at first; then zero, as
the summit is reached; then negative, as the curve
slopes downwards, as in \Fig{16}. In this case $y$~is
said to pass by a \emph{maximum}, but the maximum
value of~$y$ is not necessarily the greatest value of~$y$.
In \Fig{28}, the maximum of~$y$ is~$2\frac{1}{3}$, but this is by no
means the greatest value $y$ can have at some other
point of the curve.

\Figures{094a}{094b}{16}{17}

\NB---For the particular value of~$x$ that makes
$y$~\emph{a maximum}, the value of $\dfrac{dy}{dx}= 0$.

If a curve has the peculiar form of \Fig{17}, the
values of~$\dfrac{dy}{dx}$ will always be positive; but there will
be one particular place where the slope is least steep,
where the value of~$\dfrac{dy}{dx}$ will be a minimum; that is,
less than it is at any other part of the curve.
\DPPageSep{095.png}{83}%

If a curve has the form of \Fig{18}, the value of~$\dfrac{dy}{dx}$
will be negative in the upper part, and positive in the
lower part; while at the nose of the curve where it
becomes actually perpendicular, the value of $\dfrac{dy}{dx}$ will
be infinitely great.

\Figure{095a}{18}

Now that we understand that $\dfrac{dy}{dx}$~measures the
steepness of a curve at any point, let us turn to some
of the equations which we have already learned how
to differentiate.

(1) As the simplest case take this:
\[
y=x+b.
\]

It is plotted out in \Fig{19}, using equal scales
for $x$~and~$y$. If we put $x = 0$, then the corresponding
ordinate will be $y = b$; that is to say, the ``curve''
\Figures{096a}{096b}{19}{20}%[** TN: Moved up]
crosses the $y$-axis at the height~$b$. From here it
\DPPageSep{096.png}{84}%
ascends at~$45°$; for whatever values we give to~$x$ to
the right, we have an equal~$y$ to ascend. The line
has a gradient of $1$~in~$1$.

Now differentiate $y = x+b$, by the rules we have
already learned (pp.\ \pageref{diffrule1}~and~\pageref{diffrule2} \textit{ante}), and we get $\dfrac{dy}{dx} = 1$.

The slope of the line is such that for every little
step~$dx$ to the right, we go an equal little step~$dy$
upward. And this slope is constant---always the
same slope.

(2) Take another case:\Pagelabel{Case2}
\[
y = ax+b.
\]
We know that this curve, like the preceding one, will
start from a height~$b$ on the $y$-axis. But before we
draw the curve, let us find its slope by differentiating;
which gives $\dfrac{dy}{dx} = a$. The slope will be constant, at
an angle, the tangent of which is here called~$a$. Let
us assign to~$a$ some numerical value---say~$\frac{1}{3}$. Then we
must give it such a slope that it ascends $1$~in~$3$; or
\DPPageSep{097.png}{85}%
$dx$ will be $3$~times as great as~$dy$; as magnified in
\Fig{21}. So, draw the line in \Fig{20} at this slope.

\Figure[1.75in]{097a}{21}
(3) Now for a slightly harder case.
\begin{DPalign*}
\lintertext{\indent Let}
y= ax^2 + b.
\end{DPalign*}

Again the curve will start on the $y$-axis at a height~$b$
above the origin.

Now differentiate. [If you have forgotten, turn
back to \Pageref{diffrule2}; or, rather, \emph{don't} turn back, but think %[ ** Page number]
out the differentiation.]
\[
\frac{dy}{dx} = 2ax.
\]

\Figure[3.25in]{097b}{22}

This shows that the steepness will not be constant:
it increases as $x$~increases. At the starting point~$P$,
\DPPageSep{098.png}{86}%
where $x = 0$, the curve (\Fig{22}) has no steepness---that
is, it is level. On the left of the origin, where
$x$ has negative values, $\dfrac{dy}{dx}$~will also have negative
values, or will descend from left to right, as in the
Figure.

Let us illustrate this by working out a particular
instance. Taking the equation
\[
y = \tfrac{1}{4}x^2 + 3,
\]
and differentiating it, we get
\[
\dfrac{dy}{dx} = \tfrac{1}{2}x.
\]
Now assign a few successive values, say from $0$~to~$5$,
to~$x$; and calculate the corresponding values of~$y$
by the first equation; and of~$\dfrac{dy}{dx}$ from the second
equation. Tabulating results, we have:
\[
\begin{array}{|c||*{6}{c|}}
\hline
\Strut
\Td[c]{x} & \Td[c]{0} & \Td{1\Z} & \Td[c]{2} & \Td{3\Z} & \Td[c]{4} & \Td{5\Z} \\
\hline
\Strut
\Td[c]{y} & \Td[c]{3} & \Td{3\frac{1}{4}} & \Td[c]{4} & \Td{5\frac{1}{4}} & \Td[c]{7} & \Td{9\frac{1}{4}} \\
\hline
\DStrut
\Td[c]{\dfrac{dy}{dx}} &
\Td[c]{0} & \Td{\frac{1}{2}} & \Td[c]{1} & \Td{1\frac{1}{2}} & \Td[c]{2} & \Td{2\frac{1}{2}} \\
\hline
\end{array}
\]
Then plot them out in two curves, \Figs{23}{and}{24}\DPtypo{}{,}
in \Fig{23} plotting the values of~$y$ against those of~$x$
and in \Fig{24} those of $\dfrac{dy}{dx}$ against those of~$x$. For
\DPPageSep{099.png}{87}%
any assigned value of~$x$, the \emph{height} of the ordinate
in the second curve is proportional to the \emph{slope} of the
first curve.

\Figures[2.5in]{099a}{099b}{23}{24}

If a curve comes to a sudden cusp, as in \Fig{25},
the slope at that point suddenly changes from a slope
\Figure{099c}{25}
upward to a slope downward. In that case $\dfrac{dy}{dx}$ will
clearly undergo an abrupt change from a positive to
a negative value.
\DPPageSep{100.png}{88}%

The following examples show further applications
of the principles just explained.

(4) Find the slope of the tangent to the curve
\[
y = \frac{1}{2x} + 3,
\]
at the point where $x = -1$. Find the angle which this
tangent makes with the curve $y = 2x^2 + 2$.

The slope of the tangent is the slope of the curve at
the point where they touch one another (see \Pageref{slope});
that is, it is the $\dfrac{dy}{dx}$ of the curve for that point. Here
$\dfrac{dy}{dx} = -\dfrac{1}{2x^2}$ and for $x = -1$, $\dfrac{dy}{dx} = -\dfrac{1}{2}$, which is the
slope of the tangent and of the curve at that point.
The tangent, being a straight line, has for equation
$y = ax + b$, and its slope is $\dfrac{dy}{dx} = a$, hence $a = -\dfrac{1}{2}$. Also
if $x= -1$, $y = \dfrac{1}{2(-1)} + 3 = 2\frac{1}{2}$; and as the tangent
passes by this point, the coordinates of the point must
satisfy the equation of the tangent, namely
\[
  y = -\dfrac{1}{2} x + b,
\]
so that $2\frac{1}{2} = -\dfrac{1}{2} × (-1) + b$ and $b = 2$; the equation of
the tangent is therefore $y = -\dfrac{1}{2} x + 2$.

Now, when two curves meet, the intersection being
a point common to both curves, its coordinates must
satisfy the equation of each one of the two curves;
\DPPageSep{101.png}{89}%
that is, it must be a solution of the system of simultaneous
equations formed by coupling together the
equations of the curves. Here the curves meet one
another at points given by the solution of
\begin{DPgather*}
\left\{
\begin{aligned}
  y &= 2x^2 + 2, \\
  y &= -\tfrac{1}{2} x + 2 \quad\text{or}\quad  2x^2 + 2 = -\tfrac{1}{2} x + 2;
\end{aligned}
\right. \displaybreak[1] \\
\lintertext{that is,}
x(2x + \tfrac{1}{2}) = 0.
\end{DPgather*}

This equation has for its solutions $x = 0$ and $x = -\tfrac{1}{4}$.
The slope of the curve $y = 2x^2 + 2$ at any point is
\[
\dfrac{dy}{dx} = 4x.
\]

For the point where $x = 0$, this slope is zero; the curve
is horizontal. For the point where
\[
x = -\dfrac{1}{4},\quad \dfrac{dy}{dx} = -1;
\]
hence the curve at that point slopes downwards to
the right at such an angle~$\theta$ with the horizontal that
$\tan \theta = 1$; that is, at~$45°$ to the horizontal.

The slope of the straight line is $-\tfrac{1}{2}$; that is, it slopes
downwards to the right and makes with the horizontal
an angle~$\phi$ such that $\tan \phi = \tfrac{1}{2}$; that is, an angle of
$26°~34'$. It follows that at the first point the curve
cuts the straight line at an angle of $26°~34'$, while at
the second it cuts it at an angle of $45° - 26°~34' = 18°~26'$.

(5) A straight line is to be drawn, through a point
whose coordinates are $x = 2$, $y = -1$, as tangent to the
curve $y = x^2 - 5x + 6$. Find the coordinates of the
point of contact.
\DPPageSep{102.png}{90}%

The slope of the tangent must be the same as the
$\dfrac{dy}{dx}$ of the curve; that is, $2x - 5$.

The equation of the straight line is $y = ax + b$, and
as it is satisfied for the values $x = 2$, $y = -1$, then
$-1 = a×2 + b$; also, its $\dfrac{dy}{dx} = a = 2x - 5$.

The $x$ and the $y$ of the point of contact must also
satisfy both the equation of the tangent and the
equation of the curve.

We have then
%[** TN: Omitting dot leaders.]
\begin{align*}
  y &= x^2 - 5x + 6, \tag*{(i)}   \\
  y &= ax + b,       \tag*{(ii)}  \\[-0.6\baselineskip]%[** Hard-coded dim]
% [** TN: Now set the left brace]
\smash{\left\{\begin{aligned} & \\ & \\ & \\ & \\ \end{aligned} \right.} \phantom{-1} & \\[-1.5ex]
 -1 &= 2a + b,       \tag*{(iii)} \\
  a &= 2x - 5,       \tag*{(iv)}
\end{align*}
four equations in $a$, $b$, $x$, $y$.

Equations (i) and~(ii) give $x^2 - 5x + 6 = ax+b$.

Replacing $a$ and~$b$ by their value in this, we get
\[
  x^2 - 5x + 6 = (2x - 5)x - 1 - 2(2x - 5),
\]
which simplifies to $x^2 - 4x + 3 = 0$, the solutions of
which are: $x = 3$ and $x = 1$. Replacing in~(i), we get
$y = 0$ and $y = 2$ respectively; the two points of contact
are then $x = 1$, $y = 2$, and $x = 3$, $y = 0$.

\Paragraph{Note.}---In all exercises dealing with curves, students
will find it extremely instructive to verify the deductions
obtained by actually plotting the curves.
\DPPageSep{103.png}{91}%


\Exercises{VIII} (See \Pageref[page]{AnsEx:VIII} for Answers.)

\begin{Problems}
\Item{(1)} Plot the curve $y = \tfrac{3}{4} x^2 - 5$, using a scale of
millimetres. Measure at points corresponding to
different values of~$x$, the angle of its slope.

Find, by differentiating the equation, the expression
for slope; and see, from a Table of Natural Tangents,
whether this agrees with the measured angle.

\Item{(2)} Find what will be the slope of the curve
\[
y = 0.12x^3 - 2,
\]
at the particular point that has as abscissa $x = 2$.

\Item{(3)} If $y = (x - a)(x - b)$, show that at the particular
point of the curve where $\dfrac{dy}{dx} = 0$, $x$ will have the value
$\tfrac{1}{2} (a + b)$.

\Item{(4)} Find the $\dfrac{dy}{dx}$ of the equation $y = x^3 + 3x$; and
calculate the numerical values of $\dfrac{dy}{dx}$ for the points
corresponding to $x = 0$, $x = \tfrac{1}{2}$, $x = 1$, $x = 2$.

\Item{(5)} In the curve to which the equation is $x^2 + y^2 = 4$,
find the values of~$x$ at those points where the slope ${} = 1$.

\Item{(6)} Find the slope, at any point, of the curve whose
equation is $\dfrac{x^2 }{3^2} + \dfrac{y^2}{2^2} = 1$; and give the numerical value
of the slope at the place where $x = 0$, and at that
where $x = 1$.

\Item{(7)} The equation of a tangent to the curve
$y = 5 - 2x + 0.5x^3$, being of the form $y = mx + n$, where
$m$ and~$n$ are constants, find the value of $m$ and~$n$ if
\DPPageSep{104.png}{92}%
the point where the tangent touches the curve has
$x=2$ for abscissa.

\Item{(8)} At what angle do the two curves
\[
y = 3.5x^2 + 2 \quad \text{and} \quad y = x^2 - 5x + 9.5
\]
cut one another?

\Item{(9)} Tangents to the curve $y = ± \sqrt{25-x^2}$ are drawn
at points for which $x = 3$ and $x = 4$. Find the coordinates
of the point of intersection of the tangents
and their mutual inclination.

\Item{(10)} A straight line $y = 2x - b$ touches a curve
$y = 3x^2 + 2$ at one point. What are the coordinates
of the point of contact, and what is the value of~$b$?
\end{Problems}
\DPPageSep{105.png}{93}%
