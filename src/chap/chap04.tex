
\Chapter{IV}{Simplest Cases}

\First{Now} let us see how, on first principles, we can
differentiate some simple algebraical expression.

\Case{1}
Let us begin with the simple expression $y=x^2$.
Now remember that the fundamental notion about
the calculus is the idea of \emph{growing}. Mathematicians
call it \emph{varying}. Now as $y$~and~$x^2$ are equal to one
another, it is clear that if $x$~grows, $x^2$~will also grow.
And if $x^2$~grows, then $y$~will also grow. What we
have got to find out is the proportion between the
growing of~$y$ and the growing of~$x$. In other words
our task is to find out the ratio between $dy$~and~$dx$,
or, in brief, to find the value of~$\dfrac{dy}{dx}$.

Let $x$, then, grow a little bit bigger and become
$x + dx$; similarly, $y$~will grow a bit bigger and will
become $y + dy$. Then, clearly, it will still be true
that the enlarged~$y$ will be equal to the square of the
enlarged~$x$. Writing this down, we have:
\begin{align*}
y + dy &= (x + dx)^2.
\intertext{\indent Doing the squaring we get:}
y + dy &= x^2 + 2x · dx+(dx)^2.
\end{align*}
\DPPageSep{031.png}{19}%

What does $(dx)^2$ mean? Remember that $dx$ meant
a bit---a little bit---of~$x$. Then $(dx)^2$~will mean a little
bit of a little bit of~$x$; that is, as explained above
(\Pageref{smallness}), it is a small quantity of the second order
of smallness. It may therefore be discarded as quite
inconsiderable in comparison with the other terms.
Leaving it out, we then have:\Pagelabel{diffexample}%
\begin{align*}
y + dy &= x^2 + 2x · dx. \displaybreak[1] \\
\intertext{\indent Now $y=x^2$; so let us subtract this from the equation
and we have left}
dy &= 2x · dx. \displaybreak[1] \\
\intertext{\indent Dividing across by $dx$, we find}
\frac{dy}{dx} &= 2x.
\end{align*}

Now \emph{this}\footnote
  {\NB---This ratio $\dfrac{dy}{dx}$ is the result of differentiating $y$ with
  respect to $x$. Differentiating means finding the differential coefficient.
  Suppose we had some other function of~$x$, as, for
  example, $u = 7x^2 + 3$. Then if we were told to differentiate this
  with respect to~$x$, we should have to find $\dfrac{du}{dx}$, or, what is the same
  thing, $\dfrac{d(7x^2 + 3)}{dx}$. On the other hand, we may have a case in which
  time was the independent variable (see \Pageref{indvar}), such as this:
  $y = b + \frac{1}{2} at^2$. Then, if we were told to differentiate it, that means we
  must find its differential coefficient with respect to~$t$. So that then
  our business would be to try to find $\dfrac{dy}{dt}$, that is, to find
  $\dfrac{d(b + \frac{1}{2} at^2)}{dt}$.}
is what we set out to find. The ratio of
the growing of $y$ to the growing of $x$ is, in the case
before us, found to be $2x$.
\DPPageSep{032.png}{20}%


\Subsection{Numerical example.}
Suppose $x=100$ and $\therefore y=10,000$. Then let $x$ grow
till it becomes $101$ (that is, let $dx=1$). Then the
enlarged~$y$ will be $101 × 101 = 10,201$. But if we agree
that we may ignore small quantities of the second
order, $1$~may be rejected as compared with $10,000$; so
we may round off the enlarged~$y$ to $10,200$. $y$~has
grown from $10,000$ to $10,200$; the bit added on is~$dy$,
which is therefore~$200$.

$\dfrac{dy}{dx} = \dfrac{200}{1} = 200$. According to the algebra-working
of the previous paragraph, we find $\dfrac{dy}{dx} = 2x$. And so
it is; for $x=100$ and $2x=200$.

But, you will say, we neglected a whole unit.

Well, try again, making $dx$ a still smaller bit.

Try $dx=\frac{1}{10}$. Then $x+dx=100.1$, and
\[
(x+dx)^2 = 100.1 × 100.1 = 10,020.01.
\]

Now the last figure $1$ is only one-millionth part of
the $10,000$, and is utterly negligible; so we may
take $10,020$ without the little decimal at the end.
And this makes $dy=20$; and $\dfrac{dy}{dx} = \dfrac{20}{0.1} = 200$, which
is still the same as~$2x$.

\Case{2}
Try differentiating $y = x^3$ in the same way.

We let $y$ grow to $y+dy$, while $x$~grows to~$x+dx$.

Then we have
\[
y + dy = (x + dx)^3.
\]
\DPPageSep{033.png}{21}%

Doing the cubing we obtain
\[
y + dy = x^3 + 3x^2 · dx + 3x(dx)^2+(dx)^3.
\]

Now we know that we may neglect small quantities
of the second and third orders; since, when $dy$~and~$dx$
are both made indefinitely small, $(dx)^2$~and~$(dx)^3$
will become indefinitely smaller by comparison. So,
regarding them as negligible, we have left:
\[
y + dy=x^3+3x^2 · dx.
\]

But $y=x^3$; and, subtracting this, we have:
\begin{DPalign*}
dy &= 3x^2 · dx, \\
\lintertext{and}
\frac{dy}{dx} &= 3x^2.
\end{DPalign*}

\Case{3}
Try differentiating $y=x^4$. Starting as before by
letting both $y$~and~$x$ grow a bit, we have:
\begin{DPalign*}
y + dy &= (x+dx)^4. \displaybreak[1] \\
%
\intertext{\indent Working out the raising to the fourth power, we get}
y + dy &= x^4 + 4x^3\, dx + 6x^2(dx)^2 + 4x(dx)^3+(dx)^4. \displaybreak[1] \\
%
\intertext{\indent Then striking out the terms containing all the
higher powers of~$dx$, as being negligible by comparison,
we have}
y + dy &= x^4+4x^3\, dx. \displaybreak[1] \\
%
\intertext{\indent Subtracting the original $y=x^4$, we have left}
dy &= 4x^3\, dx, \\
\lintertext{and}
\frac{dy}{dx} &= 4x^3.
\end{DPalign*}

\tb
\DPPageSep{034.png}{22}%

Now all these cases are quite easy. Let us collect
the results to see if we can infer any general rule.
Put them in two columns, the values of~$y$ in one
and the corresponding values found for~$\dfrac{dy}{dx}$ in the
other: thus
\[
\begin{array}{|@{\quad}c@{\quad}|@{\quad}l@{\quad}|}
\hline
y   & \DStrut\dfrac{dy}{dx} \\
\hline
x^2 & 2x \Strut \\
x^3 & 3x^2  \\
x^4 & 4x^3  \\
\hline
\end{array}
\]

\Pagelabel{diffrule1}%
Just look at these results: the operation of differentiating
appears to have had the effect of diminishing
the power of~$x$ by~$1$ (for example in the last case
reducing $x^4$ to~$x^3$), and at the same time multiplying
by a number (the same number in fact which originally
appeared as the power). Now, when you have once
seen this, you might easily conjecture how the others
will run. You would expect that differentiating $x^5$
would give~$5x^4$, or differentiating $x^6$ would give~$6x^5$.
If you hesitate, try one of these, and see whether
the conjecture comes right.

Try $y = x^5$.
\begin{DPalign*}
\lintertext{\indent Then}
y+dy &= (x+dx)^5     \\
     &= x^5 + 5x^4\, dx + 10x^3(dx)^2  + 10x^2(dx)^3 \\
     &\phantom{{}= x^5 + 5x^4\, dx} + 5x(dx)^4 + (dx)^5.
\end{DPalign*}

Neglecting all the terms containing small quantities
of the higher orders, we have left
\begin{DPalign*}
y + dy &= x^5 + 5x^4\, dx, \displaybreak[1] \\
\DPPageSep{035.png}{23}%
\lintertext{\rlap{and subtracting}}
y &= x^5 \text{ leaves us} \\
dy &= 5x^4\, dx, \displaybreak[1] \\
\lintertext{whence}
\frac{dy}{dx} &= 5x^4, \text{ exactly as we supposed.}
\end{DPalign*}

\tb

Following out logically our observation, we should
conclude that if we want to deal with any higher
power,---call it $n$---we could tackle it in the same
way.
\begin{DPalign*}
\lintertext{\indent Let}
y &= x^n, \\
\intertext{then, we should expect to find that}
\frac{dy}{dx} &= nx^{(n-1)}.
\end{DPalign*}

For example, let $n=8$, then $y=x^8$; and differentiating
it would give $\dfrac{dy}{dx} = 8x^7$.

And, indeed, the rule that differentiating $x^n$ gives as
the result $nx^{n-1}$ is true for all cases where $n$ is a
whole number and positive. [Expanding $(x + dx)^n$ by
the binomial theorem will at once show this.] But
the question whether it is true for cases where $n$
has negative or fractional values requires further
consideration.


\Subsection{Case of a negative power.}
Let $y = x^{-2}$. Then proceed as before:
\begin{align*}
y+dy &= (x+dx)^{-2} \\
     &= x^{-2} \left(1 + \frac{dx}{x}\right)^{-2}.
\end{align*}
\DPPageSep{036.png}{24}%
Expanding this by the binomial theorem (see \Pageref{binomtheo}),
we get
\begin{align*}
&=x^{-2} \left[1 - \frac{2\, dx}{x} +
    \frac{2(2+1)}{1×2} \left(\frac{dx}{x}\right)^2 -
    \text{etc.}\right]  \\
&=x^{-2} - 2x^{-3} · dx + 3x^{-4}(dx)^2 - 4x^{-5}(dx)^3 + \text{etc.} \\
\intertext{%
\indent So, neglecting the small quantities of higher orders
of smallness, we have:}
       y + dy &= x^{-2} - 2x^{-3} · dx.
\intertext{Subtracting the original $y = x^{-2}$, we find}
           dy &= -2x^{-3}dx,   \\
\frac{dy}{dx} &= -2x^{-3}.
\end{align*}
And this is still in accordance with the rule inferred
above.


\Subsection{Case of a fractional power.}
Let $y= x^{\efrac{1}{2}}$. Then, as before,
\settowidth{\TmpLen}{terms with higher}%
\begin{align*}
y+dy &= (x+dx)^{\efrac{1}{2}} = x^{\efrac{1}{2}}
        \left(1 + \frac{dx}{x} \right)^{\efrac{1}{2}} \\
     &= \sqrt{x} + \frac{1}{2} \frac{dx}{\sqrt{x}} - \frac{1}{8}
        \frac{(dx)^2}{x\sqrt{x}} +
        \raisebox{-1.5ex}{\parbox[c]{\TmpLen}{\begin{center}
          terms with higher\\
          powers of $dx$.\end{center}}}\DPnote{[ **\raisebox optional]}
\end{align*}

Subtracting the original $y = x^{\efrac{1}{2}}$, and neglecting higher
powers we have left:
\[
dy = \frac{1}{2} \frac{dx}{\sqrt{x}} = \frac{1}{2} x^{-\efrac{1}{2}} · dx,
\]
and $\dfrac{dy}{dx} = \dfrac{1}{2} x^{-\efrac{1}{2}}$. Agreeing with the general rule.
\DPPageSep{037.png}{25}%

\Paragraph{Summary.} Let us see how far we have got. We
have arrived at the following rule:\Pagelabel{multipow} To differentiate~$x^n$,
multiply by the power and reduce the power by
one, so giving us~$nx^{n-1}$ as the result.


\Exercises{I} (See \Pageref{AnsEx:I} for Answers.)

Differentiate the following:
\begin{Problems}[2]
\Item{(1)} $y = x^{13}$
\Item{(2)} $y = x^{-\efrac{3}{2}}$
\ResetCols{2}

\Item{(3)} $y = x^{2a}$
\Item{(4)} $u = t^{2.4}$
\ResetCols{2}

\Item{(5)} $z = \sqrt[3]{u}$
\Item{(6)} $y = \sqrt[3]{x^{-5}}$
\ResetCols{2}

\Item{(7)} $u = \sqrt[5]{\dfrac{1}{x^8}}$
\Item{(8)} $y = 2x^a$\DPtypo{.}{}
\ResetCols{2}

\Item{(9)} $y = \sqrt[q]{x^3}$
\Item{(10)} $y = \sqrt[n]{\dfrac{1}{x^m}}$
\end{Problems}

\textit{You have now learned how to differentiate powers
of~$x$. How easy it is!}
\DPPageSep{038.png}{26}%