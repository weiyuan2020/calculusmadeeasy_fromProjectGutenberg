
\Chapter{IX}{Introducing a Useful Dodge}

\First{Sometimes} one is stumped by finding that the expression
to be differentiated is too complicated to
tackle directly.

Thus, the equation\Pagelabel{dodge}
\[
y = (x^2+a^2)^{\efrac{3}{2}}
\]
is awkward to a beginner.

Now the dodge to turn the difficulty is this: Write
some symbol, such as~$u$, for the expression $x^2 + a^2$;
then the equation becomes
\[
y = u^{\efrac{3}{2}},
\]
which you can easily manage; for
\[
\frac{dy}{du} = \frac{3}{2} u^{\efrac{1}{2}}.
\]
Then tackle the expression
\[
u = x^2 + a^2,
\]
and differentiate it with respect to~$x$,
\[
\frac{du}{dx} = 2x.
\]
\DPPageSep{080.png}{68}%
Then all that remains is plain sailing;
\begin{DPalign*}
\lintertext{for}
\frac{dy}{dx} &= \frac{dy}{du} × \frac{du}{dx}; \\
\lintertext{that is,}
\frac{dy}{dx}
  &= \frac{3}{2} u^{\efrac{1}{2}} × 2x \\
  &= \tfrac{3}{2} (x^2 + a^2)^{\efrac{1}{2}} × 2x \\
  &= 3x(x^2 + a^2)^{\efrac{1}{2}};
\end{DPalign*}
and so the trick is done.

By and bye,\DPnote{** TN: [sic], archaic spelling} when you have learned how to deal
with sines, and cosines, and exponentials, you will
find this dodge of increasing usefulness.

\tb

\Examples.
Let us practise this dodge on a few examples.

\Pagelabel{ExNo1}%
(1) Differentiate $y = \sqrt{a+x}$.

Let $a+x = u$.
\begin{align*}
\frac{du}{dx} &= 1;\quad y=u^{\efrac{1}{2}};\quad
  \frac{dy}{du} = \tfrac{1}{2} u^{-\efrac{1}{2}}
                = \tfrac{1}{2} (a+x)^{-\efrac{1}{2}}.\\
%
\frac{dy}{dx} &= \frac{dy}{du} × \frac{du}{dx} = \frac{1}{2\sqrt{a+x}}.
\end{align*}

(2) Differentiate  $y = \dfrac{1}{\sqrt{a+x^2}}$.

Let $a + x^2 = u$.
\begin{align*}
\frac{du}{dx} &= 2x;\quad y=u^{-\efrac{1}{2}};\quad
  \frac{dy}{du} = -\tfrac{1}{2}u^{-\efrac{3}{2}}.\\
%
\frac{dy}{dx} &= \frac{dy}{du}×\frac{du}{dx} = - \frac{x}{\sqrt{(a+x^2)^3}}.
\end{align*}
\DPPageSep{081.png}{69}%

(3) Differentiate $y = \left(m - nx^{\efrac{2}{3}} + \dfrac{p}{x^{\efrac{4}{3}}}\right)^a$.

Let $m - nx^{\efrac{2}{3}} + px^{-\efrac{4}{3}} = u$.
\begin{gather*}
\frac{du}{dx} = -\tfrac{2}{3} nx^{-\efrac{1}{3}} - \tfrac{4}{3} px^{-\efrac{7}{3}};\\
%
y = u^a;\quad \frac{dy}{du} = a u^{a-1}. \\
%
\frac{dy}{dx} = \frac{dy}{du}×\frac{du}{dx}
  = -a\left(m -nx^{\efrac{2}{3}} + \frac{p}{x^{\efrac{4}{3}}}\right)^{a-1}
     (\tfrac{2}{3} nx^{-\efrac{1}{3}} + \tfrac{4}{3} px^{-\efrac{7}{3}}).
\end{gather*}

(4) Differentiate $y=\dfrac{1}{\sqrt{x^3 - a^2}}$.

Let $u = x^3 - a^2$.
\begin{align*}
\frac{du}{dx} &= 3x^2;\quad y = u^{-\efrac{1}{2}};\quad
  \frac{dy}{du}=-\frac{1}{2}(x^3 - a^2)^{-\efrac{3}{2}}. \\
\frac{dy}{dx} &= \frac{dy}{du} × \frac{du}{dx} = -\frac{3x^2}{2\sqrt{(x^3 - a^2)^3}}.
\end{align*}

\Pagelabel{examples4}
(5) Differentiate $y=\sqrt{\dfrac{1-x}{1+x}}$.

Write this as $y=\dfrac{(1-x)^{\efrac{1}{2}}}{(1+x)^{\efrac{1}{2}}}$.
\[
\frac{dy}{dx}
  = \frac{(1+x)^{\efrac{1}{2}}\, \dfrac{d(1-x)^{\efrac{1}{2}}}{dx}
        - (1-x)^{\efrac{1}{2}}\, \dfrac{d(1+x)^{\efrac{1}{2}}}{dx}}{1+x}.
\]

(We may also write $y = (1-x)^{\efrac{1}{2}} (1+x)^{-\efrac{1}{2}}$ and differentiate
as a product.)
\DPPageSep{082.png}{70}%

Proceeding as in \DPtypo{exercise}{example}~(1) above, we get
\[
\frac{d(1-x)^{\efrac{1}{2}}}{dx} = -\frac{1}{2\sqrt{1-x}};
\quad\text{and}\quad
\frac{d(1+x)^{\efrac{1}{2}}}{dx} = \frac{1}{2\sqrt{1+x}}.
\]

Hence
\begin{DPalign*}
\frac{dy}{dx}
  &= - \frac{(1 + x)^{\efrac{1}{2}}}{2(1 + x)\sqrt{1-x}}
     - \frac{(1 - x)^{\efrac{1}{2}}}{2(1 + x)\sqrt{1+x}} \\
  &= - \frac{1}{2\sqrt{1+x}\sqrt{1-x}} - \frac{\sqrt{1-x}}{2 \sqrt{(1+x)^3}};\\
\lintertext{or}
\frac{dy}{dx}
  &= - \frac{1}{(1+x)\sqrt{1-x^2}}.
\end{DPalign*}

(6) Differentiate $y = \sqrt{\dfrac{x^3}{1+x^2}}$.

We may write this
\begin{gather*}
y = x^{\efrac{3}{2}}(1+x^2)^{-\efrac{1}{2}}; \\
\frac{dy}{dx}
  = \tfrac{3}{2} x^{\efrac{1}{2}}(1 + x^2)^{-\efrac{1}{2}}
  + x^{\efrac{3}{2}} × \frac{d\bigl[(1+x^2)^{-\efrac{1}{2}}\bigr]}{dx}.
\end{gather*}

Differentiating $(1+x^2)^{-\efrac{1}{2}}$, as shown in \DPtypo{exercise}{example}~(2)
above, we get
\[
\frac{d\bigl[(1+x^2)^{-\efrac{1}{2}}\bigr]}{dx} = - \frac{x}{\sqrt{(1+x^2)^3}};
\]
so that
\[
\frac{dy}{dx}
  = \frac{3\sqrt{x}}{2\sqrt{1+x^2}} - \frac{\sqrt{x^5}}{\sqrt{(1+x^2)^3}}
  = \frac{\sqrt{x}(3+x^2)}{2\sqrt{(1+x^2)^3}}.
\]
\DPPageSep{083.png}{71}%

(7) Differentiate $y=(x+\sqrt{x^2+x+a})^3$.

Let $x+\sqrt{x^2+x+a}=u$.
\begin{gather*}
\frac{du}{dx} = 1 + \frac{d\bigl[(x^2+x+a)^{\efrac{1}{2}}\bigr]}{dx}. \\
y = u^3;\quad\text{and}\quad \frac{dy}{du} = 3u^2= 3\left(x+\sqrt{x^2+x+a}\right)^2.
\end{gather*}

Now let $(x^2+x+a)^{\efrac{1}{2}}=v$ and $(x^2+x+a) = w$.
\begin{DPalign*}
\frac{dw}{dx}
  &= 2x+1;\quad v = w^{\efrac{1}{2}};\quad \frac{dv}{dw} = \tfrac{1}{2}w^{-\efrac{1}{2}}. \\
\frac{dv}{dx}
  &= \frac{dv}{dw} × \frac{dw}{dx} = \tfrac{1}{2}(x^2+x+a)^{-\efrac{1}{2}}(2x+1). \\
\lintertext{\indent Hence}
\frac{du}{dx}
  &= 1 + \frac{2x+1}{2\sqrt{x^2+x+a}}, \\
\frac{dy}{dx}
  &= \frac{dy}{du} × \frac{du}{dx}\\
  &= 3\left(x+\sqrt{x^2+x+a}\right)^2
      \left(1 +\frac{2x+1}{2\sqrt{x^2+x+a}}\right).
\end{DPalign*}

(8) Differentiate $y=\sqrt{\dfrac{a^2+x^2}{a^2-x^2}} \sqrt[3]{\dfrac{a^2-x^2}{a^2+x^2}}$.

We get
\begin{align*}
y &= \frac{(a^2+x^2)^{\efrac{1}{2}} (a^2-x^2)^{\efrac{1}{3}}}
          {(a^2-x^2)^{\efrac{1}{2}} (a^2+x^2)^{\efrac{1}{3}}}
  = (a^2+x^2)^{\efrac{1}{6}} (a^2-x^2)^{-\efrac{1}{6}}. \\
\frac{dy}{dx}
  &= (a^2+x^2)^{\efrac{1}{6}} \frac{d\bigl[(a^2-x^2)^{-\efrac{1}{6}}\bigr]}{dx}
   + \frac{d\bigl[(a^2+x^2)^{\efrac{1}{6}}\bigr]}{(a^2-x^2)^{\efrac{1}{6}}\, dx}.
\end{align*}
\DPPageSep{084.png}{72}%

Let $u = (a^2-x^2)^{-\efrac{1}{6}}$ and $v = (a^2 - x^2)$.
\begin{align*}
u &= v^{-\efrac{1}{6}};\quad
  \frac{du}{dv} = -\frac{1}{6}v^{-\efrac{7}{6}};\quad
  \frac{dv}{dx} = -2x. \\
%
\frac{du}{dx} &= \frac{du}{dv} × \frac{dv}{dx} = \frac{1}{3}x(a^2-x^2)^{-\efrac{7}{6}}.
\end{align*}

Let $w = (a^2 + x^2)^{\efrac{1}{6}}$ and $z = (a^2 + x^2)$.
\begin{align*}
w &= z^{\efrac{1}{6}};\quad
  \frac{dw}{dz} = \frac{1}{6}z^{-\efrac{5}{6}};\quad
  \frac{dz}{dx} = 2x. \\
%
\frac{dw}{dx} &= \frac{dw}{dz} × \frac{dz}{dx} = \frac{1}{3} x(a^2 + x^2)^{-\efrac{5}{6}}.
\end{align*}

Hence
\begin{DPalign*}
\frac{dy}{dx}
  &= (a^2+x^2)^{\efrac{1}{6}} \frac{x}{3(a^2-x^2)^{\efrac{7}{6}}}
   + \frac{x}{3(a^2-x^2)^{\efrac{1}{6}} (a^2+x^2)^{\efrac{5}{6}}}; \\
%
\lintertext{or}
\frac{dy}{dx}
  &= \frac{x}{3}
     \left[\sqrt[6]{\frac{a^2+x^2}{(a^2-x^2)^7}}
           + \frac{1}{\sqrt[6]{(a^2-x^2)(a^2+x^2)^5]}} \right].
\end{DPalign*}

(9) Differentiate $y^n$ with respect to~$y^5$.
\[
\frac{d(y^n)}{d(y^5)} = \frac{ny^{n-1}}{5y^{5-1}} = \frac{n}{5} y^{n-5}.
\]

%[** TN: Manual linebreak improves typeset appearance]
(10)\Pagelabel{Example10} Find the first and second differential coefficients \\
of $y = \dfrac{x}{b} \sqrt{(a-x)x}$.
\[
\frac{dy}{dx}
  = \frac{x}{b}\,
    \frac{d\bigl\{\bigl[(a-x)x\bigr]^{\efrac{1}{2}}\bigr\}}{dx}
  + \frac{\sqrt{(a-x)x}}{b}.
\]

Let $\bigl[(a-x)x\bigr]^{\efrac{1}{2}} = u$ and let $(a-x)x = w$; then $u = w^{\efrac{1}{2}}$.
\BindMath{\[
\frac{du}{dw}
  = \frac{1}{2} w^{-\efrac{1}{2}}
  = \frac{1}{2w^{\efrac{1}{2}}} = \frac{1}{2\sqrt{(a-x)x}}.
\]
\DPPageSep{085.png}{73}%
\begin{align*}
&\frac{dw}{dx} = a-2x.\\
&\frac{du}{dw} × \frac{dw}{dx} = \frac{du}{dx} = \frac{a-2x}{2\sqrt{(a-x)x}}.
\end{align*}}%

Hence
\[
\frac{dy}{dx}
  = \frac{x(a-2x)}{2b\sqrt{(a-x)x}} + \frac{\sqrt{(a-x)x}}{b}
  = \frac{x(3a-4x)}{2b\sqrt{(a-x)x}}.
\]

Now
\begin{align*}
\frac{d^2y}{dx^2}
  &= \frac{2b \sqrt{(a-x)x}\, (3a-8x)
           - \dfrac{(3ax-4x^2)b(a-2x)}{\sqrt{(a-x)x}}}
          {4b^2(a-x)x} \\
  &= \frac{3a^2-12ax+8x^2}{4b(a-x)\sqrt{(a-x)x}}.
\end{align*}

(We shall need these two last differential coefficients
later on. See \hyperref[Ex:X11]{Ex.~X. No.~11}.)


\Exercises{VI} (See \Pageref[page]{AnsEx:VI} for Answers.)

Differentiate the following:

\begin{Problems}[2]
\Item{(1)} $y = \sqrt{x^2 + 1}$.
\Item{(2)} $y = \sqrt{x^2+a^2}$.
\ResetCols{2}

\Item{(3)} $y = \dfrac{1}{\sqrt{a+x}}$.
\Item{(4)} $y = \dfrac{a}{\sqrt{a-x^2}}$.
\ResetCols{2}

\Item{(5)} $y = \dfrac{\sqrt{x^2-a^2}}{x^2}$.
\Item{(6)} $y = \dfrac{\sqrt[3]{x^4+a}}{\sqrt[2]{x^3+a}}$.
\ResetCols{1}

\Item{(7)} $y = \dfrac{a^2+x^2}{(a+x)^2}$.
\DPPageSep{086.png}{74}%

\Item{(8)} Differentiate $y^5$ with respect to~$y^2$.

\Item{(9)} Differentiate $y = \dfrac{\sqrt{1 - \theta^2}}{1 - \theta}$.
\end{Problems}
\tb

The process can be extended to three or more
differential coefficients, so that $\dfrac{dy}{dx} = \dfrac{dy}{dz} × \dfrac{dz}{dv} × \dfrac{dv}{dx}$.


\Examples.
(1) If $z = 3x^4$;\quad $v = \dfrac{7}{z^2}$;\quad $y =\sqrt{1+v}$, find~$\dfrac{dv}{dx}$.

We have
\begin{align*}
\frac{dy}{dv} &= \frac{1}{2\sqrt{1+v}};\quad
\frac{dv}{dz} = -\frac{14}{z^3};\quad
\frac{dz}{dx} = 12x^3. \\
%
\frac{dy}{dx} &= -\frac{168x^3}{(2\sqrt{1+v})z^3}
               = -\frac{28}{3x^5\sqrt{9x^8+7}}.
\end{align*}

(2) If $t = \dfrac{1}{5\sqrt{\theta}}$;\quad $x = t^3 + \dfrac{t}{2}$;\quad $v = \dfrac{7x^2}{\sqrt[3]{x-1}}$, find~$\dfrac{dv}{d\theta}$.
\begin{DPgather*}
\frac{dv}{dx} = \frac{7x(5x-6)}{3\sqrt[3]{(x-1)^4}};\quad
\frac{dx}{dt} = 3t^2 + \tfrac{1}{2};\quad
\frac{dt}{d\theta} = -\frac{1}{10\sqrt{\theta^3}}. \\
\lintertext{Hence}
\frac{dv}{d\theta}
  = -\frac{7x(5x-6)(3t^2+\frac{1}{2})}
          {30\sqrt[3]{(x-1)^4} \sqrt{\theta^3}},
\end{DPgather*}
an expression in which $x$ must be replaced by its
value, and $t$ by its value in terms of~$\theta$.

(3) If $\theta = \dfrac{3a^2x}{\sqrt{x^3}}$;\quad $\omega = \dfrac{\sqrt{1-\theta^2}}{1+\theta}$;\quad and $\phi = \sqrt{3} - \dfrac{1}{\omega\sqrt{2}}$,
find~$\dfrac{d\phi}{dx}$.
\DPPageSep{087.png}{75}%

We get
\begin{gather*}
\theta = 3a^2x^{-\efrac{1}{2}};\quad
\omega = \sqrt{\frac{1-\theta}{1+\theta}};\quad \text{and}\quad
\phi = \sqrt{3} \DPtypo{=}{-} \frac{1}{\sqrt{2}} \omega^{-1}. \\
\frac{d\theta}{dx} = -\frac{3a^2}{2\sqrt{x^3}};\quad
\frac{d\omega}{d\theta} = -\frac{1}{(1+\theta)\sqrt{1-\theta^2}}
\end{gather*}
(see example~5, \Pageref{examples4}); and
\[
\frac{d\phi}{d\omega} = \frac{1}{\sqrt{2}\omega^2}.
\]

So that $\dfrac{d\theta}{dx} = \dfrac{1}{\sqrt{2} × \omega^2}
  × \dfrac{1}{(1+\theta) \sqrt{1-\theta^2}}
  × \dfrac{3a^2}{2\sqrt{x^3}}$.

Replace now first~$\omega$, then~$\theta$ by its value.


\Exercises{VII}
%[ ** TN: Inconsistent formatting in original; heading on separate line]
You can now successfully try the following. (See
\Pageref[page]{AnsEx:VII} for Answers.)
\begin{Problems}
\Item{(1)} If $u = \frac{1}{2}x^3$;\quad $v = 3(u+u^2)$;\quad and $w = \dfrac{1}{v^2}$, find~$\dfrac{dw}{dx}$.

\Item{(2)} If $y = 3x^2 + \sqrt{2}$;\quad $z = \sqrt{1+y}$;\quad and $v = \dfrac{1}{\sqrt{3}+4z}$,
find~$\dfrac{dv}{dx}$.

\Item{(3)} If $y = \dfrac{x^3}{\sqrt{3}}$;\quad $z = (1+y)^2$;\quad and $u = \dfrac{1}{\sqrt{1+z}}$, find~$\dfrac{du}{dx}$.
\end{Problems}
\DPPageSep{088.png}{76}%

