
\Chapter[DIFFERENT DEGREES OF SMALLNESS]
{II}{On Different Degrees of Smallness}

\First{We} shall find that in our processes of calculation we
have to deal with small quantities of various degrees
of smallness.

We shall have also to learn under what circumstances
we may consider small quantities to be so minute
that we may omit them from consideration. Everything
depends upon relative minuteness.

Before we fix any rules let us think of some
familiar cases. There are $60$~minutes in the hour,
$24$~hours in the day, $7$~days in the week. There are
therefore $1440$~minutes in the day and $10080$~minutes
in the week.

Obviously $1$~minute is a very small quantity of
time compared with a whole week. Indeed, our
forefathers considered it small as compared with an
% [** TN: [sic] accent, though "minùte" seems to be neither French nor Latin]
hour, and called it ``one minùte,'' meaning a minute
fraction---namely one sixtieth---of an hour. When
they came to require still smaller subdivisions of time,
they divided each minute into $60$ still smaller parts,
which, in Queen Elizabeth's days, they called ``second
minùtes''\DPnote{** TN: [sic]} (\IE~small quantities of the second order of
minuteness). Nowadays we call these small quantities
\DPPageSep{016.png}{4}%
of the second order of smallness ``seconds.'' But few
people know \emph{why} they are so called.

Now if one minute is so small as compared with a
whole day, how much smaller by comparison is one
second!

Again, think of a farthing as compared with a
sovereign: it is barely worth more than $\frac{1}{1000}$ part.
A farthing more or less is of precious little importance
compared with a sovereign: it may certainly be regarded
as a \emph{small} quantity. But compare a farthing
with £$1000$: relatively to this greater sum, the
farthing is of no more importance than $\frac{1}{1000}$ of a
farthing would be to a sovereign. Even a golden
sovereign is relatively a negligible quantity in the
wealth of a millionaire.

Now if we fix upon any numerical fraction as
constituting the proportion which for any purpose
we call relatively small, we can easily state other
fractions of a higher degree of smallness. Thus if,
for the purpose of time, $\frac{1}{60}$ be called a \emph{small} fraction,
then $\frac{1}{60}$ of $\frac{1}{60}$ (being a \emph{small} fraction of a \emph{small}
fraction) may be regarded as a \emph{small quantity of the
second order}\Pagelabel{smallness} of smallness.\footnote
  {The mathematicians talk about the second order of ``magnitude''
  (\IE~greatness) when they really mean second order of \emph{smallness}.
  This is very confusing to beginners.}

Or, if for any purpose we were to take $1$~per~cent.\
(\IE~$\frac{1}{100}$) as a \emph{small} fraction, then $1$~per~cent.\ of
$1$~per~cent.\ (\IE~$\frac{1}{10,000}$) would be a small fraction
of the second order of smallness; and $\frac{1}{1,000,000}$ would
\DPPageSep{017.png}{5}%
be a small fraction of the third order of smallness,
being $1$~per~cent.\ of $1$~per~cent.\ of $1$~per~cent.

Lastly, suppose that for some very precise purpose
we should regard $\frac{1}{1,000,000}$ as ``small.'' Thus, if a
first-rate chronometer is not to lose or gain more than
half a minute in a year, it must keep time with an
accuracy of $1$~part in $1,051,200$. Now if, for such a
purpose, we regard $\frac{1}{1,000,000}$ (or one millionth) as a
small quantity, then $\frac{1}{1,000,000}$ of $\frac{1}{1,000,000}$, that is
$\frac{1}{1,000,000,000,000}$ (or one billionth) will be a small
quantity of the second order of smallness, and may
be utterly disregarded, by comparison.

Then we see that the smaller a small quantity itself
is, the more negligible does the corresponding small
quantity of the second order become. Hence we
know that \emph{in all cases we are justified in neglecting
the small quantities of the second---or third \emph{(or
higher)}---orders}, if only we take the small quantity
of the first order small enough in itself.

But, it must be remembered, that small quantities
if they occur in our expressions as factors multiplied
by some other factor, may become important if the
other factor is itself large. Even a farthing becomes
important if only it is multiplied by a few hundred.

Now in the calculus we write $dx$ for a little bit
of~$x$. These things such as~$dx$, and~$du$, and~$dy$, are
called ``differentials,'' the differential of~$x$, or of~$u$,
or of~$y$, as the case may be. [You \emph{read} them as
\emph{dee-eks}, or \emph{dee-you}, or \emph{dee-wy}.] If $dx$ be a small bit
of~$x$, and relatively small of itself, it does not follow
\DPPageSep{018.png}{6}%
that such quantities as $x · dx$, or $x^2\, dx$, or $a^x\, dx$ are
negligible. But $dx × dx$ would be negligible, being a
small quantity of the second order.

A very simple example will serve as illustration.

Let us think of $x$ as a quantity that can grow by
a small amount so as to become $x + dx$, where $dx$~is
the small increment added by growth. The square
of this is $x^2 + 2x · dx + (dx)^2$. The second term is
not negligible because it is a first-order quantity;
while the third term is of the second order of smallness,
being a bit of,
a bit of $x^2$. Thus if we took
$dx$ to mean numerically, say, $\frac{1}{60}$~of~$x$, then the second
term would be $\frac{2}{60}$~of~$x^2$, whereas the third term would
be $\frac{1}{3600}$~of~$x^2$. This last term is clearly less important
than the second. But if we go further and take
$dx$ to mean only $\frac{1}{1000}$~of~$x$, then the second term
will be $\frac{2}{1000}$~of~$x^2$, while the third term will be
only $\frac{1}{1,000,000}$~of~$x^2$.

\Figure[1.5in]{018a}{1}

Geometrically this may be depicted as follows:
Draw a square (\Fig{1}) the side of which we will
take to represent~$x$. Now suppose the square to
grow by having a bit~$dx$ added to its size each
\DPPageSep{019.png}{7}%
way. The enlarged square is made up of the original
square~$x^2$, the two rectangles at the top and on the
right, each of which is of area $x · dx$ (or together
$2x · dx$), and the little square at the top right-hand
corner which is~$(dx)^2$. In \Fig{2} we have taken $dx$ as
\Figures[]{019a}{019b}{2}{3}
quite a big fraction of $x$---about~$\frac{1}{5}$. But suppose we
had taken it only $\frac{1}{100}$---about the thickness of an
inked line drawn with a fine pen. Then the little
corner square will have an area of only $\frac{1}{10,000}$ of~$x^2$,
and be practically invisible. Clearly $(dx)^2$ is negligible
if only we consider the increment~$dx$ to be itself
small enough.

Let us consider a simile.

Suppose a millionaire were to say to his secretary:
next week I will give you a small fraction of any
money that comes in to me. Suppose that the
secretary were to say to his boy: I will give you a
small fraction of what I get. Suppose the fraction
in each case to be $\frac{1}{100}$ part. Now if Mr.~Millionaire
received during the next week £$1000$, the secretary
\DPPageSep{020.png}{8}%
would receive £$10$ and the boy $2$~shillings. Ten
pounds would be a small quantity compared with
£$1000$; but two shillings is a small small quantity
indeed, of a very secondary order. But what would
be the disproportion if the fraction, instead of being~$\frac{1}{100}$,
had been settled at $\frac{1}{1000}$ part? Then, while
Mr.~Millionaire got his £$1000$, Mr.~Secretary would
get only~£$1$, and the boy less than one farthing!

The witty Dean Swift\footnote
  {\textit{On Poetry: a Rhapsody} (p.~20), printed 1733---usually misquoted.}
once wrote:\enlargethispage{12pt}
\begin{center}\small%
\settowidth{\TmpLen}{``And these have smaller Fleas to bite 'em,}%
\begin{minipage}{\TmpLen}\raggedright
``So, Nat'ralists observe, a Flea\\
``Hath smaller Fleas that on him prey.\\
``And these have smaller Fleas to bite 'em,\\
``And so proceed \textit{ad infinitum}.''
\end{minipage}
\end{center}

An ox might worry about a flea of ordinary
size---a small creature of the first order of smallness.
But he would probably not trouble himself about a
flea's flea; being of the second order of smallness, it
would be negligible. Even a gross of fleas' fleas
would not be of much account to the ox.
\DPPageSep{021.png}{9}%
