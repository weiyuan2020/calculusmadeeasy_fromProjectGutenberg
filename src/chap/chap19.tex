

\Chapter[FINDING AREAS BY INTEGRATING]{XIX}{On Finding Areas by Integrating}

\First{One} use of the integral calculus is to enable us to
ascertain the values of areas bounded by curves.

Let us try to get at the subject bit by bit.

\Figure[2.5in]{218a}{52}

Let $AB$ (\Fig{52}) be a curve, the equation to which
is known. That is, $y$~in this curve is some known
function of~$x$. Think of a piece of the curve from
the point~$P$ to the point~$Q$.

Let a perpendicular~$PM$ be dropped from~$P$, and
another~$QN$ from the point~$Q$. Then call $OM = x_1$
and $ON = x_2$, and the ordinates $PM = y_1$ and $QN = y_2$.
We have thus marked out the area~$PQNM$ that lies
\DPPageSep{219.png}{207}%
beneath the piece~$PQ$. The problem is, \emph{how can we
calculate the value of this area}?

%[Illustration]

The secret of solving this problem is to conceive
the area as being divided up into a lot of narrow
strips, each of them being of the width~$dx$. The
smaller we take~$dx$, the more of them there will be
between $x_1$ and~$x_2$. Now, the whole area is clearly
equal to the sum of the areas of all such strips. Our
business will then be to discover an expression for
the area of any one narrow strip, and to integrate it
so as to add together all the strips. Now think of
any one of the strips. It
\begin{wrapfigure}[7]{r}{0.5in}
  \hfill\smash[t]{\raisebox{-1.375in}{\Graphic[0.375in]{219a}}}
\end{wrapfigure}
will be like this:
being bounded between two vertical sides, with
a flat bottom~$dx$, and with a slightly curved
sloping top. Suppose we take its \emph{average}
height as being~$y$; then, as its width is~$dx$, its
area will be~$y\, dx$. And seeing that we may
take the width as narrow as we please, if we
only take it narrow enough its average height will be
the same as the height at the middle of it. Now
let us call the unknown value of the whole area~$S$,
meaning surface. The area of one strip will be
simply a bit of the whole area, and may therefore
be called~$dS$. So we may write
\[
\text{area of $1$~strip} = dS = y · dx.
\]
If then we add up all the strips, we get
\[
\text{total area~$S$} = \int dS = \int y\, dx.
\]

So then our finding $S$ depends on whether we can
\DPPageSep{220.png}{208}%
integrate $y · dx$ for the particular case, when we know
what the value of~$y$ is as a function of~$x$.

For instance, if you were told that for the particular
curve in question $y = b + ax^2$, no doubt you could put
that value into the expression and say: then I must
find $\ds\int (b + ax^2)\, dx$. %[ ** \displaystyle]

That is all very well; but a little thought will show
you that something more must be done. Because the
area we are trying to find is not the area under the
whole length of the curve, but only the area limited
on the left by~$PM$, and on the right by~$QN$, it follows
that we must do something to define our area between
those `\emph{limits}.'\Pagelabel{limits} %[ ** F2: Single quotes, elsewhere double]

This introduces us to a new notion, namely that of
\emph{integrating between limits}. We suppose $x$ to vary,
and for the present purpose we do not require any
value of~$x$ below~$x_1$ (that is~$OM$), nor any value of~$x$
above~$x_2$ (that is~$ON$). When an integral is to be
thus defined between two limits, we call the lower
of the two values \emph{the inferior limit}, and the upper
value \emph{the superior limit}. Any integral so limited
we designate as a \emph{definite integral}, by way of distinguishing
it from a \emph{general integral} to which no
limits are assigned.

In the symbols which give instructions to integrate,
the limits are marked by putting them at the top
and bottom respectively of the sign of integration.
Thus the instruction
\[
\int_{x=x_1}^{x=x_2} y · dx
\]
\DPPageSep{221.png}{209}%
will be read: find the integral of~$y · dx$ between the
inferior limit~$x_1$ and the superior limit~$x_2$.

Sometimes the thing is written more simply
\[
\int^{x_2}_{x_1} y · dx.
\]
Well, but \emph{how} do you find an integral between limits,
when you have got these instructions?

Look again at \Fig{52} (\Pageref{fig:52}). Suppose we could %[ ** Page]
find the area under the larger piece of curve from
$A$ to~$Q$, that is from $x = 0$ to~$x = x_2$, naming the area
$AQNO$. Then, suppose we could find the area under
the smaller piece from $A$ to~$P$, that is from $x = 0$ to
$x = x_1$, namely the area $APMO$. If then we were to
subtract the smaller area from the larger, we should
have left as a remainder the area $PQNM$, which is
what we want. Here we have the clue as to what
to do; the definite integral between the two limits is
\emph{the difference} between the integral worked out for
the superior limit and the integral worked out for the
lower limit.

Let us then go ahead. First, find the general
integral thus:
\[
\int y\, dx,
\]
and, as $y = b + ax^2$ is the equation to the curve (\Fig{52}),
\[
\int (b + ax^2)\, dx
\]
is the general integral which we must find.

Doing the integration in question by the rule
(\Pageref{section:9}), we get
\[
bx + \frac{a}{3} x^3 + C;
\]
\DPPageSep{222.png}{210}%
and this will be the whole area from~$0$ up to any
value of~$x$ that we may assign.

Therefore, the larger area up to the superior limit~$x_2$
will be
\[
bx_2 + \frac{a}{3} x_2^3 + C;
\]
and the smaller area up to the inferior limit~$x_1$ will be
\[
bx_1 + \frac{a}{3} x_1^3 + C.
\]

Now, subtract the smaller from the larger, and we
get for the area~$S$ the value,
\[
\text{area~$S$} = b(x_2 - x_1) + \frac{a}{3}(x_2^3 - x_1^3).
\]

This is the answer we wanted. Let us give some
numerical values. Suppose $b = 10$, $a = 0.06$, and $x_2 = 8$
and $x_1 = 6$. Then the area~$S$ is equal to
\begin{gather*}
10(8 - 6) + \frac{0.06}{3} (8^3 - 6^3) \\
\begin{aligned}
&= 20 + 0.02(512 - 216)    \\
&= 20 + 0.02 × 296    \\
&= 20 + 5.92     \\
&= 25.92.
\end{aligned}
\end{gather*}

Let us here put down a symbolic way of stating
what we have ascertained about limits:
\[
\int^{x=x_2}_{x=x_1} y\, dx = y_2 - y_1,
\]
where $y_2$ is the integrated value of~$y\, dx$ corresponding
to~$x_2$, and $y_1$~that corresponding to~$x_1$.
\DPPageSep{223.png}{211}%

All integration between limits requires the difference
between two values to be thus found. Also note
that, in making the subtraction the added constant~$C$
has disappeared.


\Examples.
(1) To familiarize ourselves with the process, let us
take a case of which we know the answer beforehand.
Let us find the area of the triangle (\Fig{53}), which
has base $x = 12$ and height $y = 4$. We know beforehand,
from obvious mensuration, that the answer will
come~$24$.

\Figure{223a}{53}

Now, here we have as the ``curve'' a sloping line
for which the equation is
\[
y = \frac{x}{3}.
\]

The area in question will be
\[
\int^{x=12}_{x=0} y · dx = \int^{x=12}_{x=0} \frac{x}{3} · dx.
\]

Integrating $\dfrac{x}{3}\, dx$ (\Pageref{diffrule}), and putting down the %[ ** Page]
\DPPageSep{224.png}{212}%
value of the general integral in square brackets with
the limits marked above and below, we get
\begin{align*}
\text{area} %[ ** F2: Nowadays the + C would be inside the brackets]
  &= \left[ \frac{1}{3} · \frac{1}{2} x^2 \right]^{x=12}_{x=0} + C \\
  &= \left[ \frac{x^2}{6} \right]^{x=12}_{x=0} + C  \\
  &= \left[ \frac{12^2}{6} \right] - \left[ \frac{0^2}{6} \right] \\
  &= \frac{144}{6} = 24.\quad \textit{Ans}.
\end{align*}

Let us satisfy ourselves about this rather surprising
dodge of calculation, by testing it on a simple
example. Get some squared paper, preferably some
\Figure[2.75in]{224a}{54}
that is ruled in little squares of one-eighth inch or
one-tenth inch each way.  On this squared paper
plot out the graph of the equation,
\[
y = \frac{x}{3}.
\]

The values to be plotted will be:
\[
\begin{array} {|c|| *{5}{c|}}
\hline
\Strut
\Td[c]{x} & \Td[c]{0} & \Td[c]{3} & \Td[c]{6} & \Td[c]{9} & \Td{12} \\
\hline
\Strut
\Td[c]{y} & \Td[c]{0} & \Td[c]{1} & \Td[c]{2} & \Td[c]{3}  & \Td{4} \\
\hline
\end{array}
\]

The plot is given in \Fig{54}.
\DPPageSep{225.png}{213}%

Now reckon out the area beneath the curve \emph{by
counting the little squares} below the line, from $x = 0$
as far as $x = 12$ on the right. There are $18$~whole
squares and four triangles, each of which has an area
equal to $1\frac{1}{2}$~squares; or, in total, $24$~squares. Hence
$24$~is the numerical value of the integral of $\dfrac{x}{3}\, dx$
between the lower limit of $x = 0$ and the higher limit
of $x = 12$.

As a further exercise, show that the value of the
same integral between the limits of $x = 3$ and $x = 15$
is~$36$.

\Figure[2.75in]{225a}{55}
(2) Find the area, between limits $x = x_1$ and $x = 0$,
of the curve $y = \dfrac{b}{x + a}$.
\begin{align*}
\text{Area}
  &= \int^{x=x_1}_{x=0} y · dx
   = \int^{x=x_1}_{x=0} \frac{b}{x+a}\, dx \displaybreak[1] \\
\DPPageSep{226.png}{214}%
  &= b \bigl[\log_\epsilon(x + a) \bigr]^{x_1} _{0} + C \displaybreak[1] \\
  &= b \bigl[\log_\epsilon(x_1 + a) - \log_\epsilon(0 + a)\bigr] \displaybreak[1] \\
  &= b \log_\epsilon \frac{x_1 + a}{a}.\quad \textit{Ans}.
\end{align*}

\NB---Notice that in dealing with definite integrals
the constant~$C$ always disappears by subtraction.

Let it be noted that this process of subtracting one
part from a larger to find the difference is really a
common practice. How do you find the area of a
\Figure[1.5in]{226a}{56}
plane ring (\Fig{56}), the outer radius of which is~$r_2$
and the inner radius is~$r_1$? You know from mensuration
that the area of the outer circle is~$\pi r_2^2$; then
you find the area of the inner circle,~$\pi r_1^2$; then you
subtract the latter from the former, and find area of
ring $= \pi(r_2^2 - r_1^2)$; which may be written
\[
\pi(r_2 + r_1)(r_2 - r_1)
\]
$= \text{mean circumference of ring} × \text{width of ring}$.
\DPPageSep{227.png}{215}%

(3) Here's another case---that of the \emph{die-away curve}
(\Pageref{section:5}). Find the area between $x = 0$ and $x = a$, of
the curve (\Fig{57}) whose equation is
\begin{align*}
y &= b\epsilon^{-x}. \\
\text{Area}
  &= b\int^{x=a} _{x=0} \epsilon^{-x} · dx. \displaybreak[1] \\
\intertext{\indent The integration (\Pageref{differ3}) gives}
  &= b\left[-\epsilon^{-x}\right]^a _0 \\
  &= b\bigl[-\epsilon^{-a} - (-\epsilon^{-0})\bigr] \\
  &= b(1-\epsilon^{-a}).
\end{align*}

\Figures{227a}{227b}{57}{58}

(4) Another example is afforded by the adiabatic
curve of a perfect gas, the equation to which is
$pv^n = c$, where $p$~stands for pressure, $v$~for volume,
and $n$~is of the value~$1.42$ (\Fig{58}).

Find the area under the curve (which is proportional
\DPPageSep{228.png}{216}%
to the work done in suddenly compressing the gas)
from volume~$v_2$ to volume~$v_1$.

Here we have
\begin{align*}
\text{area}
  &= \int^{v=v_2}_{v=v_1} cv^{-n} · dv \\
  &= c\left[\frac{1}{1-n} v^{1-n} \right]^{v_2} _{v_1} \\
  &= c \frac{1}{1-n} (v_2^{1-n} - v_1^{1-n}) \\
  &= \frac{-c}{0.42}\left(\frac{1}{v_2^{0.42}} - \frac{1}{v_1^{0.42}}\right).
\end{align*}


\Subsection{An Exercise.}
Prove the ordinary mensuration formula, that the
area~$A$ of a circle whose radius is~$R$, is equal to~$\pi R^2$.

\Figure{228a}{59}

Consider an elementary zone or annulus of the
surface (\Fig{59}), of breadth~$dr$, situated at a distance~$r$
from the centre. We may consider the entire surface
as consisting of such narrow zones, and the
whole area~$A$ will simply be the integral of all
\DPPageSep{229.png}{217}%
such elementary zones from centre to margin, that is,
integrated from $r = 0$ to $r = R$.

We have therefore to find an expression for the
elementary area~$dA$ of the narrow zone. Think of
it as a strip of breadth~$dr$, and of a length that is
the periphery of the circle of radius~$r$, that is, a
length of~$2 \pi r$. Then we have, as the area of the
narrow zone,
\[
dA = 2 \pi r\, dr.
\]

Hence the area of the whole circle will be:
\[
A = \int dA
  = \int^{r=R}_{r=0} 2 \pi r · dr
  = 2 \pi \int^{r=R}_{r=0} r · dr.
\]

Now, the general integral of $r · dr$ is~$\frac{1}{2} r^2$. Therefore,
\begin{DPalign*}
A &= 2 \pi \bigl[\tfrac{1}{2} r^2 \bigr]^{r=R}_{r=0}; \\
\lintertext{or}
A &= 2 \pi \bigl[\tfrac{1}{2} R^2 - \tfrac{1}{2}(0)^2\bigr]; \\
\lintertext{whence}
A &= \pi R^2.
\end{DPalign*}


\Subsection{Another Exercise.}
Let us find the mean ordinate of the positive part
of the curve $y = x - x^2$, which is shown in \Fig{60}.
\Figure[3in]{229a}{60}
To find the mean ordinate, we shall have to find the
area of the piece~$OMN$, and then divide it by the
\DPPageSep{230.png}{218}%
length of the base~$ON$. But before we can find
the area we must ascertain the length of the base,
so as to know up to what limit we are to integrate.
At $N$ the ordinate~$y$ has zero value; therefore, we
must look at the equation and see what value of~$x$
will make $y = 0$. Now, clearly, if $x$ is~$0$, $y$~will also be~$0$,
the curve passing through the origin~$O$; but also,
if $x=1$, $y=0$; so that $x=1$ gives us the position of
the point~$N$.

Then the area wanted is
\begin{align*}
  &= \int^{x=1}_{x=0} (x-x^2)\, dx \\
  &= \left[\tfrac{1}{2} x^2 - \tfrac{1}{3} x^3 \right]^{1}_{0} \\
  &= \left[\tfrac{1}{2} - \tfrac{1}{3} \right] - [0-0] \\
  &= \tfrac{1}{6}.
\end{align*}

But the base length is~$1$.

Therefore, the average ordinate of the curve $= \frac{1}{6}$.

[\NB---It will be a pretty and simple exercise in
maxima and minima to find by differentiation what
is the height of the maximum ordinate. It \emph{must} be
greater than the average.]

The mean ordinate of any curve, over a range from
$x= 0$ to $x = x_1$, is given by the expression,
\[
\text{mean~$y$} = \frac{1}{x_1} \int^{x=x_1}_{x=0} y · dx.
\]

One can also find in the same way the surface area
of a solid of revolution.
\DPPageSep{231.png}{219}%

%[** TN: Original contains a run-in heading]
\Example. The curve $y=x^2-5$ is revolving about
the axis of~$x$. Find the area of the surface generated
by the curve between $x=0$ and~$x=6$.

A point on the curve, the ordinate of which is~$y$,
describes a circumference of length~$2\pi y$, and a narrow
belt of the surface, of width~$dx$, corresponding to this
point, has for area~$2\pi y\, dx$. The total area is
\begin{align*}
2\pi \int^{x=6}_{x=0} y\, dx
  &= 2\pi \int^{x=6}_{x=0} (x^2-5)\, dx
   = 2\pi \left[\frac{x^3}{3} - 5x\right]^6_0 \\
  &= 6.28 × 42=263.76.
\end{align*}


\Section{Areas in Polar Coordinates.}

When the equation of the boundary of an area is
given as a function of the distance~$r$ of a point of it
from a fixed point~$O$ (see \Fig{61}) called the \emph{pole}, and
\Figure[2.5in]{231a}{61}
of the angle which $r$~makes with the positive horizontal
direction~$OX$, the process just explained can
be applied just as easily, with a small modification.
Instead of a strip of area, we consider a small triangle
$OAB$, the angle at~$O$ being~$d\theta$, and we find the sum
\DPPageSep{232.png}{220}%
of all the little triangles making up the required
area.

{\loosen%
The area of such a small triangle is approximately
$\dfrac{AB}{2}×r$ or $\dfrac{r\, d\theta}{2}×r$;} hence the portion of the area
included between the curve and two positions of~$r$
corresponding to the angles $\theta_1$~and~$\theta_2$ is given by
\[
\tfrac{1}{2} \int^{\theta=\theta_2}_{\theta=\theta_1} r^2\, d\theta.
\]

\tb


\Examples.
(1) Find the area of the sector of $1$~radian in a
circumference of radius $a$~\DPchg{inch}{inches}.

The polar equation of the circumference is evidently
$r=a$. The area is
\[
\tfrac{1}{2} \int^{\theta=\theta_2}_{\theta=\theta_1} a^2\, d\theta
  = \frac{a^2}{2} \int^{\theta=1}_{\theta=0} d\theta
  = \frac{a^2}{2}.
\]

(2) Find the area of the first quadrant of the curve
(known as ``Pascal's Snail''), the polar equation of
which is $r=a(1+\cos \theta)$.
\begin{align*}
\text{Area}
  &= \tfrac{1}{2}  \int^{\theta=\frac{\pi}{2}}_{\theta=0} a^2(1+\cos \theta)^2\, d\theta  \\
  &= \frac{a^2}{2} \int^{\theta=\frac{\pi}{2}}_{\theta=0} (1+2 \cos \theta + \cos^2 \theta)\, d\theta  \\
  &= \frac{a^2}{2} \left[\theta + 2 \sin \theta + \frac{\theta}{2} + \frac{\sin 2 \theta}{4} \right]^{\efrac{\pi}{2}}_{0} \\
  &= \frac{a^2(3\pi+8)}{8}.
\end{align*}
\DPPageSep{233.png}{221}%

\Section{Volumes by Integration.}

What we have done with the area of a little strip
of a surface, we can, of course, just as easily do with
the volume of a little strip of a solid. We can add
up all the little strips that make up the total solid,
and find its volume, just as we have added up all the
small little bits that made up an area to find the final
area of the figure operated upon.

\tb


\Examples.
(1) Find the volume of a sphere of radius~$r$.

A thin spherical shell has for volume~$4\pi x^2\, dx$ (see
\Fig{59}, \Pageref{fig:59}); summing up all the concentric shells %[xref, Page]
which make up the sphere, we have
\[
\text{volume sphere}
  = \int^{x=r}_{x=0} 4\pi x^2\, dx
  = 4\pi \left[\frac{x^3}{3} \right]^r_0
  = \tfrac{4}{3} \pi r^3.
\]

\Figure[2in]{233a}{62}

We can also proceed as follows: a slice of the
sphere, of thickness~$dx$, has for volume~$\pi y^2\, dx$ (see
\Fig{62}). Also $x$ and~$y$ are related by the expression
\[
y^2 = r^2 - x^2.
\]
\DPPageSep{234.png}{222}%
\begin{DPalign*}
\lintertext{\indent Hence}
\text{volume sphere}
  &= 2 \int^{x=r}_{x=0} \pi(r^2-x^2)\, dx \\
  &= 2 \pi \left[ \int^{x=r}_{x=0} r^2\, dx - \int^{x=r}_{x=0} x^2\, dx \right] \\
  &= 2 \pi \left[r^2x - \frac{x^3}{3} \right]^r_0 = \frac{4\pi}{3} r^3.
\end{DPalign*}

(2) Find the volume of the solid generated by the
revolution of the curve $y^2=6x$ about the axis of~$x$,
between $x=0$ and $x=4$.

The volume of a strip of the solid is~$\pi y^2\, dx$.
\begin{DPalign*}
\lintertext{\indent Hence}
\text{volume}
  &= \int^{x=4}_{x=0} \pi y^2\, dx = 6\pi \int^{x=4}_{x=0} x\, dx  \\
  &= 6\pi \left[ \frac{x^2}{2} \right]^4_0 = 48\pi = 150.8.
\end{DPalign*}

\Section{On Quadratic Means.}

In certain branches of physics, particularly in the
study of alternating electric currents, it is necessary
to be able to calculate the \emph{quadratic mean} of a
variable quantity. By ``quadratic mean'' is denoted
the square root of the mean of the squares of all the
values between the limits considered. Other names
for the quadratic mean of any quantity are its
``virtual'' value, or its ``\textsc{r.m.s.}''\ (meaning root-mean-square)
value. The French term is \textit{valeur efficace}. If~$y$
is the function under consideration, and the quadratic
mean is to be taken between the limits of $x=0$
and $x=l$; then the quadratic mean is expressed as
\[
\sqrt[2] {\frac{1}{l} \int^l_0 y^2\, dx}.
\]
\DPPageSep{235.png}{223}%


\Examples.
(1) To find the quadratic mean of the function
$y=ax$ (\Fig{63}).

Here the integral is $\ds\int^l_0 a^2 x^2\, dx$,\DPnote{** TN: Omitting line break}
which is~$\frac{1}{3} a^2 l^3$.

\Figure[2.5in]{235a}{63}

Dividing by~$l$ and taking the square root, we have
\[
\text{quadratic mean} = \frac{1}{\sqrt 3}\, al.
\]

Here the arithmetical mean is~$\frac{1}{2}al$; and the ratio
of quadratic to arithmetical mean (this ratio is called
the \emph{form-factor}) is $\dfrac{2}{\sqrt 3}=1.155$.

(2) To find the quadratic mean of the function $y=x^a$.

The integral is $\ds\int^{x=l}_{x=0} x^{2a}\, dx$, that is $\dfrac{l^{2a+1}}{2a+1}$.
\begin{DPgather*}
\lintertext{Hence}
\text{quadratic mean} = \sqrt[2]{\dfrac{l^{2a}}{2a+1}}.
\end{DPgather*}

(3) To find the quadratic mean of the function $y=a^{\efrac{x}{2}}$.

The integral is $\ds\int^{x=l}_{x=0} (a^{\efrac{x}{2}})^2\, dx$, that is $\ds\int^{x=l}_{x=0} a^x\, dx$,
\DPPageSep{236.png}{224}%
\begin{DPgather*}
\lintertext{or}
\left[ \frac{a^x}{\log_\epsilon a} \right]^{x=l}_{x=0},
\end{DPgather*}
which is $\dfrac{a^l-1}{\log_\epsilon a}$.

Hence the quadratic mean is $\sqrt[2] {\dfrac{a^l - 1}{l \log_\epsilon a}}$.


\Exercises{XVIII} (See \Pageref{AnsEx:XVIII} for Answers.)
\begin{Problems}
\Item{(1)} Find the area of the curve $y=x^2+x-5$ between
$x=0$ and $x=6$, and the mean ordinates between
these limits.

\Item{(2)} Find the area of the parabola $y=2a\sqrt x$ between
$x=0$ and $x=a$. Show that it is two-thirds of the
rectangle of the limiting ordinate and of its abscissa.

\Item{(3)} Find the area of the positive portion of a sine
curve and the mean ordinate.

\Item{(4)} Find the area of the positive portion of the
curve $y=\sin^2 x$, and find the mean ordinate.

\Item{(5)} Find the area included between the two branches
of the curve $y=x^2 ± x^{\efrac{5}{2}}$ from $x=0$ to $x=1$, also the
area of the positive portion of the lower branch of
the curve (see \Fig{30}, \Pageref{fig:30}). %[ ** Page, xref]

\Item{(6)} Find the volume of a cone of radius of base~$r$,
and of height~$h$.

\Item{(7)} Find the area of the curve $y=x^3-\log_\epsilon x$ between
$x=0$ and $x=1$.

\Item{(8)} Find the volume generated by the curve
$y=\sqrt{1+x^2}$, as it revolves about the axis of~$x$, between
$x=0$ and $x=4$.
\DPPageSep{237.png}{225}%

\Item{(9)} Find the volume generated by a sine curve
revolving about the axis of~$x$. Find also the area of
its surface.

\Item{(10)} Find the area of the portion of the curve
$xy=a$ included between $x=1$ and $x = a$. Find the
mean ordinate between these limits.

\Item{(11)} Show that the quadratic mean of the function
$y=\sin x$, between the limits of $0$~and~$\pi$ radians, is~$\dfrac{\sqrt2}{2}$.
Find also the arithmetical mean of the same
function between the same limits; and show that the
form-factor is~$=1.11$.

\Item{(12)} Find the arithmetical and quadratic means of
the function $x^2+3x+2$, from $x=0$ to $x=3$.

\Item{(13)} Find the quadratic mean and the arithmetical
mean of the function $y=A_1 \sin x + A_1 \sin 3x$.

\Item{(14)} A certain curve has the equation $y=3.42\epsilon^{0.21x}$.
Find the area included between the curve and the
axis of~$x$, from the ordinate at $x=2$ to the ordinate
at $x = 8$. Find also the height of the mean ordinate
of the curve between these points.

\Item{(15)} Show that the radius of a circle, the area of
which is twice the area of a polar diagram, is equal
to the quadratic mean of all the values of~$r$ for that
polar diagram.

\Item{(16)} Find the volume generated by the curve
$y=±\dfrac{x}{6}\sqrt{x(10-x)}$ rotating about the axis of~$x$.
\end{Problems}
\DPPageSep{238.png}{226}%

