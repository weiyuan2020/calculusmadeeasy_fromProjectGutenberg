
\Chapter[SINES AND COSINES]{XV}{How to deal with Sines and Cosines}

\First{Greek} letters being usual to denote angles, we will
take as the usual letter for any variable angle the
letter~$\theta$ (``theta'').

Let us consider the function
\[
y= \sin \theta.
\]

\Figure{177a}{43}

What we have to investigate is the value of $\dfrac{d(\sin \theta)}{d \theta}$;
or, in other words, if the angle~$\theta$ varies, we have to
find the relation between the increment of the sine
and the increment of the angle, both increments being
indefinitely small in themselves. Examine \Fig{43},   %[ **","?]
wherein, if the radius of the circle is unity, the height
of~$y$ is the sine, and $\theta$ is the angle. Now, if $\theta$ is
\DPPageSep{178.png}{166}%
supposed to increase by the addition to it of the
small angle $d \theta$---an element of angle---the height
of~$y$, the sine, will be increased by a small element~$dy$.
The new height~$y + dy$ will be the sine of the new
angle $\theta + d \theta$, or, stating it as an equation,
\[
y+dy = \sin(\theta + d \theta);
\]
and subtracting from this the first equation gives
\[
dy = \sin(\theta + d \theta)- \sin \theta.
\]

The quantity on the right-hand side is the difference
between two sines, and books on trigonometry tell
us how to work this out. For they tell us that if
$M$ and~$N$ are two different angles,
\[
\sin M - \sin N = 2 \cos\frac{M+N}{2}·\sin\frac{M-N}{2}.
\]

If, then, we put $M= \theta + d \theta$ for one angle, and
$N= \theta$ for the other, we may write
\begin{DPalign*}
dy &= 2 \cos\frac{\theta + d\theta + \theta}{2}
      · \sin\frac{\theta + d\theta - \theta}{2},\\
\lintertext{or,}
dy &= 2\cos(\theta + \tfrac{1}{2}d\theta)
     · \sin\tfrac{1}{2} d\theta.
\end{DPalign*}

But if we regard $d \theta$ as indefinitely small, then in
the limit we may neglect~$\frac{1}{2} d \theta$ by comparison with~$\theta$,
and may also take $\sin\frac{1}{2} d \theta$ as being the same as~$\frac{1}{2} d \theta$.
The equation then becomes:\Pagelabel{differsin}
\begin{DPalign*}
dy &= 2 \cos \theta × \tfrac{1}{2} d \theta; \\
dy &= \cos \theta · d \theta, \\
\lintertext{and, finally,}
\dfrac{dy}{d \theta} &= \cos \theta.
\end{DPalign*}
\DPPageSep{179.png}{167}%

The accompanying curves, \Figs{44}{and}{45}, show,
plotted to scale, the values of $y=\sin \theta$, and $\dfrac{dy}{d\theta}=\cos\theta$,
for the corresponding values of~$\theta$.
%[** TN: Graphs in the original are reversed; cos is labeled Fig. 44., etc.]
\Figure[4in]{179a}{44}\Pagelabel{erratum2}
\Figure[4in]{179b}{45}
\tb
\DPPageSep{180.png}{168}%

Take next the cosine.\Pagelabel{differcos}

Let $y=\cos \theta$.

Now $\cos \theta=\sin\left(\dfrac{\pi}{2}-\theta\right)$.

Therefore
\begin{align*}
&\begin{aligned}
dy = d\left(\sin\left(\frac{\pi}{2} - \theta\right)\right)
  &= \cos\left(\frac{\pi}{2} - \theta\right) × d(-\theta), \\
  &= \cos\left(\frac{\pi}{2} - \theta\right) × (-d\theta),
\end{aligned} \\
&\frac{dy}{d\theta} = -\cos\left(\frac{\pi}{2} - \theta\right).
\intertext{\indent And it follows that}
&\frac{dy}{d\theta} = -\sin \theta.
\end{align*}

\tb

Lastly, take the tangent.
%
\begin{DPalign*}
\lintertext{\indent Let}
y  &= \tan \theta, \\
dy &= \tan(\theta + d\theta) - \tan\theta. \\
\intertext{\indent Expanding, as shown in books on trigonometry,}
\tan(\theta + d\theta)
   &= \frac{\tan\theta + \tan d\theta}
           {1 - \tan\theta·\tan d\theta}; \\
\lintertext{whence}
dy &= \frac{\tan\theta + \tan d\theta}
           {1-\tan\theta·\tan d\theta} - \tan\theta \\
   &= \frac{(1 + \tan^2\theta)\tan d\theta}
           {1-\tan\theta·\tan d\theta}.
\end{DPalign*}
\DPPageSep{181.png}{169}%

Now remember that if $d\theta$ is indefinitely diminished,
the value of~$\tan d\theta$ becomes identical with~$d\theta$, and
$\tan\theta · d\theta$ is negligibly small compared with~$1$, so that
the expression reduces to
\begin{DPalign*}
dy &= \frac{(1+\tan^2 \theta)\, d\theta}{1}, \\
\lintertext{so that}
\frac{dy}{d\theta} &= 1 + \tan^2\theta, \\
\lintertext{or}
\frac{dy}{d\theta} &= \sec^2 \theta.
\end{DPalign*}

Collecting these results, we have:
\[
\begin{array}{|*{2}{>{\quad}c<{\quad}|}}
\hline
\DStrut y   & \dfrac{dy}{d\theta} \\
\hline
\Strut\sin\theta & \cos\theta \\
\cos\theta & -\sin\theta \\
\Strut\tan\theta & \sec^2 \theta\\
\hline
\end{array}
\]

Sometimes, in mechanical and physical questions,
as, for example, in simple harmonic motion and in
wave-motions, we have to deal with angles that increase
in proportion to the time. Thus, if $T$ be the
time of one complete \emph{period}, or movement round the
circle, then, since the angle all round the circle is $2\pi$~radians,
or~$360°$, the amount of angle moved through
in time~$t$, will be
\begin{DPalign*}
\theta &= 2\pi\frac{t}{T},\quad \text{in radians,} \\
\lintertext{or}
\theta &= 360\frac{t}{T},\quad \text{in degrees.}
\end{DPalign*}
\DPPageSep{182.png}{170}%

If the \emph{frequency}, or number of periods per second,
be denoted by~$n$, then $n = \dfrac{1}{T}$, and we may then write:
\[
\theta=2\pi nt.
\]
Then we shall have
\[
y = \sin 2\pi nt.
\]

If, now, we wish to know how the sine varies with
respect to time, we must differentiate with respect, not
to~$\theta$, but to~$t$. For this we must resort to the artifice
explained in Chapter~IX., \Pageref{chap:IX}, and put %[ ** Page]
\[
\frac{dy}{dt} = \frac{dy}{d\theta} · \frac{d\theta}{dt}.
\]

Now $\dfrac{d\theta}{dt}$ will obviously be~$2\pi n$; so that
\begin{align*}
\frac{dy}{dt} &= \cos \theta × 2\pi n \\
              &= 2\pi n · \cos 2\pi nt. \\
\intertext{\indent Similarly, it follows that}
\frac{d(\cos 2\pi nt)}{dt} &= -2\pi n · \sin 2\pi nt.
\end{align*}


\Section{Second Differential Coefficient of Sine or Cosine.}

We have seen that when $\sin \theta$ is differentiated with
respect to~$\theta$ it becomes $\cos \theta$; and that when $\cos \theta$ is
differentiated with respect to~$\theta$ it becomes $-\sin \theta$;
or, in symbols,
\[
\frac{d^2(\DPtypo{\cos \theta}{\sin \theta})}{d\theta^2} = -\sin \theta.
\]
\DPPageSep{183.png}{171}%

So we have this curious result that we have found
a function such that if we differentiate it twice over,
we get the same thing from which we started, but
with the sign changed from $+$~to~$-$.

The same thing is true for the cosine; for differentiating
$\cos\theta$ gives us $-\sin\theta$, and differentiating
$-\sin\theta$ gives us $-\cos\theta$; or thus:
\[
\frac{d^2(\cos\theta)}{d\theta^2} = -\cos\theta.
\]

\emph{Sines and cosines are the only functions of which
the second differential coefficient is equal \emph{(and of
opposite sign to)} the original function.}

\tb

\Examples.\Pagelabel{intex3}
With what we have so far learned we can now
differentiate expressions of a more complex nature.

(1) $y=\arcsin x$.

If $y$ is the arc whose sine is~$x$, then $x = \sin y$.
\[
\frac{dx}{dy}=\cos y.
\]

Passing now from the inverse function to the original
one, we get
\begin{DPalign*}
\frac{dy}{dx}
  &= \frac{1}{\;\dfrac{dx}{dy}\;} = \frac{1}{\cos y}. \\
\lintertext{\indent Now}
\cos y
  &= \sqrt{1-\sin^2 y}=\sqrt{1-x^2}; \\
\lintertext{hence}
\frac{dy}{dx}
  &= \frac{1}{\sqrt{1-x^2}},
\end{DPalign*}
a rather unexpected result\DPtypo{}{.}
\DPPageSep{184.png}{172}%

(2) $y=\cos^3 \theta$.

This is the same thing as $y=(\cos \theta)^3$.

Let $\cos\theta=v$;\quad then $y=v^3$;\quad $\dfrac{dy}{dv}=3v^2$.
\begin{align*}
\frac{dv}{d\theta} &= -\sin\theta.\\
\frac{dy}{d\theta} &=  \frac{dy}{dv} × \frac{dv}{d\theta}
                    = -3 \cos^2 \theta \sin\theta.
\end{align*}

(3) $y=\sin(x+a)$.

Let $x+a=v$;\quad then $y=\sin v$.
\[
\frac{dy}{dv}=\cos v;\qquad
\frac{dv}{dx}=1 \quad\text{and}\quad
\frac{dy}{dx}=\cos(x+a).
\]

(4) $y=\log_\epsilon \sin \theta$.

Let $\sin\theta=v$;\quad $y=\log_\epsilon v$.
\begin{align*}
\frac{dy}{dv}      &= \frac{1}{v};\quad \frac{dv}{d\theta}=\cos\theta;\\
\frac{dy}{d\theta} &= \frac{1}{\sin\theta} × \cos\theta = \cot\theta.
\end{align*}

(5) $y=\cot\theta=\dfrac{\cos\theta}{\sin\theta}$.
\begin{align*}
\frac{dy}{d\theta}
  &= \frac{-\sin^2\theta - \cos^2 \theta}{\sin^2 \theta}\\
  &= -(1+\cot^2 \theta) = -\cosec^2 \theta.
\end{align*}

(6) $y=\tan 3\theta$.

Let $3\theta=v$;\quad $y=\tan v$;\quad $\dfrac{dy}{dv}=\sec^2 v$.
\[
\frac{dv}{d\theta}=3;\quad
\frac{dy}{d\theta}=3 \sec^2 3\theta.
\]
\DPPageSep{185.png}{173}%

(7) $y = \sqrt{1+3\tan^2\theta}$;\quad $y=(1+3 \tan^2 \theta)^{\efrac{1}{2}}$.

Let $3\tan^2\theta=v$.
\begin{DPalign*}
y &= (1+v)^{\efrac{1}{2}};\quad
\frac{dy}{dv} = \frac{1}{2\sqrt{1+v}} \text{ (see \Pageref{ExNo1});} \\
%
\frac{dv}{d\theta}
  &= 6\tan\theta \sec^2 \theta \\
%
\intertext{(for, if $\tan \theta = u$,}
v &= 3u^2;\quad \frac{dv}{du} = 6u;\quad \frac{du}{d\theta} = \sec^2 \theta; \\
%
\lintertext{hence}
\frac{dv}{d\theta}
  &= 6 \DPtypo{}{(}\tan \theta \sec^2 \theta) \\
\lintertext{hence}
%
\frac{dy}{d\theta}
  &= \frac{6\tan\theta \sec^2\theta}{2\sqrt{1 + 3\tan^2\theta}}.
\end{DPalign*}

(8) $y=\sin x \cos x$. \Pagelabel{example1}
\begin{align*}
\frac{dy}{dx}
  &= \sin x(-\sin x) + \cos x × \cos x \\
  &= \cos^2 x - \sin^2 x.
\end{align*}


\Exercises{XIV} (See \Pageref[page]{AnsEx:XIV} for Answers.)
\begin{Problems}
\Item{(1)} Differentiate the following:
\begin{align*}
\text{(i)}\quad   y &= A \sin\left(\theta - \frac{\pi}{2}\right).\\
\text{(ii)}\quad  y &= \sin^2 \theta;\quad \text{and } y = \sin 2\theta.\\
\text{(iii)}\quad y &= \sin^3 \theta;\quad \text{and } y = \sin 3\theta.
\end{align*}

\Item{(2)} Find the value of~$\theta$ for which $\sin\theta × \cos\theta$ is a
maximum.

\Item{(3)} Differentiate $y=\dfrac{1}{2\pi} \cos 2\pi nt$.
\DPPageSep{186.png}{174}%

\Item{(4)} If $y = \sin a^x$, find~$\dfrac{dy}{dx}$.

\Item{(5)} Differentiate $y=\log_\epsilon \cos x$.

\Item{(6)} Differentiate $y=18.2 \sin(x+26°)$.

\Item{(7)} Plot the curve $y=100 \sin(\theta-15°)$; and show
that the slope of the curve at $\theta = 75°$ is half the
maximum slope.

\Item{(8)} If $y=\sin \theta·\sin 2\theta$, find~$\dfrac{dy}{d\theta}$.

\Item{(9)}  If $y=a·\tan^m(\theta^n)$, find the differential coefficient
of~$y$ with respect to~$\theta$.

\Item{(10)} Differentiate $y=\epsilon^x \sin^2 x$.

\Item{(11)} Differentiate the three equations of Exercises~XIII.
(\Pageref{XIII:4}), No.~4, and compare their differential
coefficients, as to whether they are equal, or nearly
equal, for very small values of~$x$, or for very large
values of~$x$, or for values of~$x$ in the neighbourhood
of $x=30$.

\Item{(12)} Differentiate the following:
\begin{align*}%[** TN: Reformatted in two columns]
\text{(i)}\quad   y &= \sec x.    &
\text{(ii)}\quad  y &= \arccos x. \\
\text{(iii)}\quad y &= \arctan x. &
\text{(iv)}\quad  y &= \arcsec x. \\
\text{(v)}\quad   y &= \tan x × \sqrt{3 \sec x}. &&
\end{align*}

\Item{(13)} Differentiate $y=\sin(2\theta +3)^{2.3}$.

\Item{(14)} Differentiate $y=\theta^3+3 \sin(\theta+3)-3^{\sin \theta} - 3^\theta$.

\Item{(15)} Find the maximum or minimum of $y=\theta \cos \theta$.
\end{Problems}
\DPPageSep{187.png}{175}%

