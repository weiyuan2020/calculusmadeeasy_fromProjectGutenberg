
\Chapter{XX}{Dodges, Pitfalls, and Triumphs}

\Paragraph{Dodges.} A great part of the labour of integrating
things consists in licking them into some shape that
can be integrated. The books---and by this is meant
the serious books---on the Integral Calculus are full
of plans and methods and dodges and artifices for
this kind of work. The following are a few of
them.

\Paragraph{Integration by Parts.}\Pagelabel{intparts} This name is given to a
dodge, the formula for which is
\[
\int u\, dx = ux - \int x\, du + C.
\]
It is useful in some cases that you can't tackle
directly, for it shows that if in any case $\ds\int x\, du$ can
be found, then $\ds\int u\, dx$ can also be found. The formula
can be deduced as follows. From \Pageref{differprod}, we have,
\[
d(ux) = u\, dx + x\, du,
\]
which may be written
\[
u(dx) = d(ux) - x\, du,
\]
which by direct integration gives the above expression.
\DPPageSep{239.png}{227}%


\Examples.
(1) Find $\ds\int w · \sin w\, dw$.

Write $u = w$, and for $\sin w · dw$ write~$dx$. We shall
then have $du = dw$, while $\ds\int \sin w · dw = -\cos w = x$.

Putting these into the formula, we get
\begin{align*}
\int w · \sin w\, dw &= w(-\cos w) - \int -\cos w\, dw  \\
                     &=-w \cos w + \sin w + C.
\end{align*}

(2) Find $\ds\int x \epsilon^x\, dx$.
%
\BindMath{\begin{DPalign*}
\lintertext{\indent Write}
 u &=  x, & \epsilon^x\, dx&=dv; \\
\lintertext{then}
du &= dx, & v &=\epsilon^x,
\end{DPalign*}
\begin{DPgather*}
\lintertext{and}
\int x\epsilon^x\, dx
   = x\epsilon^x - \int \epsilon^x\, dx
      \quad \text{(by the formula)} \\
   = x \epsilon^x - \epsilon^x = \epsilon^x(x-1) + C.
\end{DPgather*}}%

(3) Try $\ds\int \cos^2 \theta\, d\theta$.\Pagelabel{moreexamples}
\begin{DPalign*}
u &= \cos \theta, &\cos \theta\, d\theta &= dv. \\
\lintertext{\indent Hence}
du&= -\sin \theta\, d\theta, & v &=\sin \theta,
\end{DPalign*}
\begin{align*}
\int \cos^2 \theta\, d\theta
  &= \cos \theta \sin \theta+ \int \sin^2 \theta\, d\theta       \\
  &= \frac{2 \cos\theta \sin\theta}{2} +\int(1-\cos^2 \theta)\, d\theta  \\
  &= \frac{\sin 2\theta}{2} + \int d\theta - \int \cos^2 \theta\, d\theta.
\end{align*}
\begin{DPalign*}
\lintertext{\indent Hence}
2 \int \cos^2 \theta\, d\theta
  &= \frac{\sin 2\theta}{2} + \theta \\
\lintertext{and}
\int \cos^2 \theta\, d\theta
  &= \frac{\sin 2\theta}{4} + \frac{\theta}{2} + C.
\end{DPalign*}
\DPPageSep{240.png}{228}%

(4) Find $\ds\int x^2 \sin x\, dx$.
%
\begin{DPalign*}
\lintertext{\indent Write}
x^2  &= u, & \sin x\, dx &= dv; \\
\lintertext{then}
du &= 2x\, dx, & v &= -\cos x,
\end{DPalign*}
\[
\int x^2 \sin x\, dx = -x^2 \cos x + 2 \int x \cos x\, dx.
\]

Now find $\ds\int x \cos x\, dx$, integrating by parts (as in
Example~1 above):
\[
\int x \cos x\, dx = x \sin x + \cos x+C.
\]

Hence
\begin{align*}
\int x^2 \sin x\, dx
  &= -x^2 \cos x + 2x \sin x + 2 \cos x + C' \\
  &= 2 \left[ x \sin x + \cos x \left(1 - \frac{x^2}{2}\right) \right] +C'.
\end{align*}

(5) Find $\ds\int \sqrt{1-x^2}\, dx$.
\begin{DPalign*}
\lintertext{Write}
u &= \sqrt{1-x^2},\quad dx=dv;  \\
\lintertext{then}
du &= -\frac{x\, dx}{\sqrt{1-x^2}}\quad \text{(see Chap.~IX., \Pageref{chap:IX})}
\end{DPalign*}
and $x=v$; so that
\[
\int \sqrt{1-x^2}\, dx=x \sqrt{1-x^2} + \int \frac{x^2\, dx}{\sqrt{1-x^2}}.
\]

Here we may use a little dodge, for we can write
\[
\int \sqrt{1-x^2}\, dx
  = \int \frac{(1-x^2)\, dx}{\sqrt{1-x^2}}
  = \int \frac{dx}{\sqrt{1-x^2}} - \int \frac{x^2\, dx}{\sqrt{1-x^2}}.
\]

Adding these two last equations, we get rid of
$\ds\int \dfrac{x^2\, dx}{\sqrt{1-x^2}}$, and we have
\[
2 \int \sqrt{1-x^2}\, dx = x\sqrt{1-x^2} + \int \frac{dx}{\sqrt{1-x^2}}.
\]
\DPPageSep{241.png}{229}%

Do you remember meeting $\dfrac {dx}{\sqrt{1-x^2}}$? it is got by
differentiating $y=\arcsin x$ (see \Pageref{intex3}); hence its integral
is $\arcsin x$, and so
\[
\int \sqrt{1-x^2}\, dx = \frac{x \sqrt{1-x^2}}{2} + \tfrac{1}{2} \arcsin x +C.
\]

You can try now some exercises by yourself; you
will find some at the end of this chapter.

\Paragraph{Substitution.} This is the same dodge as explained
in Chap.~IX., \Pageref{chap:IX}. Let us illustrate its application
to integration by a few examples.

(1) $\ds\int \sqrt{3+x}\, dx$.
\begin{DPalign*}
\lintertext{\indent Let}
3+x &= u,\quad dx = du; \\
\lintertext{replace}
\int u^{\efrac{1}{2}}\, du
  &= \tfrac{2}{3} u^{\efrac{3}{2}} = \tfrac{2}{3}(3+x)^{\efrac{3}{2}}.
\end{DPalign*}

(2) $\ds\int \dfrac{dx}{\epsilon^x+\epsilon^{-x}}$.
\begin{DPgather*}
\lintertext{\indent Let}
\epsilon^x = u,\quad \frac{du}{dx} = \epsilon^x,\quad\text{and}\quad
dx = \frac{du}{\epsilon^x}; \\
\lintertext{so that}
\int \frac{dx}{\epsilon^x+\epsilon^{-x}}
  = \int \frac{du}{\epsilon^x(\epsilon^x+\epsilon^{-x})}
  = \int \frac{du}{u\left(u + \dfrac{1}{u}\right)}
  = \int \frac{du}{u^2+1}.
\end{DPgather*}

$\dfrac{du}{1+u^2}$ is the result of differentiating $\arctan x$.

Hence the integral is $\arctan \epsilon^x$.

(3) $\ds\int \dfrac{dx}{x^2+2x+3} = \ds\int \dfrac{dx}{x^2+2x+1+2} = \ds\int \dfrac{dx}{(x+1)^2+(\sqrt 2)^2}$.
\DPPageSep{242.png}{230}%
\begin{DPgather*}
\lintertext{\indent Let}
x+1=u,\quad dx=du;
\end{DPgather*}
then the integral becomes $\ds\int \dfrac{du}{u^2+(\sqrt2)^2}$; but $\dfrac{du}{u^2+a^2}$ is
the result of differentiating $u=\dfrac{1}{a} \arctan \dfrac{u}{a}$.

Hence one has finally $\dfrac{1}{\sqrt2} \arctan \dfrac{x+1}{\sqrt 2}$ for the value
of the given integral.

\emph{Formulæ of Reduction} are special forms applicable
chiefly to binomial and trigonometrical expressions
that have to be integrated, and have to be reduced
into some form of which the integral is known.

\textit{Rationalization}, and \textit{Factorization of Denominator}
are dodges applicable in special cases, but they do not
admit of any short or general explanation. Much
practice is needed to become familiar with these preparatory
processes.

The following example shows how the process of
splitting\Pagelabel{partfracs} into partial fractions, which we learned in
Chap.~XIII., \Pageref{partfracs2}, can be made use of in integration.

Take again $\ds\int \dfrac{dx}{x^2+2x+3}$; if we split $\dfrac{1}{x^2+2x+3}$
into partial fractions, this becomes (see \Pageref{partfracs3}):
\[
\dfrac{1}{2\sqrt{-2}} \left[\int \dfrac{dx}{x+1-\sqrt{-2}} - \int \dfrac{dx}{x+1+\sqrt{-2}} \right]
\]
\[
=\dfrac{1}{2\sqrt{-2}} \log_\epsilon \dfrac{x+1-\sqrt{-2}}{x+1+\sqrt{-2}}.
\]
Notice that the same integral can be expressed
\DPPageSep{243.png}{231}%
sometimes in more than one way (which are equivalent
to one another).

\Paragraph{Pitfalls.} A beginner is liable to overlook certain
points that a practised hand would avoid; such as
the use of factors that are equivalent to either zero or
infinity, and the occurrence of indeterminate quantities
such as $\tfrac{0}{0}$. There is no golden rule that will meet
every possible case. Nothing but practice and intelligent
care will avail. An example of a pitfall which
had to be circumvented arose in Chap.~XVIII., \Pageref{chap:XVIII},
when we came to the problem of integrating $x^{-1}\, dx$.

\Paragraph{Triumphs.} By triumphs must be understood the
successes with which the calculus has been applied to
the solution of problems otherwise intractable. Often
in the consideration of physical relations one is able
to build up an expression for the law governing the
interaction of the parts or of the forces that govern
them, such expression being naturally in the form of
a \emph{differential equation}, that is an equation containing
differential coefficients with or without other algebraic
quantities. And when such a differential equation
has been found, one can get no further until it has
been integrated. Generally it is much easier to state
the appropriate differential equation than to solve it:---the
real trouble begins then only when one wants to
integrate, unless indeed the equation is seen to possess
some standard form of which the integral is known,
and then the triumph is easy. The equation which
results from integrating a differential equation is
\DPPageSep{244.png}{232}%
called\footnote
{This means that the actual result of solving it is called its
``solution.'' But many mathematicians would say, with Professor
Forsyth, ``every differential equation \emph{is considered as solved} when
the value of the dependent variable is expressed as a function of
the independent variable by means either of known functions, or of
integrals, whether the integrations in the latter can or cannot be
expressed in terms of functions already known.''}
 its ``solution''; and it is quite astonishing
how in many cases the solution looks as if it had no
relation to the differential equation of which it is
the integrated form. The solution often seems as
different from the original expression as a butterfly
does from the caterpillar that it was. Who would
have supposed that such an innocent thing as
\[
\dfrac{dy}{dx} = \dfrac{1}{a^2-x^2}
\]
could blossom out into
\[
y = \dfrac{1}{2a} \log_\epsilon \dfrac{a+x}{a-x} + C?
\]
yet the latter is the \textit{solution} of the former.

As a last example, let us work out the above together.

By partial fractions,\Pagelabel{partfracs3}
\begin{align*}
\frac{1}{a^2-x^2} &= \frac{1}{2a(a+x)} + \frac{1}{2a(a-x)},  \\
dy &= \frac {dx}{2a(a+x)}+ \frac{dx}{2a(a-x)},  \\
y  &= \frac{1}{2a}
       \left( \int \frac{dx}{a+x}
            + \int \frac{dx}{a-x} \right) \displaybreak[1] \\
   &= \frac{1}{2a} \left(\log_\epsilon (a+x) - \log_\epsilon (a-x) \right) \displaybreak[1] \\
   &= \frac{1}{2a} \log_\epsilon \frac{a+x}{a-x} + C.
\end{align*}
\DPPageSep{245.png}{233}%
% [** TN: Hack removes vertical space, giving a better page break below]
\indent Not a very difficult metamorphosis!

There are whole treatises, such as Boole's \textit{Differential
Equations}, devoted to the subject of thus finding
the ``solutions'' for different original forms.


\Exercises{XIX} (See \Pageref{AnsEx:XIX} for Answers.)

\begin{Problems}[2]
\Item{(1)} Find $\ds\int \sqrt {a^2 - x^2}\, dx$.
\Item{(2)} Find $\ds\int x \log_\epsilon x\, dx$.

\ResetCols{2}
\Item{(3)} Find $\ds\int x^a \log_\epsilon x\, dx$.
\Item{(4)} Find $\ds\int \epsilon^x \cos \epsilon^x\, dx$.

\ResetCols{2}
\Item{(5)} Find $\ds\int \dfrac{1}{x} \cos (\log_\epsilon x)\, dx$.
\Item{(6)} Find $\ds\int x^2 \epsilon^x\, dx$.

\ResetCols{2}
\Item{(7)} Find $\ds\int \dfrac{(\log_\epsilon x)^a}{x}\, dx$.
\Item{(8)} Find $\ds\int \dfrac{dx}{x \log_\epsilon x}$.

\ResetCols{2}
\Item{(9)} Find $\ds\int \dfrac{5x+1}{x^2 +x-2}\, dx$.
\Item{(10)} Find $\ds\int \dfrac{(x^2 -3)\, dx}{x^3 - 7x+6}$.

\ResetCols{2}
\Item{(11)} Find $\ds\int \dfrac{b\, dx}{x^2 -a^2}$.
\Item{(12)} Find $\ds\int \dfrac{4x\, dx}{x^4 -1}$.

\ResetCols{2}
\Item{(13)} Find $\ds\int \dfrac{dx}{1-x^4}$.
\Item{(14)} Find $\ds\int \dfrac{dx}{x \sqrt {a-bx^2}}$.
\end{Problems}
\DPPageSep{246.png}{234}%
