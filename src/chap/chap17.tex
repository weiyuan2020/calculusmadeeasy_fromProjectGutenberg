
\Chapter{XVII}{Integration}

\First{The} great secret has already been revealed that this
mysterious symbol~$\ds\int$, which is after all only a long~$S$,
merely means ``the sum of,'' or ``the sum of all such
quantities as.'' It therefore resembles that other
symbol~$\sum$ (the Greek \emph{Sigma}), which is also a sign
of summation. There is this difference, however, in
the practice of mathematical men as to the use of
these signs, that while $\sum$ is generally used to indicate
the sum of a number of finite quantities, the integral
sign~$\ds\int$ is generally used to indicate the summing up
of a vast number of small quantities of indefinitely
minute magnitude, mere elements in fact, that go
to make up the total required. Thus $\ds\int dy = y$, and
$\ds\int dx = x$.

Any one can understand how the whole of anything
can be conceived of as made up of a lot of little bits;
and the smaller the bits the more of them there will
be. Thus, a line one inch long may be conceived as
made up of $10$~pieces, each $\frac{1}{10}$~of an inch long; or
of $100$~parts, each part being $\frac{1}{100}$~of an inch long;
\DPPageSep{195.png}{183}%
or of $1,000,000$ parts, each of which is $\frac{1}{1,000,000}$~of an
inch long; or, pushing the thought to the limits of
conceivability, it may be regarded as made up of
an infinite number of elements each of which is
infinitesimally small.

Yes, you will say, but what is the use of thinking
of anything that way? Why not think of it straight
off, as a whole? The simple reason is that there are
a vast number of cases in which one cannot calculate
the bigness of the thing as a whole without reckoning
up the sum of a lot of small parts. The process of
``\emph{integrating}'' is to enable us to calculate totals that
otherwise we should be unable to estimate directly.

Let us first take one or two simple cases to
familiarize ourselves with this notion of summing
up a lot of separate parts.

Consider the series:
\[
1 + \tfrac{1}{2} + \tfrac{1}{4} + \tfrac{1}{8}
  + \tfrac{1}{16} + \tfrac{1}{32} + \tfrac{1}{64} + \text{etc.}
\]

Here each member of the series is formed by taking
it half the value of the preceding. What is the value
of the total if we could go on to an infinite number
of terms? Every schoolboy knows that the answer
is~$2$. Think of it, if you like, as a line. Begin with
\Figure[3.5in]{195a}{46}
one inch; add a half inch, add a quarter; add an
eighth; and so on. If at any point of the operation
\DPPageSep{196.png}{184}%
we stop, there will still be a piece wanting to make
up the whole $2$~inches; and the piece wanting will
always be the same size as the last piece added.
Thus, if after having put together $1$,~$\frac{1}{2}$, and~$\frac{1}{4}$, we stop,
there will be $\frac{1}{4}$~wanting. If we go on till we have
added~$\frac{1}{64}$, there will still be $\frac{1}{64}$~wanting. The
remainder needed will always be equal to the last
term added. By an infinite number of operations
only should we reach the actual $2$~inches. Practically
we should reach it when we got to pieces so small
that they could not be drawn---that would be after
about $10$~terms, for the eleventh term is~$\frac{1}{1024}$. If we
want to go so far that not even a Whitworth's
measuring machine would detect it, we should merely
have to go to about $20$~terms. A microscope would
not show even the $18^{\text{th}}$~term! So the infinite number
of operations is no such dreadful thing after all.
The \emph{integral} is simply the whole lot. But, as we
shall see, there are cases in which the integral
calculus enables us to get at the \emph{exact} total that
there would be as the result of an infinite number
of operations. In such cases the integral calculus
gives us a \emph{rapid} and easy way of getting at a result
that would otherwise require an interminable lot of
elaborate working out. So we had best lose no time
in learning \emph{how to integrate}.
\DPPageSep{197.png}{185}%

\Section{Slopes of Curves, and the Curves themselves.}

Let us make a little preliminary enquiry about the
slopes of curves. For we have seen that differentiating
a curve means finding an expression for its slope (or
for its slopes at different points). Can we perform
the reverse process of reconstructing the whole curve
if the slope (or slopes) are prescribed for us?

Go back to case~(2) on \Pageref{Case2}. Here we have the %[ ** Page]
simplest of curves, a sloping line with the equation
\[
y = ax+b.
\]
\Figure[2.5in]{197a}{47}

We know that here $b$~represents the initial height
of~$y$ when $x= 0$, and that~$a$, which is the same as~$\dfrac{dy}{dx}$,
is the ``slope'' of the line. The line has a constant
slope. All along it the elementary triangles \raisebox{-12pt}{\Graphic{197b}}
have the same proportion between height and base.
Suppose we were to take the~$dx$'s, and~$dy$'s of finite
\DPPageSep{198.png}{186}%
magnitude, so that $10$~$dx$'s made up one inch, then
there would be ten little triangles like
\begin{center}
  \Graphic{198a}
\end{center}

Now, suppose that we were ordered to reconstruct
the ``curve,'' starting merely from the information
that $\dfrac{dy}{dx} = a$. What could we do? Still taking the
little~$d$'s as of finite size, we could draw $10$~of them,
all with the same slope, and then put them together,
end to end, like this:
\Figure[3.25in]{198b}{48}
And, as the slope is the same for all, they would join
to make, as in \Fig{48}, a sloping line sloping with the
correct slope $\dfrac{dy}{dx} = a$. And whether we take the $dy$'s
and~$dx$'s as finite or infinitely small, as they are all
\DPPageSep{199.png}{187}%
alike, clearly $\dfrac{y}{x} = a$, if we reckon $y$~as the total of
all the~$dy$'s, and $x$~as the total of all the~$dx$'s. But
whereabouts are we to put this sloping line? Are
we to start at the origin~$O$, or higher up? As the
only information we have is as to the slope, we are
without any instructions as to the particular height
above~$O$; in fact the initial height is undetermined.
The slope will be the same, whatever the initial height.
Let us therefore make a shot at what may be wanted,
and start the sloping line at a height~$C$ above~$O$.
That is, we have the equation
\[
y = ax + C.
\]

It becomes evident now\Pagelabel{constant} that in this case the added
constant means the particular value that $y$~has when
$x = 0$.

Now let us take a harder case, that of a line, the
slope of which is not constant, but turns up more and
more. Let us assume that the upward slope gets
greater and greater in proportion as $x$~grows. In
symbols this is:
\[
\frac{dy}{dx} = ax.
\]
Or, to give a concrete case, take $a = \frac{1}{5}$, so that
\[
\frac{dy}{dx} = \tfrac{1}{5} x.
\]

Then we had best begin by calculating a few of
the values of the slope at different values of~$x$, and
also draw little diagrams of them.
\DPPageSep{200.png}{188}%
\begin{DPalign*}
\lintertext{\indent When}
x &=0,\quad \frac{dy}{dx} = 0,\Z && \Graphic{200a1} \\
x &=1,\quad \frac{dy}{dx} = 0.2, && \Graphic{200a2} \\
x &=2,\quad \frac{dy}{dx} = 0.4, && \Graphic{200a3} \\
x &=3,\quad \frac{dy}{dx} = 0.6, && \Graphic{200a4} \\
x &=4,\quad \frac{dy}{dx} = 0.8, && \Graphic{200a5} \\
x &=5,\quad \frac{dy}{dx} = 1.0. && \Graphic{200a6}
\end{DPalign*}

Now try to put the pieces together, setting each so
that the middle of its base is the proper distance to
the right, and so that they fit together at the corners;
thus (\Fig{49}). The result is, of course, not a smooth
\Figure[3in]{200b}{49}
curve: but it is an approximation to one. If we had
taken bits half as long, and twice as numerous, like
\Fig{50}, we should have a better approximation. But
\DPPageSep{201.png}{189}%
for a perfect curve we ought to take each~$dx$ and its
corresponding~$dy$ infinitesimally small, and infinitely
numerous.

\Figure[3in]{201a}{50}
Then, how much ought the value of any~$y$ to be?
Clearly, at any point~$P$ of the curve, the value of~$y$
will be the sum of all the little~$dy$'s from~$0$ up to
that level, that is to say, $\ds\int dy = y$. And as each~$dy$ is
equal to $\frac{1}{5}x · dx$, it follows that the whole~$y$ will be
equal to the sum of all such bits as~$\frac{1}{5}x · dx$, or, as we
should write it, $\ds\int \tfrac{1}{5}x · dx$.

Now if $x$ had been constant, $\ds\int \tfrac{1}{5}x · dx$ would have
been the same as $\frac{1}{5} x \ds\int dx$, or~$\frac{1}{5}x^2$. But $x$~began by
being~$0$, and increases to the particular value of~$x$ at
the point~$P$, so that its average value from~$0$ to that
point is~$\frac{1}{2}x$. Hence $\ds\int \tfrac{1}{5} x\, dx = \tfrac{1}{10} x^2$; or $y=\frac{1}{10}x^2$.

But, as in the previous case, this requires the addition
of an undetermined constant~$C$, because we have not
\DPPageSep{202.png}{190}%
been told at what height above the origin the curve
will begin, when $x = 0$. So we write, as the equation
of the curve drawn in \Fig{51},
\[
y = \tfrac{1}{10}x^2 + C.
\]
\Figure{202a}{51}


\Exercises{XVI} (See \Pageref[page]{AnsEx:XVI} for Answers.)
\begin{Problems}
\Item{(1)} Find the ultimate sum of $\frac{2}{3} + \frac{1}{3} + \frac{1}{6} + \frac{1}{12} + \frac{1}{24} + \text{etc}$.

\Item{(2)} Show that the series $1 - \frac{1}{2} + \frac{1}{3} - \frac{1}{4} + \frac{1}{5} - \frac{1}{6} + \frac{1}{7}$\DPnote{[** TN: [sic], no +]}~etc.,
is convergent, and find its sum to $8$~terms.

\Item{(3)} If $\log_\epsilon(1+x) = x - \dfrac{x^2}{2} + \dfrac{x^3}{3} - \dfrac{x^4}{4} + \text{etc}$., find $\log_\epsilon 1.3$.

\Item{(4)} Following a reasoning similar to that explained
in this chapter, find~$y$,
\[
\text{(\textit{a}) if $\frac{dy}{dx} = \tfrac{1}{4} x$;\quad
(\textit{b}) if $\frac{dy}{dx} = \cos x$.}
\]

\Item{(5)} If $\dfrac{dy}{dx} = 2x + 3$, find~$y$.
\end{Problems}
\DPPageSep{203.png}{191}%
