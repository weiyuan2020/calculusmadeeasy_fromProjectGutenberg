
\Chapter{XIII}{Other Useful Dodges}

\Section{Partial Fractions.}

\First{We} have seen that when we differentiate a fraction
we have to perform a rather complicated operation;
and, if the fraction is not itself a simple one, the result
is bound to be a complicated expression. If we could
split the fraction into two or more simpler fractions
such that their sum is equivalent to the original
fraction, we could then proceed by differentiating
each of these simpler expressions. And the result of
differentiating would be the sum of two (or more)
differentials, each one of which is relatively simple;
while the final expression, though of course it will be
the same as that which could be obtained without
resorting to this dodge, is thus obtained with much
less effort and appears in a simplified form.

Let us see how to reach this result. Try first the
job of adding two fractions together to form a resultant
fraction. Take, for example, the two fractions $\dfrac{1}{x+1}$
and~$\dfrac{2}{x-1}$. Every schoolboy can add these together
and find their sum to be~$\dfrac{3x+1}{x^2-1}$. And in the same
\DPPageSep{134.png}{122}%
way he can add together three or more fractions.
Now this process can certainly be reversed:\Pagelabel{partfracs2} that is to
say, that if this last expression were given, it is certain
that it can somehow be split back again into its
original components or partial fractions. Only we do
not know in every case that may be presented to us
\emph{how} we can so split it. In order to find this out
we shall consider a simple case at first. But it is
important to bear in mind that all which follows
applies only to what are called ``proper'' algebraic
fractions, meaning fractions like the above, which have
the numerator of \emph{a lesser degree} than the denominator;
that is, those in which the highest index of~$x$ is less
in the numerator than in the denominator. If we
have to deal with such an expression as~$\dfrac{x^2+2}{x^2-1}$, we can
simplify it by division, since it is equivalent to
$1+\dfrac{3}{x^2-1}$; and $\dfrac{3}{x^2-1}$ is a proper algebraic fraction
to which the operation of splitting into partial fractions
can be applied, as explained hereafter.

% [** TN: Retaining inconsistent formatting; headings not on their own line]
\Paragraph{Case~I\@.} If we perform many additions of two or
more fractions the denominators of which contain only
terms in~$x$, and no terms in $x^2$,~$x^3$, or any other powers
of~$x$, we \emph{always} find that \emph{the denominator of the final
resulting fraction is the product of the denominators}
of the fractions which were added to form the result.
It follows that by factorizing the denominator of this
final fraction, we can find every one of the denominators
of the partial fractions of which we are in search.
\DPPageSep{135.png}{123}%

Suppose we wish to go back from $\dfrac{3x+1}{x^2-1}$ to the
components which we know are $\dfrac{1}{x+1}$ and~$\dfrac{2}{x-1}$. If
we did not know what those components were we can
still prepare the way by writing:
\[
\frac{3x+1}{x^2-1} = \frac{3x+1}{(x+1)(x-1)} = \frac{}{x+1} + \frac{}{x-1},
\]
leaving blank the places for the numerators until we
know what to put there. We always may assume the
sign between the partial fractions to be \emph{plus}, since, if
it be \emph{minus}, we shall simply find the corresponding
numerator to be negative. Now, since the partial
fractions are \emph{proper} fractions, the numerators are
mere numbers without~$x$ at all, and we can call them
$A$,~$B$, $C\dots$ as we please. So, in this case, we have:
\[
\frac{3x+1}{x^2-1} = \frac{A}{x+1} + \frac{B}{x-1}.
\]

If now we perform the addition of these two
partial fractions, we get $\dfrac{A(x-1)+B(x+1)}{(x+1)(x-1)}$; and this
must be equal to $\dfrac{3x+1}{(x+1)(x-1)}$. And, as the denominators
in these two expressions are the same,
the numerators must be equal, giving us:
\[
3x + 1 = A(x-1) + B(x + 1).
\]

Now, this is an equation with two unknown
quantities, and it would seem that we need another
equation before we can solve them and find $A$~and~$B$.
\DPPageSep{136.png}{124}%
But there is another way out of this difficulty. The
equation must be true for all values of~$x$; therefore
it must be true for such values of~$x$ as will cause
$x-1$ and~$x+1$ to become zero, that is for $x=1$ and
for $x=-1$ respectively. If we make $x=1$, we get
$4 = (A × 0)+(B × 2)$, so that $B=2$; and if we make
$x=-1$, we get $-2 = (A × -2) + (B × 0)$, so that $A=1$.
Replacing the $A$ and~$B$ of the partial fractions by
these new values, we find them to become $\dfrac{1}{x+1}$ and
$\dfrac{2}{x-1}$; and the thing is done.

As a farther example, let us take the fraction
$\dfrac{4x^2 + 2x - 14}{x^3 + 3x^2 - x - 3}$. The denominator becomes zero when
$x$ is given the value~$1$; hence $x-1$ is a factor of it,
and obviously then the other factor will be $x^2 + 4x + 3$;
and this can again be decomposed into $(x+1)(x+3)$.
So we may write the fraction thus:
\[
\frac{4x^2 + 2x - 14}{x^3 + 3x^2 - x - 3}
  = \frac{A}{x+1} + \frac{B}{x-1} + \frac{C}{x+3},
\]
making three partial factors.

Proceeding as before, we find
\[%[** TN: Set on two lines in the original]
4x^2 + 2x - 14 = A(x-1)(x+3) + B(x+1)(x+3) + C(x+1)(x-1).
\]

Now, if we make $x=1$, we get:
\[
-8 = (A × 0) + B(2 × 4) + (C × 0);\quad \text{that is, } B = -1.
\]

If $x= -1$, we get:
\[
-12 = A(-2 × 2) + (B × 0) + (C × 0);\quad \text{whence } A = 3.
\]
\DPPageSep{137.png}{125}%

If $x = -3$, we get:
\[
16 = (A × 0) + (B × 0) + C(-2 × -4);\quad \text{whence } C = 2.
\]

So then the partial fractions are:
\[
\frac{3}{x+1} - \frac{1}{x-1} + \frac{2}{x+3},
\]
which is far easier to differentiate with respect to~$x$
than the complicated expression from which it is
derived.

\Paragraph{Case~II\@.} If some of the factors of the denominator
contain terms in~$x^2$, and are not conveniently put
into factors, then the corresponding numerator may
contain a term in~$x$, as well as a simple number; and
hence it becomes necessary to represent this unknown
numerator not by the symbol~$A$ but by $Ax + B$; the
rest of the calculation being made as before.
%
\begin{DPgather*}
\lintertext{\rlap{\indent Try, for instance:}}
\frac{-x^2 - 3}{(x^2+1)(x+1)}. \\
\frac{-x^2 - 3}{(x^2+1)(x+1)} = \frac{Ax+B}{x^2+1} + \frac{C}{x+1};\\
-x^2 - 3 = (Ax + B)(x+1) + C(x^2+1).
\end{DPgather*}

Putting $x= -1$, we get $-4 = C × 2$; and $C = -2$;
\begin{DPalign*}
\lintertext{hence}
-x^2 - 3 &= (Ax + B)(x + 1) - 2x^2 - 2; \\
\lintertext{and}
x^2 - 1 &= Ax(x+1) + B(x+1).
\end{DPalign*}

Putting $x = 0$, we get $-1 = B$; \\
hence
\begin{DPgather*}
x^2 - 1 = Ax(x + 1) - x - 1;\quad \text{or } x^2 + x = Ax(x+1); \\
\lintertext{and}
x+1 = A(x+1),
\end{DPgather*}
\DPPageSep{138.png}{126}%
so that $A=1$, and the partial fractions are:
\[
\frac{x-1}{x^2+1} - \frac{2}{x+1}.
\]

Take as another example the fraction
\[
\frac{x^3-2}{(x^2+1)(x^2+2)}.
\]

We get
\begin{align*}
\frac{x^3-2}{(x^2+1)(x^2+2)}
  &= \frac{Ax+B}{x^2+1} + \frac{Cx+D}{x^2+2}\\
  &= \frac{(Ax+B)(x^2+2)+(Cx+D)(x^2+1)}{(x^2+1)(x^2+2)}.
\end{align*}

In this case the determination of $A$,~$B$, $C$,~$D$ is not
so easy. It will be simpler to proceed as follows:
Since the given fraction and the fraction found by
adding the partial fractions are equal, and have
\emph{identical} denominators, the numerators must also be
identically the same. In such a case, and for such
algebraical expressions as those with which we are
dealing here, \emph{the coefficients of the same powers of~$x$
are equal and of same sign}.

Hence, since
\begin{align*}
x^3-2
  &= (Ax+B)(x^2+2) + (Cx+D)(x^2+1) \\
  &= (A+C)x^3 + (B+D)x^2 + (2A+C)x + 2B+D,
\end{align*}
we have $1=A+C$;\quad $0=B+D$ (the coefficient of~$x^2$
in the left expression being zero); $0=2A+C$; and
$-2=2B+D$. Here are four equations, from which
we readily obtain $A=-1$; $B=-2$; $C=2$; $D=0$;
so that the partial fractions are $\dfrac{2(x+1)}{x^2+2} - \dfrac{x+2}{x^2+1}$.
\DPPageSep{139.png}{127}%
This method can always be used; but the method
shown first will be found the quickest in the case of
factors in~$x$ only.

\Paragraph{Case~III\@.} When, among the factors of the denominator
there are some which are raised to some power,
one must allow for the possible existence of partial
fractions having for denominator the several powers
of that factor up to the highest. For instance, in
splitting the fraction $\dfrac{3x^2-2x+1}{(x+1)^2(x-2)}$ we must allow for
the possible existence of a denominator~$x+1$ as well
as $(x+1)^2$ and~$(x-2)$.

It maybe thought, however, that, since the numerator
of the fraction the denominator of which is $(x+1)^2$
may contain terms in~$x$, we must allow for this in
writing $Ax+B$ for its numerator, so that
\[
\frac{3x^2 - 2x + 1}{(x+1)^2(x-2)}
  = \frac{Ax+B}{(x+1)^2} + \frac{C}{x+1} + \frac{D}{x-2}.
\]
If, however, we try to find $A$,~$B$,~$C$ and~$D$ in this case,
we fail, because we get four unknowns; and we have
only three relations connecting them, yet
\[
\frac{3x^2 - 2x + 1}{(x+1)^2(x-2)}
  = \frac{x-1}{(x+1)^2} + \frac{1}{x+1} + \frac{1}{x-2}.
\]

But if we write
\[
\frac{3x^2 - 2x + 1}{(x+1)^2(x-2)}
  = \frac{A}{(x+1)^2} + \frac{B}{x+1} + \frac{C}{x-2},
\]
we get
\[
3x^2 - 2x+1 = A(x-2) + B(x+1)(x-2) + C(x+1)^2,
\]
\DPPageSep{140.png}{128}%
which gives $C=1$ for $x=2$. Replacing $C$ by its value,
transposing, gathering like terms and dividing by
$x-2$, we get $-2x= A+B(x+1)$, which gives $A=-2$
for $x=-1$. Replacing $A$ by its value, we get
\[
2x = -2+B(x+1).
\]

Hence $B=2$; so that the partial fractions are:
\[
\frac{2}{x+1} - \frac{2}{(x+1)^2} + \frac{1}{x-2},
\]
instead of $\dfrac{1}{x+1} + \dfrac{x-1}{(x+1)^2} + \dfrac{1}{x-2}$ stated above as being
the fractions from which $\dfrac{3x^2-2x+1}{(x+1)^2(x-2)}$ was obtained.
The mystery is cleared if we observe that $\dfrac{x-1}{(x+1)^2}$ can
itself be split into the two fractions $\dfrac{1}{x+1} - \dfrac{2}{(x+1)^2}$, so
that the three fractions given are really equivalent to
\[
\frac{1}{x+1} + \frac{1}{x+1} - \frac{2}{(x+1)^2} + \frac{1}{x-2}
  = \frac{2}{x+1} - \frac{2}{(x+1)^2} + \frac{1}{x-2},
\]
which are the partial fractions obtained.

We see that it is sufficient to allow for one numerical
term in each numerator, and that we always get the
ultimate partial fractions.

When there is a power of a factor of~$x^2$ in the
denominator, however, the corresponding numerators
must be of the form $Ax+B$; for example,
\[
\frac{3x-1}{(2x^2-1)^2(x+1)}
  = \frac{Ax+B}{(2x^2-1)^2} + \frac{Cx+D}{2x^2-1} + \frac{E}{x+1},
\]
\DPPageSep{141.png}{129}%
which gives
\[%[** TN: Set on two lines in the original]
3x - 1 = (Ax + B)(x + 1)
       + (Cx + D)(x + 1)(2x^2 - 1) + E(2x^2 - 1)^2.
\]

For $x = -1$, this gives $E = -4$. Replacing, transposing,
collecting like terms, and dividing by $x + 1$,
we get
\[
16x^3 - 16x^2 + 3 = 2Cx^3 + 2Dx^2 + x(A - C) + (B - D).
\]

Hence $2C = 16$ and $C = 8$; $2D = -16$ and $D = -8$;
$A - C = 0$ or $A - 8 = 0$ and $A = 8$, and finally, $B - D = 3$
or $B = -5$. So that we obtain as the partial fractions:
\[
\frac{(8x - 5)}{(2x^2 - 1)^2} + \frac{8(x - 1)}{2x^2 - 1} - \frac{4}{x + 1}.
\]

It is useful to check the results obtained. The
simplest way is to replace $x$ by a single value, say~$+1$,
both in the given expression and in the partial
fractions obtained.

Whenever the denominator contains but a power of
a single factor, a very quick method is as follows:

Taking, for example, $\dfrac{4x + 1}{(x + 1)^3}$, let $x + 1 = z$; then
$x = z - 1$.

Replacing, we get
\[
\frac{4(z - 1) + 1}{z^3} = \frac{4z - 3}{z^3} = \frac{4}{z^2} - \frac{3}{z^3}.
\]

The partial fractions are, therefore,
\[
\frac{4}{(x + 1)^2} - \frac{3}{(x + 1)^3}.
\]
\DPPageSep{142.png}{130}%

Application to differentiation. Let it be required
to differentiate $y = \dfrac{5-4x}{6x^2 + 7x - 3}$; we have
\begin{align*}
\frac{dy}{dx}
  &= -\frac{(6x^2+7x-3) × 4 + (5 - 4x)(12x + 7)}{(6x^2 + 7x - 3)^2}\\
  &=  \frac{24x^2 - 60x - 23}{(6x^2 + 7x - 3)^2}.
\end{align*}

If we split the given expression into
\[
\frac{1}{3x-1} - \frac{2}{2x+3},
\]
we get, however,
\[
\frac{dy}{dx} = -\frac{3}{(3x-1)^2} + \frac{4}{(2x+3)^2},
\]
which is really the same result as above split into
partial fractions. But the splitting, if done after
differentiating, is more complicated, as will easily be
seen. When we shall deal with the \emph{integration} of
such expressions, we shall find the splitting into
partial fractions a precious auxiliary (see \Pageref{partfracs}).


\Exercises{XI} (See \Pageref[page]{AnsEx:XI} for Answers.)

Split into fractions:
\begin{Problems}[2]
\Item{(1)} $\dfrac{3x + 5}{(x - 3)(x + 4)}$.
\Item{(2)} $\dfrac{3x - 4}{(x - 1)(x - 2)}$.
\ResetCols{2}

\Item{(3)} $\dfrac{3x + 5}{x^2 + x - 12}$.
\Item{(4)} $\dfrac{x + 1}{x^2 - 7x + 12}$.
\ResetCols{2}

\Item{(5)} $\dfrac{x - 8}{(2x + 3)(3x - 2)}$.
\Item{(6)} $\dfrac{x^2 - 13x + 26}{(x - 2)(x - 3)(x - 4)}$.
\ResetCols{1}
\DPPageSep{143.png}{131}%

\Item{(7)} $\dfrac{x^2 - 3x + 1}{(x - 1)(x + 2)(x - 3)}$.

\Item{(8)} $\dfrac{5x^2 + 7x + 1}{(2x + 1)(3x - 2)(3x + 1)}$.
\ResetCols{2}

\Item{(9)} $\dfrac{x^2}{x^3 - 1}$.
\Item{(10)} $\dfrac{x^4 + 1}{x^3 + 1}$.
\ResetCols{2}

\Item{(11)} $\dfrac{5x^2 + 6x + 4}{(x +1)(x^2 + x + 1)}$.
\Item{(12)} $\dfrac{x}{(x - 1)(x - 2)^2}$.
\ResetCols{2}

\Item{(13)} $\dfrac{x}{(x^2 - 1)(x + 1)}$.
\Item{(14)} $\dfrac{x + 3}{ (x +2)^2(x - 1)}$.
\ResetCols{2}

\Item{(15)} $\dfrac{3x^2 + 2x + 1}{(x + 2)(x^2 + x + 1)^2}$.
\Item{(16)} $\dfrac{5x^2 + 8x - 12}{(x + 4)^3}$.
\ResetCols{2}

\Item{(17)} $\dfrac{7x^2 + 9x - 1}{(3x - 2)^4}$.
\Item{(18)} $\dfrac{x^2}{(x^3 - 8)(x - 2)}$.
\end{Problems}


\Section{Differential of an Inverse Function.}

Consider the function (see \Pageref{function}) $y = 3x$; it can be
expressed in the form $x = \dfrac{y}{3}$; this latter form is called
the \emph{inverse function} to the one originally given.

If $y = 3x$,\quad $\dfrac{dy}{dx} = 3$; if $x=\dfrac{y}{3}$,\quad $\dfrac{dx}{dy} = \dfrac{1}{3}$, and we see that
\[
\frac{dy}{dx} = \frac{1}{\ \dfrac{dx}{dy}\ }\quad \text{or}\quad
\frac{dy}{dx} × \frac{dx}{dy} = 1.
\]

Consider $y= 4x^2$, $\dfrac{dy}{dx} = 8x$; the inverse function is
\[
x = \frac{y^{\efrac{1}{2}}}{2},\quad \text{and}\quad
\frac{dx}{dy} = \frac{1}{4\sqrt{y}} = \frac{1}{4 × 2x} = \frac{1}{8x}.
\]
\DPPageSep{144.png}{132}%
%
\begin{DPalign*}
\lintertext{\indent Here again}
\frac{dy}{dx}×\frac{dx}{dy} &= 1.
\end{DPalign*}

It can be shown that for all functions which can be
put into the inverse form, one can always write
\[
\frac{dy}{dx} × \frac{dx}{dy} = 1\quad \text{or}\quad
\frac{dy}{dx} = \frac{1}{\ \dfrac{dx}{dy}\ }.
\]

It follows that, being given a function, if it be
easier to differentiate the inverse function, this may
be done, and the reciprocal of the differential coefficient
of the inverse function gives the differential coefficient
of the given function itself.

As an example, suppose that we wish to differentiate
$y=\sqrt[2]{\dfrac{3}{x}-1}$. We have seen one way of doing this,
by writing $u=\dfrac{3}{x}-1$, and finding $\dfrac{dy}{du}$ and~$\dfrac{du}{dx}$. This
gives
\[
\frac{dy}{dx} = -\frac{3}{2x^2\sqrt{\dfrac{3}{x} -1}}.
\]

If we had forgotten how to proceed by this method,
or wished to check our result by some other way of
obtaining the differential coefficient, or for any other
reason we could not use the ordinary method, we can
proceed as follows: The inverse function is $x=\dfrac{3}{1+y^2}$.
\[
\frac{dx}{dy} = -\frac{3 × 2y}{(1+y^2)^2} = -\frac{6y}{(1+y^2)^2};
\]
\DPPageSep{145.png}{133}%
hence
\[
\frac{dy}{dx} = \frac{1}{\ \dfrac{dx}{dy}\ }
  = -\frac{(1+y^2)^2}{6y}
  = -\frac{\left(1+\dfrac{3}{x} -1\right)^2}{6×\sqrt[2]{\dfrac{3}{x}-1}}
  = -\frac{3}{2x^2\sqrt{\dfrac{3}{x}-1}}.
\]

Let us take as an other example $y=\dfrac{1}{\sqrt[3]{\theta +5}}$.

The inverse function is $\theta=\dfrac{1}{y^3}-5$ or $\theta=y^{-3}-5$, and
\[
\frac{d\theta}{dy} = -3y^{-4} = -3\sqrt[3]{(\theta + 5)^4}.
\]

It follows that $\dfrac{dy}{dx} = -\dfrac{1}{3\sqrt{(\theta +5)^4}}$, as might have
been found otherwise.

We shall find this dodge most useful later on;
meanwhile you are advised to become familiar with
it by verifying by its means the results obtained in
Exercises~I. (\Pageref{Ex:I}), Nos.\ 5,~6,~7; Examples (\Pageref{ExNo1}), %[ ** xrefs and hard-coded pages]
Nos.\ 1,~2,~4; and Exercises~VI. (\Pageref{Ex:VI}), Nos.\ 1,~2,~3
and~4.

\tb

You will surely realize from this chapter and the
preceding, that in many respects the calculus is an
\emph{art} rather than a \emph{science}: an art only to be acquired,
as all other arts are, by practice. Hence you should
work many examples, and set yourself other examples,
to see if you can work them out, until the various
artifices become familiar by use.
\DPPageSep{146.png}{134}%

