
\ChapterStar{Epilogue and Apologue}

\First{It} may be confidently assumed that when this
tractate ``Calculus made Easy'' falls into the hands
of the professional mathematicians, they will (if not
too lazy) rise up as one man, and damn it as being a
thoroughly bad book. Of that there can be, from
their point of view, no possible manner of doubt
whatever. It commits several most grievous and
deplorable errors.

First, it shows how ridiculously easy most of the
operations of the calculus really are.

Secondly, it gives away so many trade secrets. By
showing you that \emph{what one fool can do, other fools
can do also}, it lets you see that these mathematical
swells, who pride themselves on having mastered such
an awfully difficult subject as the calculus, have no
such great reason to be puffed up. They like you to
think how terribly difficult it is, and don't want that
superstition to be rudely dissipated.

Thirdly, among the dreadful things they will say
about ``So Easy'' is this: that there is an utter failure
on the part of the author to demonstrate with rigid
\DPPageSep{262.png}{250}%
and satisfactory completeness the validity of sundry
methods which he has presented in simple fashion,
and has even \emph{dared to use} in solving problems! But
why should he not? You don't forbid the use of
a watch to every person who does not know how to
make one? You don't object to the musician playing
on a violin that he has not himself constructed. You
don't teach the rules of syntax to children until they
have already become fluent in the \emph{use} of speech. It
would be equally absurd to require general rigid
demonstrations to be expounded to beginners in the
calculus.

One other thing will the professed mathematicians
say about this thoroughly bad and vicious book: that
the reason why it is \emph{so easy} is because the author has
left out all the things that are really difficult. And
the ghastly fact about this accusation is that---\emph{it
is true!} That is, indeed, why the book has been
written---written for the legion of innocents who have
hitherto been deterred from acquiring the elements of
the calculus by the stupid way in which its teaching
is almost always presented. Any subject can be made
repulsive by presenting it bristling with difficulties.
The aim of this book is to enable beginners to learn
its language, to acquire familiarity with its endearing
simplicities, and to grasp its powerful methods of
solving problems, without being compelled to toil
through the intricate out-of-the-way (and mostly
irrelevant) mathematical gymnastics so dear to the
unpractical mathematician.
\DPPageSep{263.png}{251}%

There are amongst young engineers a number on
whose ears the adage that \emph{what one fool can do,
another can}, may fall with a familiar sound. They
are earnestly requested not to give the author
away, nor to tell the mathematicians what a fool
he really is.
\DPPageSep{264.png}{252}%


\AltChapter{Table of Standard Forms}

%[** TN: Manually split onto three pages]
\vfil
\begin{center}
\Pagelabel{stdforms1}
\begin{tabular}{|c|c|c|}
\hline
\multicolumn{1}{|c}{\DStrut$\dfrac{dy}{dx}$}
  & \multicolumn{1}{c}{$\longleftarrow\quad y\quad\longrightarrow$}
  & $\ds\int y\, dx$ \\
\hline
\ColumnHead{Algebraic.} \\
\DStrut$1$ & $x$     & $\frac{1}{2} x^2 + C$ \\
\DStrut$0$ & $a$     & $ax + C $             \\
\DStrut$1$ & $x ± a$ & $\frac{1}{2} x^2 ± ax + C$ \\
\DStrut$a$ & $ax $   & $\frac{1}{2} ax^2 + C $\\
\DStrut$2x$ & $x^2$  & $\frac{1}{3} x^3 + C $ \\
\DStrut$nx^{n-1}$ & $x^n$ &$ \dfrac{1}{n+1} x^{n+1} + C $\\
\DStrut$-x^{-2} $ & $x^{-1}$ & $\log_\epsilon x + C$ \\
\DStrut$\dfrac{du}{dx} ± \dfrac{dv}{dx} ± \dfrac{dw}{dx}$
   & $u ± v ± w$     & $\int u\, dx ± \int v\, dx ± \int w\, dx$ \\
\DStrut$u\, \dfrac{dv}{dx} + v\, \dfrac{du}{dx}$
   & $uv$  & No general form known \\
\DStrut$\dfrac{v\, \dfrac{du}{dx} - u\, \dfrac{dv}{dx}}{v^2}$
   & $\dfrac{u}{v}$ & No general form known \\
\DStrut$\dfrac{du}{dx}$ & $u$ & $ux - \int x\, du + C$ \\
\hline
\end{tabular}
\vfil\newpage
\null\vfil
%
\begin{tabular}{|c|c|c|}
\hline
\multicolumn{1}{|c}{\DStrut$\dfrac{dy}{dx}$}
  & \multicolumn{1}{c}{$\longleftarrow\quad y\quad\longrightarrow$}
  & $\ds\int y\, dx$ \\
\hline
\ColumnHead{Exponential and Logarithmic.} \\
$\epsilon^x$ & $\epsilon^x$ & $\epsilon^x + C$ \\
$x^{-1}$     & $\log_\epsilon x$ & $ x(\log_\epsilon x - 1) + C$ \\
$0.4343 × x^{-1}$ & $\log_{10} x$ & $0.4343x (\log_\epsilon x - 1) + C$ \\
\DStrut$a^x \log_\epsilon a$ & $a^x$ & $\dfrac{a^x}{\log_\epsilon a} + C$ \\
\hline
%
\ColumnHead{Trigonometrical.} \\
$\cos x$  & $\sin x$ & $-\cos x + C $ \\
$-\sin x$ & $\cos x$ & $\sin x + C $ \\
$\sec^2 x$& $\tan x$ & $-\log_\epsilon \cos x + C $ \\
\hline
%
\ColumnHead{Circular (Inverse).} \\
$\dfrac{1}{\sqrt{(1-x^2)}}$ & $\arcsin x$ & $x · \arcsin x + \sqrt{1 - x^2} + C$ \\
\DStrut$-\dfrac{1}{\sqrt{(1-x^2)}}$ & $\arccos x$ & $x · \arccos x - \sqrt{1 - x^2} + C$ \\
$\dfrac{1}{1+x^2}$ & $\arctan x$ & $x · \arctan x - \frac{1}{2} \log_\epsilon (1 + x^2) + C$ \\
\hline
%\DPPageSep{265.png}{253}%
\ColumnHead{Hyperbolic.} \\
$\cosh x   $ & $\sinh x$ & $\cosh x + C$ \\
$\sinh x   $ & $\cosh x$ & $\sinh x + C$ \\
$\sech^2 x $ & $\tanh x$ & $\log_\epsilon \cosh x + C $ \\
\hline
\end{tabular}
\vfil\newpage
%
%[** TN: Manual coaxing to get this portion to fit page horizontally/vertically]
\small\Pagelabel{stdforms2}%
\makebox[0pt][c]{%
\begin{tabular}{|c|c|c|}
\hline
\multicolumn{1}{|c}{\DStrut$\dfrac{dy}{dx}$}
  & \multicolumn{1}{c}{$\longleftarrow\quad y\quad\longrightarrow$}
  & $\ds\int y\, dx$ \\
\hline
\ColumnHead{Miscellaneous.} \\
\DStrut$-\dfrac{1}{(x + a)^2}$ & $\dfrac{1}{x + a}$ & $ \log_\epsilon (x+a) + C $ \\
$-\dfrac{x}{(a^2 + x^2)^{\efrac{3}{2}}}$
  & $\dfrac{1}{\sqrt{a^2 + x^2}}$
  & $\log_\epsilon (x + \sqrt{a^2 + x^2}) + C $ \\
\DStrut$\mp \dfrac{b}{(a ± bx)^2}$
  & $\dfrac{1}{a ± bx}$
  & $± \dfrac{1}{b} \log_\epsilon (a ± bx) + C $ \\
$-\dfrac{3a^2x}{(a^2 + x^2)^{\efrac{5}{2}}}$
  & $\dfrac{a^2}{(a^2 + x^2)^{\efrac{3}{2}}}$
  & $\dfrac{x}{\sqrt{a^2 + x^2}} + C $ \\
\DStrut$ a · \cos ax$ & $\sin ax$ & $-\dfrac{1}{a} \cos ax + C $ \\
$-a · \sin ax$ & $\cos ax$ & $ \dfrac{1}{a} \sin ax + C $ \\
\DStrut$ a · \sec^2ax$& $\tan ax$ & $-\dfrac{1}{a} \log_\epsilon \cos ax + C $ \\
$ \sin 2x$ & $\sin^2 x$ & $\dfrac{x}{2} - \dfrac{\sin 2x}{4} + C $ \\
\DStrut$-\sin 2x$ & $\cos^2 x$ & $\dfrac{x}{2} + \dfrac{\sin 2x}{4} + C $ \\
$n · \sin^{n-1} x · \cos x$
  & $ \sin^n x$
  & \footnotesize$\DStrut\ds-\frac{\cos x}{n} \sin^{n-1} x
     + \frac{n-1}{n} \int \sin^{n-2} x\, dx + C$ \\
$-\dfrac{\cos x}{\sin^2 x}$
  & $\dfrac{1}{\sin x}$
  & $\log_\epsilon \tan \dfrac{x}{2} + C$ \\
\DStrut$-\dfrac{\sin 2x}{\sin^4 x}$
  & $\dfrac{1}{\sin^2 x}$
  & $ -\cotan x + C$ \\
$\dfrac{\sin^2 x - \cos^2 x}{\sin^2 x · \cos^2 x}$
  & $ \dfrac{1}{\sin x · \cos x}$
  & $ \log_\epsilon \tan x + C $ \\
\DStrut\parbox{0.2\linewidth}{\raggedright\scriptsize
  $n · \sin mx · \cos nx + m · \sin nx · \cos mx $}
  & $\sin mx · \sin nx$
  & $\frac{1}{2} \cos(m - n)x - \frac{1}{2} \cos(m + n)x + C$ \\
$ 2a·\sin 2ax$ & $\sin^2 ax$ & $\dfrac{x}{2} - \dfrac{\sin 2ax}{4a} + C $ \\
\DStrut$-2a·\sin 2ax$ & $\cos^2 ax$ & $\dfrac{x}{2} + \dfrac{\sin 2ax}{4a} + C $ \\
\hline
\end{tabular}}%
\end{center}
\DPPageSep{266.png}{254}%


\AltChapter[Answers to Exercises]{Answers}

\begin{Answers}[3]{I}{(\Pageref{Ex:I}.)}{}
\Item{(1)} $\dfrac{dy}{dx} = 13x^{12}$.
\Item{(2)} $\dfrac{dy}{dx} = - \dfrac{3}{2} x^{-\efrac{5}{2}}$.
\Item{(3)} $\dfrac{dy}{dx} = 2ax^{(2a-1)}$.

\ResetCols{3}
\Item{(4)} $\dfrac{du}{dt} = 2.4t^{1.4}$.
\Item{(5)} $\dfrac{dz}{du} = \dfrac{1}{3} u^{-\efrac{2}{3}}$.
\Item{(6)} $\dfrac{dy}{dx} = -\dfrac{5}{3}x^{\DPtypo{-\efrac{8}{5}}{-\efrac{8}{3}}}$.

\ResetCols{2}
\Item{(7)} $\dfrac{du}{dx} = -\dfrac{8}{5}x^{-\efrac{13}{5}}$.
\Item{(8)} $\dfrac{dy}{dx} = 2ax^{a-1}$.

\ResetCols{2}
\Item{(9)} $\dfrac{dy}{dx} = \dfrac{3}{q} x^{\efrac{3-q}{q}}$.
\Item{(10)} $\dfrac{dy}{dx} = -\dfrac{m}{n} x^{-\efrac{m+n}{n}}$.
\end{Answers}


\begin{Answers}[3]{II}{(\Pageref{Ex:II}.)}{}

\Item{(1)} $\dfrac{dy}{dx} = 3ax^2$.
\Item{(2)} $\dfrac{dy}{dx} = 13 × \frac{3}{2}x^{\efrac{1}{2}}$.
\Item{(3)} $\dfrac{dy}{dx} = 6x^{-\efrac{1}{2}}$.

\ResetCols{3}
\Item{(4)} $\dfrac{dy}{dx} = \dfrac{1}{2}c^{\efrac{1}{2}} x^{-\efrac{1}{2}}$.
\Item{(5)} $\dfrac{du}{dz} = \dfrac{an}{c} z^{n-1}$.
\Item{(6)} $\dfrac{dy}{dt} = 2.36t$.

\ResetCols{1}
\Item{(7)} $\dfrac{dl_t}{dt} = 0.000012×l_0$.

\Item{(8)} $\dfrac{dC}{dV} = abV^{b-1}$, $0.98$, $3.00$ and $7.47$~candle power per volt respectively.

\Item{(9)} $\begin{aligned}[t]
  \dfrac{dn}{dD} &= -\dfrac{1}{LD^2} \sqrt{\dfrac{gT}{\pi \sigma}}, &
  \dfrac{dn}{dL} &= -\dfrac{1}{DL^2} \sqrt{\dfrac{gT}{\pi \sigma}}, \\
%
  \dfrac{dn}{d \sigma}
  &= -\dfrac{1}{2DL} \sqrt{\dfrac{gT}{\pi \sigma^3}}, &
  \dfrac{dn}{dT} &=  \dfrac{1}{2DL} \sqrt{\dfrac{g}{\pi \sigma T}}.
\end{aligned}$
\DPPageSep{267.png}{255}%

\Item{(10)} $\dfrac{\text{Rate of change of~$P$ when $t$~varies}}
            {\text{Rate of change of~$P$ when $D$~varies}}
  = - \dfrac{D}{t}$.

\Item{(11)} $2\pi$, $2\pi r$, $\pi l$, $\frac{2}{3}\pi rh$, $8\pi r$, $4\pi r^2$. \hfil
(12)~$\dfrac{dD}{dT} = \dfrac{0.000012l_t}{\pi}$.
\end{Answers}


\begin{Answers}{III}{(\Pageref{Ex:III}.)}{}

\Item{(1)} (\textit{a}) $1 + x + \dfrac{x^2}{2} + \dfrac{x^3}{6} + \dfrac{x^4}{24} + \ldots$ \qquad
    (\textit{b}) $2ax + b$. \qquad (\textit{c}) $2x + 2a$.

    (\textit{d}) $3x^2 + 6ax + 3a^2$.

\ResetCols{2}
\Item{(2)} $\dfrac{dw}{dt} = a - bt$.
\Item{(3)} $\dfrac{dy}{dx} = 2x$.

\ResetCols{1}
\Item{(4)} $14110x^4 - 65404x^3 - 2244x^2 + 8192x + 1379$.

\ResetCols{2}
\Item{(5)} $\dfrac{dx}{dy} = 2y + 8$.
\Item{(6)} $185.9022654x^2 + 154.36334$.

\ResetCols{2}
\Item{(7)} $\dfrac{-5}{(3x + 2)^2}$.
\Item{(8)} $\dfrac{6x^4 + 6x^3 + 9x^2}{(1 + x + 2x^2)^2}$.

\ResetCols{2}
\Item{(9)} $\dfrac{ad - bc}{(cx + d)^2}$.
\Item{(10)} $\dfrac{anx^{-n-1} + bnx^{n-1} + 2nx^{-1}}{(x^{-n} + b)^2}$.

\ResetCols{1}
\Item{(11)} $b + 2ct$.

\Item{(12)} $R_0(a + 2bt)$,\quad $R_0 \left(a + \dfrac{b}{2\sqrt{t}}\right)$,\quad
  $-\dfrac{R_0(a + 2bt)}{(1 + at + bt^2)^2}$\quad or\quad $\dfrac{R^2 (a + 2bt)}{R_0}$.

\Item{(13)} $1.4340(0.000014t - \DPtypo{0.000828}{0.001024})$,\quad $-0.00117$,\quad $-0.00107$,\quad $-0.00097$.

\Item{(14)} $\dfrac{dE}{dl} = b + \dfrac{k}{i}$,\quad $\dfrac{dE}{di} = -\dfrac{c + kl}{i^2}$.
\end{Answers}


\begin{Answers}[2]{IV}{(\Pageref{Ex:IV}.)}{}

\Item{(1)} $17 + 24x$;\quad $24$.
\Item{(2)} $\dfrac{x^2 + 2ax - a}{(x + a)^2}$;\quad $\dfrac{2a(a + 1)}{(x + a)^3}$.

\ResetCols{1}
\Item{(3)} $1 + x + \dfrac{x^2}{1 × 2} + \dfrac{x^3}{1 × 2 × 3}$;\quad $1 + x + \dfrac{x^2}{1 × 2}$.

% [** TN: Freely reformatting subitems]
\Item{(4)} (\textit{Exercises III.}):
\begin{itemize}
\item[(1)] (\textit{a}) $\dfrac{d^2 y}{dx^2} = \dfrac{d^3 y}{dx^3} = 1 + x + \frac{1}{2}x^2 + \frac{1}{6} x^3 + \ldots$. \\
  (\textit{b}) $2a$, $0$.\hfil
  (\textit{c}) $2$, $0$.\hfil
  (\textit{d}) $6x + 6a$, $6$.

\DPPageSep{268.png}{256}%
\item[(2)] $-b$, $0$.\hfil (3) $2$, $0$.\hfil

\item[(4)] $\begin{gathered}[t]
    56440x^3 - 196212x^2 - 4488x + 8192. \\
    169320x^2 - 392424x - 4488.
    \end{gathered}$

\item[(5)] $2$, $0$. \hfil (6) $371.80453x$, $371.80453$. \hfil

\item[(7)] $\dfrac{30}{(3x + 2)^3}$,\quad $-\dfrac{270}{(3x + 2)^4}$.
\end{itemize}

\paragraph{\normalfont(\textit{Examples}, \Pageref{examples3}):}
\begin{itemize}
\item[(1)] $\dfrac{6a}{b^2} x$,\quad $\dfrac{6a}{b^2}$.\hfil
(2) $\dfrac{3a \sqrt{b}} {2 \sqrt{x}} - \dfrac{6b \sqrt[3]{a}}{x^3}$,\quad
$\dfrac{18b \sqrt[3]{a}}{x^4} - \dfrac{3a \sqrt{b}}{4 \sqrt{x^3}}$\DPtypo{}{.}

\Item{(3)}
$\dfrac{2}{\sqrt[3]{\theta^8}} - \dfrac{1.056}{\sqrt[5]{\theta^{11}}}$,\quad
$\dfrac{2.3232}{\sqrt[5]{\theta^{16}}} - \dfrac{16}{3 \sqrt[3]{\theta^{11}}}$.

\Item{(4)} $\begin{gathered}[t]
  810t^4 - 648t^3 + 479.52t^2 - 139.968t + 26.64. \\
  3240t^3 - 1944t^2 + 959.04t - 139.968.
  \end{gathered}$

\Item{(5)}  $12x + 2$, $12$.\hfil
(6) $6x^2 - 9x$,\quad $12x - 9$.\hfil

\Item{(7)}
$\begin{aligned}[t]
&\dfrac{3}{4} \left(\dfrac{1}{\sqrt{\theta}} + \dfrac{1}{\sqrt{\theta^5}}\right)
+\dfrac{1}{4} \left(\dfrac{15}{\sqrt{\theta^7}} - \dfrac{1}{\sqrt{\theta^3}}\right). \\
&\dfrac{3}{8} \left(\dfrac{1}{\sqrt{\theta^5}} - \dfrac{1}{\sqrt{\theta^3}}\right)
-\dfrac{15}{8}\left(\dfrac{7}{\sqrt{\theta^9}} + \dfrac{1}{\sqrt{\theta^7}}\right).
\end{aligned}$
\end{itemize}
\end{Answers}


\begin{Answers}{V}{(\Pageref{Ex:V}.)}{}

\Item{(2)}  64; 147.2; and 0.32 feet per second.

\ResetCols{2}
\Item{(3)}  $x = a - gt$; $\ddot{x} = -g$.

\Item{(4)}  $45.1$ feet per second.

\ResetCols{1}
\Item{(5)}  $12.4$ feet per second per second.\quad Yes.

\Item{(6)} Angular velocity ${} = 11.2$ radians per second; angular
  acceleration ${}= 9.6$ radians per second per second.

\Item{(7)}  $v = 20.4t^2 - 10.8$.\quad $a = 40.8t$.\quad $172.8$~in./sec., $122.4~\text{in./sec}^2$.

\Item{(8)}  $v = \dfrac{1}{30 \sqrt[3]{(t - 125)^2}}$,\quad $a = - \dfrac{1}{45 \sqrt[3]{(t - 125)^5}}$.

\Item{(9)}  $v = 0.8 - \dfrac{8t}{(4 + t^2)^2}$,\quad $a = \dfrac{24t^2 - 32}{(4 + t^2)^3}$,\quad $0.7926$ and $0.00211$.

\Item{(10)}  $n = 2$, $n = 11$.
\end{Answers}
\DPPageSep{269.png}{257}%


\begin{Answers}[3]{VI}{(\Pageref{Ex:VI}.)}{}

\Item{(1)}  $\dfrac{x}{\sqrt{ x^2 + 1}}$.
\Item{(2)}  $\dfrac{x}{\sqrt{ x^2 + a^2}}$.
\Item{(3)}  $- \dfrac{1}{2 \sqrt{(a + x)^3}}$.

\ResetCols{2}
\Item{(4)}  $\dfrac{ax}{\sqrt{(a - x^2)^3}}$.
\Item{(5)}  $\dfrac{2a^2 - x^2}{x^3 \sqrt{ x^2 - a^2}}$.

\ResetCols{2}
\Item{(6)}  $ \dfrac{\frac{3}{2} x^2 \left[ \frac{8}{9} x \left( x^3 + a \right) - \left( x^4 + a \right) \right]}{(x^4 + a)^{\efrac{2}{3}} (x^3 + a)^{\efrac{3}{2}}}$
\Item{(7)}  $\dfrac{2a \left(x - a \right)}{(x + a)^3}$.

\ResetCols{2}
\Item{(8)}  $\frac{5}{2} y^3$.
\Item{(9)}  $\dfrac{1}{(1 - \theta) \sqrt{1 - \theta^2}}$.
\end{Answers}


\begin{Answers}{VII}{(\Pageref{Ex:VII}.)}{}

\Item{(1)}  $\dfrac{dw}{dx} = \dfrac{3x^2 \left( 3 + 3x^3 \right)} {27 \left(\frac{1}{2} x^3 + \frac{1}{4} x^6 \right)^3}$.

\Item{(2)}  $\dfrac{dv}{dx} = - \dfrac{12x}{\sqrt{1 + \sqrt{2} + 3x^2} \left(\sqrt{3} + 4 \sqrt{1 + \sqrt{2} + 3x^2}\right)^2}$.

\Item{(3)}  $\dfrac{du}{dx} = - \dfrac{x^2 \left(\sqrt{3} + x^3 \right)} {\sqrt{ \left[ 1 + \left( 1 + \dfrac{x^3}{\sqrt{3}} \right) ^2 \right]^3}} $
\end{Answers}


\begin{Answers}{VIII}{(\Pageref{Ex:VIII}.)}{}

\Item{(2)}  $1.44$.

\Item{(4)}  $\dfrac{dy}{dx} = 3x^2 + 3$; and the numerical values are:
     $3$,~$3 \frac{3}{4}$, $6$,~and~$15$.

\Item{(5)}  $ ± \sqrt{2}$.

\Item{(6)}  $ \dfrac{dy}{dx} = - \dfrac{4}{9} \dfrac{x}{y}$. Slope is zero where $x = 0$; and is $\mp \dfrac{1}{3 \sqrt{2}}$ where $x = 1$.

\Item{(7)}  $m = 4$, $n = -3$.

\Item{(8)}  Intersections at $x = 1$, $x = -3$. Angles $153°\;26'$, $2°\;28'$.

\Item{(9)}  Intersection at $x = 3.57$, $y = 3.50$. Angle $16°\;16'$.

\Item{(10)}  $x = \frac{1}{3}$, $y = 2 \frac{1}{3}$, $b = -\frac{5}{3}$.
\end{Answers}
\DPPageSep{270.png}{258}%


\begin{Answers}{IX}{(\Pageref{Ex:IX}.)}{}

\Item{(1)}  Min.: $x = 0$, $y = 0$; max.: $x = -2$, $y = -4$.

\ResetCols{2}
\Item{(2)}  $x = a$.

\Item{(4)}  $25 \sqrt{3}$ square inches.

\ResetCols{1}
\Item{(5)}  $\dfrac{dy}{dx} = - \dfrac{10}{x^2} + \dfrac{10}{(8 - x)^2}$; $x = 4$; $y = 5$.

\Item{(6)}  Max.\ for $x = -1$; min.\ for $x = 1$.

\Item{(7)}  Join the middle points of the four sides.

\Item{(8)}  $r = \frac{2}{3} R$, $r = \dfrac{R}{2}$, no max.

\Item{(9)}  $r = R \sqrt{\dfrac{2}{3}}$, $r = \dfrac{R}{\sqrt{2}}$, $r = 0.8506R$.

\Item{(10)}  At the rate of $\dfrac{8}{r}$ square feet per second.

\ResetCols{2}
\Item{(11)}  $r = \dfrac{R \sqrt{8}}{3}$.
\Item{(12)}  $n = \sqrt{\dfrac{NR}{r}}$.
\end{Answers}


\begin{Answers}{X}{(\Pageref{Ex:X}.)}{}

\Item{(1)}  Max.: $x = -2.19$, $y = 24.19$; min.:, $x = 1.52$, $y = -1.38$.

\Item{(2)}  $\dfrac{dy}{dx} = \dfrac{b}{a} - 2cx$; $\dfrac{d^2 y}{dx^2} = -2c$; $x = \dfrac{b}{2ac}$ (\emph{a maximum}).

\Item{(3)}  (\textit{a}) One maximum and two minima.  \\
(\textit{b}) One maximum. ($x = 0$; other points unreal.)

\ResetCols{2}
\Item{(4)}  Min.: $x = 1.71$, $y = 6.14$.

\Item{(5)}  Max: $x = -.5$, $y = 4$.

\ResetCols{1}
\Item{(6)}  Max.: $x = 1.414$, $y = 1.7675$.  \\
Min.: $x = -1.414$, $y = 1.7675$.

\Item{(7)}  Max.: $x = -3.565$, $y = 2.12$.  \\
Min.: $x = +3.565$, $y = 7.88$.

\ResetCols{2}
\Item{(8)}  $0.4N$, $0.6N$.

\Item{(9)}  $x = \sqrt{\dfrac{a}{c}}$.

\ResetCols{1}
\Item{(10)}  Speed $8.66$ nautical miles per hour. Time taken $115.47$~hours. \\
Minimum cost £$112$.~$12$\textit{s}.
\DPPageSep{271.png}{259}%

\Item{(11)}  Max.\ and min.\ for $x = 7.5$, $y = ±5.414$. (See example
no.~10, \Pageref{Example10}.)

\Item{(12)}  Min.: $x = \frac{1}{2}$, $y= 0.25$; max.: $x = - \frac{1}{3}$, $y= 1.408$.
\end{Answers}

%[** TN: Text block-dependent page break]
\begin{Answers}[3]{XI}{(\Pageref{Ex:XI}.)}{\newpage}

\Item{(1)}  $\dfrac{2}{ x - 3} + \dfrac{1}{ x + 4}$.

\Item{(2)}  $\dfrac{1}{ x - 1} + \dfrac{2}{ x - 2}$.

\Item{(3)}  $\dfrac{2}{ x - 3} + \dfrac{1}{ x + 4}$.

\ResetCols{2}
\Item{(4)}  $\dfrac{5}{ x - 4} - \dfrac{4}{ x - 3}$.

\Item{(5)}  $\dfrac{19}{13(2x + 3)} - \dfrac{22}{13(3x - 2)}$.

\ResetCols{1}
\Item{(6)}  $\dfrac{2}{ x - 2} + \dfrac{4}{ x - 3} - \dfrac{5}{ x - 4}$.

\Item{(7)}  $\dfrac{1}{6(x - 1)} + \dfrac{11}{15(x + 2)} + \dfrac{1}{10(x - 3)}$.

\Item{(8)}  $\dfrac{7}{9(3x + 1)} + \dfrac{71}{63(3x - 2)} - \dfrac{5}{7(2x + 1)}$.

\ResetCols{2}
\Item{(9)}  $\dfrac{1}{3(x - 1)} + \dfrac{2x + 1}{3(x^2 + x + 1)}$.

\Item{(10)} $x + \dfrac{2}{3(x + 1)} + \dfrac{1 - 2x}{3(x^2 - x + 1)}$.

\ResetCols{2}
\Item{(11)} $\dfrac{3}{(x + 1)} + \dfrac{2x + 1}{x^2 + x + 1}$.

\Item{(12)} $\dfrac{1}{ x - 1} - \dfrac{1}{ x - 2} + \dfrac{2}{(x - 2)^2}$.

\ResetCols{1}
\Item{(13)} $\dfrac{1}{4(x - 1)} - \dfrac{1}{4(x + 1)} + \dfrac{1}{2(x + 1)^2}$.

\Item{(14)} $\dfrac{4}{9(x - 1)} - \dfrac{4}{9(x + 2)} - \dfrac{1}{3(x + 2)^2}$.

\Item{(15)} $\dfrac{1}{ x + 2} - \dfrac{x - 1}{ x^2 + x + 1} - \dfrac{1}{(x^2 + x + 1)^2}$.

\Item{(16)} $\dfrac{5}{ x + 4} -\dfrac{32}{(x + 4)^2} + \dfrac{36}{(x + 4)^3}$.

\Item{(17)} $\dfrac{7}{9(3x - 2)^2} + \dfrac{55}{9(3x - 2)^3} + \dfrac{73}{9(3x - 2)^4}$.

\Item{(18)} $\dfrac{1}{6(x - 2)} + \dfrac{1}{3(x - 2)^2} - \dfrac{x}{6(x^2 + 2x + 4)}$.
\end{Answers}
\DPPageSep{272.png}{260}%


\begin{Answers}[3]{XII}{(\Pageref{Ex:XII}.)}{}

\Item{(1)}  $ab(\epsilon^{ax} + \epsilon^{-ax})$.

\Item{(2)}  $2at + \dfrac{2}{t}$.

\Item{(3)}  $\log_\epsilon n$.

\ResetCols{3}
\Item{(5)}  $npv^{n-1}$.

\Item{(6)}  $\dfrac{n}{x}$.

\Item{(7)}  $\dfrac{3\epsilon^{- \frac{x}{x-1}}}{(x - 1)^2}$.

\ResetCols{2}
\Item{(8)}  $6x \epsilon^{-5x} - 5(3x^2 + 1)\epsilon^{-5x}$.

\Item{(9)}  $\dfrac{ax^{a-1}}{x^a + a}$.

\ResetCols{1}
\Item{(10)}  $\left(\dfrac{6x}{3x^2-1} + \dfrac{1}{2\left(\sqrt x + x\right)}\right) \left(3x^2-1\right)\left(\sqrt x + 1\right)$.

\Item{(11)}  $\dfrac{1 - \log_\epsilon \left(x + 3\right)}{\left(x + 3\right)^2}$.

\ResetCols{2}
\Item{(12)}  $a^x\left(ax^{a-1} + x^a \log_\epsilon a\right)$.

\Item{(14)}  Min.: $y = 0.7$ for $x = 0.694$.

\ResetCols{2}
\Item{(15)}  $\dfrac{1 + x}{x}$.

\Item{(16)}  $\dfrac{3}{x} (\log_\epsilon ax)^2$.
\end{Answers}


\begin{Answers}{XIII}{(\Pageref{Ex:XIII}.)}{}

\Item{(1)}  Let $\dfrac{t}{T} = x$ ($\therefore t = 8x$), and use the Table on \Pageref[page]{littletable}.

\Item{(2)}  $T = 34.627$; $159.46$ minutes.

\Item{(3)}  Take $2t = x$; and use the Table on \Pageref[page]{littletable}.

\Item{(5)}  (\textit{a}) $x^x \left(1 + \log_\epsilon x\right)$;\quad
(\textit{b}) $2x(\epsilon^x)^x$;\quad
(\textit{c}) $\epsilon^{x^x} × x^x \left(1 + \log_\epsilon x\right)$.

\ResetCols{2}
\Item{(6)}  $0.14$ second.

\Item{(7)}  (\textit{a}) $1.642$;\quad (\textit{b}) $15.58$.

\ResetCols{1}
\Item{(8)}  $\mu = 0.00037$, $31^m \frac{1}{4}$. %[** Time units]

\Item{(9)}  $i$ is $63.4$\% of~$i_0$, $220$~kilometres.

\Item{(10)}  $0.133$, $0.145$, $0.155$, mean $0.144$; $-10.2$\%, $-0.9$\%, $+77.2$\%.

\ResetCols{2}
\Item{(11)}  Min.\ for $x = \dfrac{1}{\epsilon}$.

\Item{(12)}  Max.\ for $x = \epsilon$.

\ResetCols{1}
\Item{(13)}  Min.\ for $x = \log_\epsilon a$.
\end{Answers}
\DPPageSep{273.png}{261}%


\begin{Answers}{XIV}{(\Pageref{Ex:XIV}.)}{}

\Item{(1)} (i) $\dfrac{dy}{d\theta} = A \cos \left( \theta - \dfrac{\pi}{2} \right)$;

(ii) $\dfrac{dy}{d\theta} = 2\sin\theta \cos\theta = \sin2\theta$ and $\dfrac{dy}{d\theta} = 2\cos2\theta$;

(iii) $\dfrac{dy}{d\theta} = 3\sin^2 \theta \cos\theta$ and $\dfrac{dy}{d\theta} = 3\cos3\theta$.

\ResetCols{2}
\Item{(2)}  $\theta = 45°$ or $\dfrac{\pi}{4}$ radians.

\Item{(3)}  $\dfrac{dy}{dt} = -n \sin 2\pi nt$.

\ResetCols{2}
\Item{(4)}  $a^x \log_\epsilon a \cos a^x$.

\Item{(5)}  $\dfrac{\cos x}{\sin x} = \cotan x$

\ResetCols{1}
\Item{(6)}  $18.2 \cos \left(x + 26° \right)$.

\Item{(7)} {\loosen The slope is $\dfrac{dy}{d\theta} = 100\cos\left(\theta - 15° \right)$, which is a maximum
 when $(\theta -15°) = 0$, or $\theta = 15°$; the value of the slope
 being then ${}= 100$. When $\theta = 75°$ the slope is
 $100\cos(75°  - 15°) = 100\cos 60° = 100 × \frac{1}{2} = 50$.}

\Item{(8)} $\begin{aligned}[t]
  \cos\theta \sin2\theta + 2\cos2\theta \sin\theta
  &= 2\sin\theta\left(\cos^2 \theta + \cos2\theta\right) \\
  &= 2\sin\theta\left(3\cos^2 \theta - 1\right).
  \end{aligned}$

\Item{(9)} $amn\theta^{n-1} \tan^{m-1}\left(\theta^n\right)\sec^2 \theta^n$.

\Item{(10)} $\epsilon^x \left(\sin^2 x + \sin2x\right)$;\quad $\epsilon^x \left(\sin^2 x + 2\sin2x + 2\cos2x\right)$.

\Item{(11)} $\left(i\right) \dfrac{dy}{dx} = \dfrac{ab}{\left(x + b\right)^2}$;\quad
(ii)~$\dfrac{a}{b} \epsilon^{-\efrac{x}{b}}$;\quad
(iii)~$\dfrac{1}{90}° × \dfrac{ab}{\left(b^2 + x^2\right)}$.

\Item{(12)} (i) $\dfrac{dy}{dx} = \sec x \tan x$;

(ii) $\dfrac{dy}{dx} = - \dfrac{1}{\sqrt{ 1 - x^2}}$;

(iii) $\dfrac{dy}{dx} = \dfrac{1}{ 1 + x^2}$;

(iv) $\dfrac{dy}{dx} = \dfrac{1}{x \sqrt{ x^2 - 1}}$;

(v) $\dfrac{dy}{dx} = \dfrac{\sqrt{ 3\sec x} \left(3\sec^2 x - 1\right)}{2}$.

\Item{(13)} $\dfrac{dy}{d\theta} = 4.6\left(2\theta + 3\right)^{1.3} \cos\left(2\theta + 3\right)^{2.3}$.
\DPPageSep{274.png}{262}%

\Item{(14)}  $\dfrac{dy}{d\theta} = 3\theta^2 + 3\cos \left( \theta + 3 \right) - \log_\epsilon 3 \left( \cos\theta × 3^{\sin\theta} + 3\theta \right)$.

\Item{(15)}  $\theta = \cot\theta; \theta = ±0.86$; is max.~for $+\theta$, min.~for $-\theta$.
\end{Answers}


\begin{Answers}{XV}{(\Pageref{Ex:XV}.)}{}

\Item{(1)}  $x^3 - 6x^2 y - 2y^2;\quad \frac{1}{3} - 2x^3 - 4xy$.

\Item{(2)}  $2xyz + y^2 z + z^2 y + 2xy^2 z^2$;\\
     $2xyz + x^2 z + xz^2 + 2x^2 yz^2$;\\
     $2xyz + x^2 y + xy^2 + 2x^2 y^2 z$.

\Item{(3)}  $\dfrac{1}{r} \{ \left(x - a\right) + \left( y - b \right) + \left( z - c \right) \} = \dfrac{ \left( x + y + z \right) - \left( a + b + c \right) }{r}$; $\dfrac{3}{r}$.

\Item{(4)}  $dy = vu^{v-1}\, du + u^v \log_\epsilon u\, dv$.

\Item{(5)}  $dy = 3\sin v u^2\, du + u^3 \cos v\, dv$,\\
     $dy = u \sin x^{u-1} \cos x\, dx + (\sin x)^u \log_\epsilon \sin x du$,\\
     $dy = \dfrac{1}{v}\, \dfrac{1}{u}\, du - \log_\epsilon u \dfrac{1}{v^2}\, dv$.

\Item{(7)}  Minimum for $x = y = -\frac{1}{2}$.

\Item{(8)}  (\textit{a}) Length $2$~feet, width = depth = $1$~foot, vol.\ = $2$~cubic
 feet.

 (\textit{b}) Radius = $\dfrac{2}{\pi}$ feet = $7.46$~in., length = $2$~feet, vol.\ = $2.54$.

\Item{(9)}   All three parts equal; the product is maximum.

\Item{(10)}  Minimum for $x = y = 1$.

\Item{(11)}  Min.: $x = \frac{1}{2}$ and $y = 2$.

\Item{(12)}  Angle at apex $= 90°$; equal sides = length = $\sqrt[3]{2V}$.
\end{Answers}

%[** TN: Text block-dependent page break]
\begin{Answers}[3]{XVI}{(\Pageref{Ex:XVI}.)}{\newpage}

\Item{(1)}  $1\frac{1}{3}$.

\Item{(2)}  $0.6344$.

\Item{(3)}  $0.2624$.

\ResetCols{1}
\Item{(4)}  (\textit{a}) $y = \frac{1}{8} x^2 + C$;\quad
     (\textit{b}) $y = \sin x + C$.

\Item{(5)}  $y = x^2 + 3x + C$.
\end{Answers}
\DPPageSep{275.png}{263}%


\begin{Answers}[3]{XVII}{(\Pageref{Ex:XVII}.)}{}

\Item{(1)} $\dfrac{4\sqrt{a} x^{\efrac{3}{2}}}{3} + C$.

\Item{(2)} $-\dfrac{1}{x^3} + C$.

\Item{(3)} $\dfrac{x^4}{4a} + C$.

\ResetCols{2}
\Item{(4)} $\tfrac{1}{3} x^3 + ax + C$.

\Item{(5)} $-2x^{-\efrac{5}{2}} + C$.

\ResetCols{2}
\Item{(6)} $x^4 + x^3 + x^2 + x + C$.

\Item{(7)} $\dfrac{ax^2}{4} + \dfrac{bx^3}{9} + \dfrac{cx^4}{16} + C$.

\ResetCols{1}
\Item{(8)} {\loosen $\dfrac{x^2 + a}{x + a} = x - a + \dfrac{a^2 + a}{x + a}$ by division. Therefore the answer
is $\dfrac{x^2}{2} - ax + (a^2 + a)\log_\epsilon (x + a) + C$.}
  (See pages \pageref{intex1} and~\pageref{intex2}.)

\ResetCols{2}
\Item{(9)} $\dfrac{x^4}{4} + 3x^3 + \dfrac{27}{2} x^2 + 27x + C$.

\Item{(10)} $\dfrac{x^3}{3} + \dfrac{2 - a}{2} x^2 - 2ax + C$.

\ResetCols{2}
\Item{(11)} $a^2(2x^{\efrac{3}{2}} + \tfrac{9}{4} x^{\efrac{4}{3}}) + C$.

\Item{(12)} $-\tfrac{1}{3} \cos\theta - \tfrac{1}{6} \theta + C$.

\ResetCols{2}
\Item{(13)} $\dfrac{\theta}{2} + \dfrac{\sin 2a\theta}{4a} + C$.

\Item{(14)} $\dfrac{\theta}{2} - \dfrac{\sin 2\theta}{4} + C$.

\ResetCols{2}
\Item{(15)} $\dfrac{\theta}{2} - \dfrac{\sin 2a\theta}{4a} + C$.

\Item{(16)} $\tfrac{1}{3} \epsilon^{3x}$. % [F1: +C?]

\ResetCols{2}
\Item{(17)} $\log(1 + x) + C$.

\Item{(18)} $-\log_\epsilon (1 - x) + C$.
\end{Answers}


\begin{Answers}{XVIII}{(\Pageref{Ex:XVIII}.)}{}

\Item{(1)} $\text{Area} = 60$; $\text{mean ordinate} = 10$.

\Item{(2)} $\text{Area} = \frac{2}{3}$ of $a × 2a \sqrt{a}$.

\Item{(3)} $\text{Area} = 2$; $\text{mean ordinate} = \dfrac{2}{\pi} = 0.637$.

\Item{(4)} $\text{Area} = 1.57$; $\text{mean ordinate} = 0.5$.

\ResetCols{2}
\Item{(5)} $0.572$, $0.0476$.

\Item{(6)} $\text{Volume} = \pi r^2 \dfrac{h}{3}$.

\ResetCols{2}
\Item{(7)} $1.25$.

\Item{(8)} $79.4$.

\ResetCols{1}
\Item{(9)} $\text{Volume} = 4.9348$; $\text{area of surface} = 12.57$ (from $0$ to~$\pi$).

\Item{(10)} $a\log_\epsilon a$,\quad $\dfrac{a}{a - 1} \log_\epsilon a$.

\Item{(12)} $\text{Arithmetical mean} = 9.5$; $\text{quadratic mean} = 10.85$.
\DPPageSep{276.png}{264}%

\Item{(13)} $\text{Quadratic mean} = \dfrac{1}{\sqrt{2}} \sqrt{A_1^2 + A_3^2}$; $\text{arithmetical mean} = 0$.

The first involves a somewhat difficult integral, and may be
stated thus: By definition the quadratic mean will be
\[
\sqrt{\dfrac{1}{2\pi} \int_0^{2\pi} (A_1 \sin x + A_3 \sin 3x)^2\, dx}. %[** TN: Moved period out of radicand]
\]
Now the integration indicated by
\[
\int (A_1^2 \sin^2 x + 2A_1 A_3 \sin x \sin 3x + A_3^2 \sin^2 3x)\, dx
\]
is more readily obtained if for $\sin^2 x$ we write
\[
\dfrac{1 - \cos 2x}{2}.
\]
For $2\sin x \sin 3x$ we write $\cos 2x - \cos 4x$; and, for $\sin^2 3x$,
\[
\dfrac{1 - \cos 6x}{2}.
\]

Making these substitutions, and integrating, we get (see
\Pageref{cosax})
\[
\dfrac{A_1^2}{2} \left( x - \dfrac{\sin 2x}{2} \right)
 + A_1 A_3 \left( \dfrac{\sin 2x}{2} - \dfrac{\sin 4x}{4} \right)
 + \dfrac{A_3^2}{2} \left( x - \dfrac{\sin 6x}{6} \right).
\]

At the lower limit the substitution of $0$ for~$x$ causes all
this to vanish, whilst at the upper limit the substitution
of $2\pi$ for~$x$ gives $A_1^2 \pi + A_3^2 \pi$. And hence the
answer follows.

\Item{(14)} Area is $62.6$~square units. Mean ordinate is $10.42$.

\Item{(16)} $436.3$. (This solid is pear shaped.)
\end{Answers}


\begin{Answers}[2]{XIX}{(\Pageref{Ex:XIX}.)}{}

\Item{(1)} $\dfrac{x\sqrt{a^2 - x^2}}{2} + \dfrac{a^2}{2} \sin^{-1} \dfrac{x}{a} + C$.

\Item{(2)} $\dfrac{x^2}{2}(\log_\epsilon x - \tfrac{1}{2}) + C$.

\ResetCols{2}
\Item{(3)} $\dfrac{x^{a+1}}{a + 1} \left(\log_\epsilon x - \dfrac{1}{a + 1}\right) + C$.

\Item{(4)} $\sin \DPtypo{e}{\epsilon}^x + C$.

\ResetCols{2}
\Item{(5)} $\sin(\log_\epsilon x) + C$.

\Item{(6)} $\DPtypo{e}{\epsilon}^x (x^2 - 2x + 2) + C$.

\ResetCols{2}
\Item{(7)} $\dfrac{1}{a + 1} (\log_\epsilon x)^{a+1} + C$.
\DPPageSep{277.png}{265}%

\Item{(8)} $\log_\epsilon(\log_\epsilon x) + C$.

\ResetCols{1}
\Item{(9)} $2\log_\epsilon(x - 1) + 3\log_\epsilon(x + 2) + C$.

\Item{(10)} $\frac{1}{2} \log_\epsilon(x - 1) + \frac{1}{5} \log_\epsilon(x - 2) + \frac{3}{10} \log_\epsilon(x + 3) + C$.

\ResetCols{2}
\Item{(11)} $\dfrac{b}{2a} \log_\epsilon \dfrac{x - a}{x + a} + C$.

\Item{(12)} $\log_\epsilon \dfrac{x^2 - 1}{x^2 + 1} + C$.

\ResetCols{1}
\Item{(13)} $\frac{1}{4} \log_\epsilon \dfrac{1 + x}{1 - x} + \frac{1}{2} \arctan x + C$.

\Item{(14)} $\dfrac{1}{\sqrt{a}} \log_\epsilon \dfrac{\sqrt{a} - \sqrt{a - bx^2}}{x\sqrt{a}}$.\quad (Let $\dfrac{1}{x} = v$; then, in the result,
let $\sqrt{v^2 - \dfrac{b}{a}} = v - u$.) %[** TN: Large () in the original]

You had better differentiate now the answer and work
back to the given expression as a check.
\end{Answers}

Every earnest student is exhorted to manufacture more
examples for himself at every stage, so as to test his powers.
When integrating he can always test his answer by differentiating
it, to see whether he gets back the expression from
which he started.

There are lots of books which give examples for practice.
It will suffice here to name two: R.~G.~Blaine's \textit{The Calculus
and its Applications}, and F.~M.~Saxelby's \textit{A Course in Practical
Mathematics}.

\vfill
\begin{center}
  \tiny GLASGOW: PRINTED AT THE UNIVERSITY PRESS BY ROBERT MACLEHOSE AND CO. LTD.
\end{center}
\DPPageSep{278.png}{266}%
% [Blank Page]
\DPPageSep{279.png}{267}%


% [** TN: Macro prints the text below]
%A SELECTION OF
%Mathematical Works
\Catalogue

\Book{\Title{An Introduction to the Calculus.}  Based on
Graphical Methods. By Prof.\ \Author{G.~A.~Gibson}, M.A., LL.D\@.
3s.~6d.}

\Book{\Title{An Elementary Treatise on the Calculus.} With
Illustrations from Geometry, Mechanics, and Physics.
By Prof.\ \Author{G.~A.~Gibson}, M.A., LL.D\@. 7s.~6d.}

\Book{\Title{Differential Calculus for Beginners.}  By
\Author{J.~Edwards}, M.A\@. 4s.~6d.}

\Book{\Title{Integral Calculus for Beginners.} With an Introduction
to the Study of Differential Equations. By
\Author{Joseph Edwards}, M.A\@. 4s.~6d.}

\Book{\Title{Calculus Made Easy.} Being a very-simplest
Introduction to those beautiful Methods of Reckoning
which are generally called by the terrifying names of the
Differential Calculus and the Integral Calculus. By \Author{F.~R.~S\@.}
2s.~net. New Edition, with many Examples.}

\Book{\Title{A First Course in the Differential and Integral
Calculus.} By Prof.\ \Author{W.~F.~Osgood}, Ph.D\@. 8s.~6d.\ net.}

\Book{\Title{Practical Integration for the use of Engineers,
etc.} By \Author{A.~S.~Percival}, M.A\@. 2s.~6d.\ net.}

\Book{\Title{Differential Calculus.} With Applications and
numerous Examples. An Elementary Treatise by \Author{Joseph
Edwards}, M.A\@. 14s.}

\Book{\Title{Differential and Integral Calculus for Technical
Schools and Colleges.} By \Author{P.~A.~Lambert}, M.A\@. 7s.~6d.}

\Book{\Title{Differential and Integral Calculus. With Applications.}
By Sir \Author{A.~G.~Greenhill}, F.R.S\@. 10s.~6d.}
%\vfill
%\begin{center}LONDON: MACMILLAN AND CO., LTD.\end{center}
\DPPageSep{280.png}{268}%

\Book{\Title{A Treatise on the Integral Calculus and its
Applications.} By \Author{I.~Todhunter}, F.R.S\@. 10s.~6d. Key.
By \Author{H.~St.~J.~Hunter}, M.A\@. 10s.~6d.}

\Book{\Title{A Treatise on the Differential Calculus and the
Elements of the Integral Calculus.} With numerous Examples.
By \Author{I.~Todhunter}, F.R.S\@. 10s.~6d. Key. By
\Author{H.~St.~J.~Hunter}, M.A\@. 10s.~6d.}

\Book{\Title{Ordinary Differential Equations. An Elementary
Text-book.} By \Author{James Morris Page}, Ph.D\@. 6s.~6d.}

\Book{\Title{An Introduction to the Modern Theory of
Equations.} By Prof.\ \Author{F.~Cajori}, Ph.D\@. 7s.~6d.\ net.}

\Book{\Title{A Treatise on Differential Equations.} By
\Author{Andrew Russell Forsyth}, Sc.D., LL.D\@. Fourth
Edition. 14s.\ net.}

\Book{\Title{A Short Course on Differential Equations.} By
Prof.\ \Author{Donald~F. Campbell}, Ph.D\@. 4s.\ net.}

\Book{\Title{A Manual of Quaternions.} By \Author{C.~J.~Joly},
M.A., D.Sc., F.R.S\@. 10s.\ net.}

\Book{\Title{The Theory of Determinants in the Historical
Order of Development.} Vol.~I\@. Part~I. General Determinants,
up to~1841. Part~II. Special Determinants, up
to~1841. 17s.\ net. Vol.~II\@. The Period 1841 to~1860. 17s.\
net. By \Author{T.~Muir}, M.A., LL.D., F.R.S.}

\Book{\Title{An Introduction to the Theory of Infinite Series.}
By \Author{T.~J.~I'a Bromwich},\DPnote{** [sic]; Thomas John I'Anson Bromwich}
 M.A., F.R.S\@. 15s.\ net.}

\Book{\Title{Introduction to the Theory of Fourier's Series
and Integrals, and the Mathematical Theory of the Conduction
of Heat.} By Prof.\ \Author{H.~S.~Carslaw}, M.A., D.Sc.,
F.R.S.E\@. 14s.\ net.}

\vfill

\begin{center}
  LONDON: MACMILLAN AND CO., LTD.
\end{center}
\DPPageSep{281.png}{269}%
%[Blank Page]


%%%%%%%%%%%%%%%%%%%%%%%%% GUTENBERG LICENSE %%%%%%%%%%%%%%%%%%%%%%%%%%
\flushpage\null\vfill
\TNote{\Corrections}

\License

\begin{PGtext}
*** END OF THE PROJECT GUTENBERG EBOOK CALCULUS MADE EASY ***

***** This file should be named 33283-pdf.pdf or 33283-pdf.zip *****
This and all associated files of various formats will be found in:
    https://www.gutenberg.org/3/3/2/8/33283/

Updated editions will replace the previous one--the old editions will
be renamed.

Creating the works from print editions not protected by U.S. copyright
law means that no one owns a United States copyright in these works,
so the Foundation (and you!) can copy and distribute it in the
United States without permission and without paying copyright
royalties. Special rules, set forth in the General Terms of Use part
of this license, apply to copying and distributing Project
Gutenberg-tm electronic works to protect the PROJECT GUTENBERG-tm
concept and trademark. Project Gutenberg is a registered trademark,
and may not be used if you charge for an eBook, except by following
the terms of the trademark license, including paying royalties for use
of the Project Gutenberg trademark. If you do not charge anything for
copies of this eBook, complying with the trademark license is very
easy. You may use this eBook for nearly any purpose such as creation
of derivative works, reports, performances and research. Project
Gutenberg eBooks may be modified and printed and given away--you may
do practically ANYTHING in the United States with eBooks not protected
by U.S. copyright law. Redistribution is subject to the trademark
license, especially commercial redistribution.

START: FULL LICENSE

THE FULL PROJECT GUTENBERG LICENSE
PLEASE READ THIS BEFORE YOU DISTRIBUTE OR USE THIS WORK

To protect the Project Gutenberg-tm mission of promoting the free
distribution of electronic works, by using or distributing this work
(or any other work associated in any way with the phrase `Project
Gutenberg'), you agree to comply with all the terms of the Full
Project Gutenberg-tm License available with this file or online at
www.gutenberg.org/license.

Section 1. General Terms of Use and Redistributing Project
Gutenberg-tm electronic works

1.A. By reading or using any part of this Project Gutenberg-tm
electronic work, you indicate that you have read, understand, agree to
and accept all the terms of this license and intellectual property
(trademark/copyright) agreement. If you do not agree to abide by all
the terms of this agreement, you must cease using and return or
destroy all copies of Project Gutenberg-tm electronic works in your
possession. If you paid a fee for obtaining a copy of or access to a
Project Gutenberg-tm electronic work and you do not agree to be bound
by the terms of this agreement, you may obtain a refund from the
person or entity to whom you paid the fee as set forth in paragraph
1.E.8.

1.B. `Project Gutenberg' is a registered trademark. It may only be
used on or associated in any way with an electronic work by people who
agree to be bound by the terms of this agreement. There are a few
things that you can do with most Project Gutenberg-tm electronic works
even without complying with the full terms of this agreement. See
paragraph 1.C below. There are a lot of things you can do with Project
Gutenberg-tm electronic works if you follow the terms of this
agreement and help preserve free future access to Project Gutenberg-tm
electronic works. See paragraph 1.E below.

1.C. The Project Gutenberg Literary Archive Foundation (`the
Foundation' or PGLAF), owns a compilation copyright in the collection
of Project Gutenberg-tm electronic works. Nearly all the individual
works in the collection are in the public domain in the United
States. If an individual work is unprotected by copyright law in the
United States and you are located in the United States, we do not
claim a right to prevent you from copying, distributing, performing,
displaying or creating derivative works based on the work as long as
all references to Project Gutenberg are removed. Of course, we hope
that you will support the Project Gutenberg-tm mission of promoting
free access to electronic works by freely sharing Project Gutenberg-tm
works in compliance with the terms of this agreement for keeping the
Project Gutenberg-tm name associated with the work. You can easily
comply with the terms of this agreement by keeping this work in the
same format with its attached full Project Gutenberg-tm License when
you share it without charge with others.

1.D. The copyright laws of the place where you are located also govern
what you can do with this work. Copyright laws in most countries are
in a constant state of change. If you are outside the United States,
check the laws of your country in addition to the terms of this
agreement before downloading, copying, displaying, performing,
distributing or creating derivative works based on this work or any
other Project Gutenberg-tm work. The Foundation makes no
representations concerning the copyright status of any work in any
country other than the United States.

1.E. Unless you have removed all references to Project Gutenberg:

1.E.1. The following sentence, with active links to, or other
immediate access to, the full Project Gutenberg-tm License must appear
prominently whenever any copy of a Project Gutenberg-tm work (any work
on which the phrase `Project Gutenberg' appears, or with which the
phrase `Project Gutenberg' is associated) is accessed, displayed,
performed, viewed, copied or distributed:

  This eBook is for the use of anyone anywhere in the United States and
  most other parts of the world at no cost and with almost no
  restrictions whatsoever. You may copy it, give it away or re-use it
  under the terms of the Project Gutenberg License included with this
  eBook or online at www.gutenberg.org. If you are not located in the
  United States, you will have to check the laws of the country where
  you are located before using this eBook.

1.E.2. If an individual Project Gutenberg-tm electronic work is
derived from texts not protected by U.S. copyright law (does not
contain a notice indicating that it is posted with permission of the
copyright holder), the work can be copied and distributed to anyone in
the United States without paying any fees or charges. If you are
redistributing or providing access to a work with the phrase `Project
Gutenberg' associated with or appearing on the work, you must comply
either with the requirements of paragraphs 1.E.1 through 1.E.7 or
obtain permission for the use of the work and the Project Gutenberg-tm
trademark as set forth in paragraphs 1.E.8 or 1.E.9.

1.E.3. If an individual Project Gutenberg-tm electronic work is posted
with the permission of the copyright holder, your use and distribution
must comply with both paragraphs 1.E.1 through 1.E.7 and any
additional terms imposed by the copyright holder. Additional terms
will be linked to the Project Gutenberg-tm License for all works
posted with the permission of the copyright holder found at the
beginning of this work.

1.E.4. Do not unlink or detach or remove the full Project Gutenberg-tm
License terms from this work, or any files containing a part of this
work or any other work associated with Project Gutenberg-tm.

1.E.5. Do not copy, display, perform, distribute or redistribute this
electronic work, or any part of this electronic work, without
prominently displaying the sentence set forth in paragraph 1.E.1 with
active links or immediate access to the full terms of the Project
Gutenberg-tm License.

1.E.6. You may convert to and distribute this work in any binary,
compressed, marked up, nonproprietary or proprietary form, including
any word processing or hypertext form. However, if you provide access
to or distribute copies of a Project Gutenberg-tm work in a format
other than `Plain Vanilla ASCII' or other format used in the official
version posted on the official Project Gutenberg-tm website
(www.gutenberg.org), you must, at no additional cost, fee or expense
to the user, provide a copy, a means of exporting a copy, or a means
of obtaining a copy upon request, of the work in its original `Plain
Vanilla ASCII' or other form. Any alternate format must include the
full Project Gutenberg-tm License as specified in paragraph 1.E.1.

1.E.7. Do not charge a fee for access to, viewing, displaying,
performing, copying or distributing any Project Gutenberg-tm works
unless you comply with paragraph 1.E.8 or 1.E.9.

1.E.8. You may charge a reasonable fee for copies of or providing
access to or distributing Project Gutenberg-tm electronic works
provided that:

* You pay a royalty fee of 20% of the gross profits you derive from
  the use of Project Gutenberg-tm works calculated using the method
  you already use to calculate your applicable taxes. The fee is owed
  to the owner of the Project Gutenberg-tm trademark, but he has
  agreed to donate royalties under this paragraph to the Project
  Gutenberg Literary Archive Foundation. Royalty payments must be paid
  within 60 days following each date on which you prepare (or are
  legally required to prepare) your periodic tax returns. Royalty
  payments should be clearly marked as such and sent to the Project
  Gutenberg Literary Archive Foundation at the address specified in
  Section 4, `Information about donations to the Project Gutenberg
  Literary Archive Foundation.'

* You provide a full refund of any money paid by a user who notifies
  you in writing (or by email) within 30 days of receipt that s/he
  does not agree to the terms of the full Project Gutenberg-tm
  License. You must require such a user to return or destroy all
  copies of the works possessed in a physical medium and discontinue
  all use of and all access to other copies of Project Gutenberg-tm
  works.

* You provide, in accordance with paragraph 1.F.3, a full refund of
  any money paid for a work or a replacement copy, if a defect in the
  electronic work is discovered and reported to you within 90 days of
  receipt of the work.

* You comply with all other terms of this agreement for free
  distribution of Project Gutenberg-tm works.

1.E.9. If you wish to charge a fee or distribute a Project
Gutenberg-tm electronic work or group of works on different terms than
are set forth in this agreement, you must obtain permission in writing
from the Project Gutenberg Literary Archive Foundation, the manager of
the Project Gutenberg-tm trademark. Contact the Foundation as set
forth in Section 3 below.

1.F.

1.F.1. Project Gutenberg volunteers and employees expend considerable
effort to identify, do copyright research on, transcribe and proofread
works not protected by U.S. copyright law in creating the Project
Gutenberg-tm collection. Despite these efforts, Project Gutenberg-tm
electronic works, and the medium on which they may be stored, may
contain `Defects,' such as, but not limited to, incomplete, inaccurate
or corrupt data, transcription errors, a copyright or other
intellectual property infringement, a defective or damaged disk or
other medium, a computer virus, or computer codes that damage or
cannot be read by your equipment.

1.F.2. LIMITED WARRANTY, DISCLAIMER OF DAMAGES - Except for the `Right
of Replacement or Refund' described in paragraph 1.F.3, the Project
Gutenberg Literary Archive Foundation, the owner of the Project
Gutenberg-tm trademark, and any other party distributing a Project
Gutenberg-tm electronic work under this agreement, disclaim all
liability to you for damages, costs and expenses, including legal
fees. YOU AGREE THAT YOU HAVE NO REMEDIES FOR NEGLIGENCE, STRICT
LIABILITY, BREACH OF WARRANTY OR BREACH OF CONTRACT EXCEPT THOSE
PROVIDED IN PARAGRAPH 1.F.3. YOU AGREE THAT THE FOUNDATION, THE
TRADEMARK OWNER, AND ANY DISTRIBUTOR UNDER THIS AGREEMENT WILL NOT BE
LIABLE TO YOU FOR ACTUAL, DIRECT, INDIRECT, CONSEQUENTIAL, PUNITIVE OR
INCIDENTAL DAMAGES EVEN IF YOU GIVE NOTICE OF THE POSSIBILITY OF SUCH
DAMAGE.

1.F.3. LIMITED RIGHT OF REPLACEMENT OR REFUND - If you discover a
defect in this electronic work within 90 days of receiving it, you can
receive a refund of the money (if any) you paid for it by sending a
written explanation to the person you received the work from. If you
received the work on a physical medium, you must return the medium
with your written explanation. The person or entity that provided you
with the defective work may elect to provide a replacement copy in
lieu of a refund. If you received the work electronically, the person
or entity providing it to you may choose to give you a second
opportunity to receive the work electronically in lieu of a refund. If
the second copy is also defective, you may demand a refund in writing
without further opportunities to fix the problem.

1.F.4. Except for the limited right of replacement or refund set forth
in paragraph 1.F.3, this work is provided to you--`AS-IS', WITH NO
OTHER WARRANTIES OF ANY KIND, EXPRESS OR IMPLIED, INCLUDING BUT NOT
LIMITED TO WARRANTIES OF MERCHANTABILITY OR FITNESS FOR ANY PURPOSE.

1.F.5. Some states do not allow disclaimers of certain implied
warranties or the exclusion or limitation of certain types of
damages. If any disclaimer or limitation set forth in this agreement
violates the law of the state applicable to this agreement, the
agreement shall be interpreted to make the maximum disclaimer or
limitation permitted by the applicable state law. The invalidity or
unenforceability of any provision of this agreement shall not void the
remaining provisions.

1.F.6. INDEMNITY - You agree to indemnify and hold the Foundation, the
trademark owner, any agent or employee of the Foundation, anyone
providing copies of Project Gutenberg-tm electronic works in
accordance with this agreement, and any volunteers associated with the
production, promotion and distribution of Project Gutenberg-tm
electronic works, harmless from all liability, costs and expenses,
including legal fees, that arise directly or indirectly from any of
the following which you do or cause to occur: (a) distribution of this
or any Project Gutenberg-tm work, (b) alteration, modification, or
additions or deletions to any Project Gutenberg-tm work, and (c) any
Defect you cause.

Section 2. Information about the Mission of Project Gutenberg-tm

Project Gutenberg-tm is synonymous with the free distribution of
electronic works in formats readable by the widest variety of
computers including obsolete, old, middle-aged and new computers. It
exists because of the efforts of hundreds of volunteers and donations
from people in all walks of life.

Volunteers and financial support to provide volunteers with the
assistance they need are critical to reaching Project Gutenberg-tm's
goals and ensuring that the Project Gutenberg-tm collection will
remain freely available for generations to come. In 2001, the Project
Gutenberg Literary Archive Foundation was created to provide a secure
and permanent future for Project Gutenberg-tm and future
generations. To learn more about the Project Gutenberg Literary
Archive Foundation and how your efforts and donations can help, see
Sections 3 and 4 and the Foundation information page at
www.gutenberg.org

Section 3. Information about the Project Gutenberg Literary
Archive Foundation

The Project Gutenberg Literary Archive Foundation is a non-profit
501(c)(3) educational corporation organized under the laws of the
state of Mississippi and granted tax exempt status by the Internal
Revenue Service. The Foundation's EIN or federal tax identification
number is 64-6221541. Contributions to the Project Gutenberg Literary
Archive Foundation are tax deductible to the full extent permitted by
U.S. federal laws and your state's laws.

The Foundation's business office is located at 809 North 1500 West,
Salt Lake City, UT 84116, (801) 596-1887. Email contact links and up
to date contact information can be found at the Foundation's website
and official page at www.gutenberg.org/contact

Section 4. Information about Donations to the Project Gutenberg
Literary Archive Foundation

Project Gutenberg-tm depends upon and cannot survive without
widespread public support and donations to carry out its mission of
increasing the number of public domain and licensed works that can be
freely distributed in machine-readable form accessible by the widest
array of equipment including outdated equipment. Many small donations
($1 to $5,000) are particularly important to maintaining tax exempt
status with the IRS.

The Foundation is committed to complying with the laws regulating
charities and charitable donations in all 50 states of the United
States. Compliance requirements are not uniform and it takes a
considerable effort, much paperwork and many fees to meet and keep up
with these requirements. We do not solicit donations in locations
where we have not received written confirmation of compliance. To SEND
DONATIONS or determine the status of compliance for any particular
state visit www.gutenberg.org/donate

While we cannot and do not solicit contributions from states where we
have not met the solicitation requirements, we know of no prohibition
against accepting unsolicited donations from donors in such states who
approach us with offers to donate.

International donations are gratefully accepted, but we cannot make
any statements concerning tax treatment of donations received from
outside the United States. U.S. laws alone swamp our small staff.

Please check the Project Gutenberg web pages for current donation
methods and addresses. Donations are accepted in a number of other
ways including checks, online payments and credit card donations. To
donate, please visit: www.gutenberg.org/donate

Section 5. General Information About Project Gutenberg-tm electronic works

Professor Michael S. Hart was the originator of the Project
Gutenberg-tm concept of a library of electronic works that could be
freely shared with anyone. For forty years, he produced and
distributed Project Gutenberg-tm eBooks with only a loose network of
volunteer support.

Project Gutenberg-tm eBooks are often created from several printed
editions, all of which are confirmed as not protected by copyright in
the U.S. unless a copyright notice is included. Thus, we do not
necessarily keep eBooks in compliance with any particular paper
edition.

Most people start at our website which has the main PG search
facility: www.gutenberg.org

This website includes information about Project Gutenberg-tm,
including how to make donations to the Project Gutenberg Literary
Archive Foundation, how to help produce our new eBooks, and how to
subscribe to our email newsletter to hear about new eBooks.
\end{PGtext}
\end{document}
